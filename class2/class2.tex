% ! TEX root = ../report.tex
%

\documentclass[a4paper]{article}
\usepackage{ctex}       
% 修正:geometry参数head→top(head是页眉高度,此处应为上边距)
\usepackage[left=1.5cm,right=1.5cm,top=1.5cm,bottom=1.5cm]{geometry}
\usepackage{graphicx}   
\usepackage{listings}
\usepackage{xcolor}
\usepackage{booktabs}
\usepackage{subcaption}
\usepackage{caption}
\usepackage{multirow} 
% 移除冲突包:subcaption与subfigure冲突,且subfigure是旧包,此处用subcaption足够
% \usepackage{subfigure}
% 修正:hyperref放在最后加载,避免与其他包冲突
\usepackage{hyperref}
\usepackage{float}


\begin{document}

\section{阶跃响应和冲激响应}
\subsection{实验目的}
\begin{itemize}
  \item 观察和测量RLC串联电路的阶跃响应与冲激响应的波形和有关参数,并研究其电路元件参数变化对响应的影响。
  \item 掌握有关信号时域的测量分析方法。
\end{itemize}

\subsection{实验原理}
\subsubsection{}
阶跃响应与阶跃激励的关系简单的表示为:
\[g(t)=H[u(t)] 或者 u(t) \rightarrow g(t)\]
\subsubsection{}
RLC串联电路的阶跃响应与冲激响应实验其响应有以下三种状态:
\begin{itemize}
  \item 欠阻尼状态:$R^2<\frac{4L}{C}$,响应波形为振荡衰减波形。
  \item 临界阻尼状态:$R^2=\frac{4L}{C}$,响应波形为最快的非振荡衰减波形。
  \item 过阻尼状态:$R^2>\frac{4L}{C}$,响应波形为非振荡衰减波形。
\end{itemize}
\subsubsection{}
相应的动态指标定义如下:
\begin{itemize}
    % 修正:闭合数学环境,文字移出$外
    \item 上升时间 $t_r$:$y(t)$从$0$到第一次上升到稳态值$y(\infty)$的时间。
    \item 峰值时间 $t_p$:$y(t)$达到第一次峰值的时间。
    \item 调节时间 $t_s$:$y(t)$进入并保持在稳态值$y(\infty)$的$5\%$范围内所需的时间。
    \item 最大超调量 $\delta_p$:$y(t)$达到的最大峰值与稳态值之差与稳态值之比,通常以百分数表示。也即
    % 修正:超调量公式逻辑(100%不应在分母)+ 规范下标
    \[\delta_p=\frac{y_{max}-y(\infty)}{y(\infty)} \times 100\%\]
\end{itemize}
\subsubsection{}
在满足条件的情况下,RC电路可以对输入信号进行微分,微分电路如图\ref{fig:dif_cir},微分电路响应如图\ref{fig:dif_echo}。

\subsection{实验器材}
\begin{itemize}
    \item 信号源及频率计模块——一块
    \item 一阶网络模块——一块
    \item 数字万用表——一台
    \item 双踪示波器——一台
\end{itemize}

\subsection{实验步骤与数据记录}

\subsubsection{任务一:阶跃响应实验波形观察与参数测量}
设激励信号为方波,频率为500Hz。模块关电,进行连线。
\begin{enumerate}
    \item 模块开电,调整激励信号源为方波(即从模块S2中的P2端口引出方波信号);调节频率调节旋钮ROL1,使频率计示数\( f=500\mathrm{Hz} \)。
    \item 连接模块S2的方波信号输出端P2至模块S5中的P12。
    \item 示波器CH1接于TP14,调整W1,使电路分别工作于欠阻尼、临界和过阻尼三种状态,观察各种状态下的输出波形,用万用表测量与波形对应的P12和P13两点间的电阻值(测量时应断开电源),并将实验数据填入表\ref{tab:step_response}中。
    \item TP12为输入信号波形的测量点,可把示波器的CH2接于TP12上,便于波形比较。
\end{enumerate}

\begin{table}[htbp]
  \centering
  \caption{阶跃响应实验记录表}
  \label{tab:step_response}
  \begin{tabular*}{\columnwidth}{@{\extracolsep{\fill}}lccc@{}}
    \toprule
    % 修正:multirow语法(行数+宽度+内容)
    \multirow{2}{*}{参数测量} & \multicolumn{3}{c}{状态} \\
    \cmidrule{2-4}
     & 欠阻尼状态\( R<2\sqrt{\frac{L}{C}} \) & 临界状态\( R=2\sqrt{\frac{L}{C}} \) & 过阻尼状态\( R>2\sqrt{\frac{L}{C}} \) \\
    \midrule
    \( R= \) &  &  &  \\ 
    TP12激励波形 &  &  &  \\
    TP14响应波形 &  &  &  \\
    \bottomrule
  \end{tabular*}
  \vspace{0.5em}
  \small 注:描绘波形要使三种状态的X轴坐标(扫描时间)一致。
\end{table}

\subsubsection{任务二:冲激响应的波形观察}
冲激信号是由阶跃信号经过微分电路而得到。模块关电,进行连线。
\begin{enumerate}
    \item 将方波输入信号接于P10(输入信号频率与幅度不变)。
    \item 连接P11与P12。
    \item 将示波器的CH1接于TP11,观察经微分后响应波形(等效为冲激激励信号)。
    \item 将示波器的CH2接于TP14,调整W1,使电路分别工作于欠阻尼、临界和过阻尼三种状态。
    \item 观察电路处于以上三种状态时激励信号与响应信号的波形,并将观察结果填入表\ref{tab:impulse_response}中。
\end{enumerate}

\begin{table}[htbp]
  \centering
  \caption{冲激响应实验记录表}
  \label{tab:impulse_response}
  \begin{tabular*}{\columnwidth}{@{\extracolsep{\fill}}lccc@{}}
    \toprule
    % 修正:multirow语法(行数+宽度+内容)
    \multirow{2}{*}{参数测量} & \multicolumn{3}{c}{状态} \\
    \cmidrule{2-4}
     & 欠阻尼状态\( R<2\sqrt{\frac{L}{C}} \) & 临界状态\( R=2\sqrt{\frac{L}{C}} \) & 过阻尼状态\( R>2\sqrt{\frac{L}{C}} \) \\
    \midrule
    \( R= \) &  &  &  \\
    TP11激励波形 &  &  &  \\
    TP14响应波形 &  &  &  \\
    \bottomrule
  \end{tabular*}
\end{table}

\subsection{实验结果及结论}

阶跃响应的波形如图\ref{fig:sqr_cir}所示,冲激响应的波形如图\ref{fig:imp_cir}所示。
\begin{figure}
  \caption{方波输入(阶跃响应)的波形观察}
  % 修正:subfigure必须指定宽度参数
  \begin{subfigure}{0.3\textwidth}
    \caption{欠阻尼状态}
    \includegraphics[width=\linewidth]{./sqr1.jpg}
  \end{subfigure}
  \begin{subfigure}{0.3\textwidth}
    \caption{临界阻尼状态}
    \includegraphics[width=\linewidth]{./sqr2.jpg}
  \end{subfigure}
  \begin{subfigure}{0.3\textwidth}
    \caption{过阻尼状态}
    \includegraphics[width=\linewidth]{./sqr3.jpg}
  \end{subfigure}
  \label{fig:sqr_cir}
\end{figure}

\begin{figure}
  \caption{脉冲输入(冲激响应)的波形观察}
  % 修正:subfigure必须指定宽度参数
  \begin{subfigure}{0.3\textwidth}
    \caption{欠阻尼状态}
    \includegraphics[width=\linewidth]{./imp1}
  \end{subfigure}
  \begin{subfigure}{0.3\textwidth}
    \caption{临界阻尼状态}
    \includegraphics[width=\linewidth]{./imp2.jpg}
  \end{subfigure}
  \begin{subfigure}{0.3\textwidth}
    \caption{过阻尼状态}
    \includegraphics[width=\linewidth]{./imp3.jpg}
  \end{subfigure}
  \label{fig:imp_cir}
\end{figure}

根据波形可填表\ref{tab:step_response}和表\ref{tab:impulse_response}如下:
\begin{table}[h!]
  \centering
  \caption{阶跃响应实验记录表(填写结果)}
  \label{tab:step_response_filled}
  \begin{tabular*}{\columnwidth}{@{\extracolsep{\fill}}lccc@{}}
    \toprule
    \multirow{2}{*}{参数测量} & \multicolumn{3}{c}{状态} \\
    \cmidrule{2-4}
     & 欠阻尼状态\( R<2\sqrt{\frac{L}{C}} \) & 临界状态\( R=2\sqrt{\frac{L}{C}} \) & 过阻尼状态\( R>2\sqrt{\frac{L}{C}} \) \\
    \midrule
    \( R= \) &  &  &  \\ 
    TP12激励波形 & 方波 & 方波 & 方波 \\
    TP14响应波形 & 振荡衰减波形 & 最快非振荡衰减波形 & 非振荡衰减波形 \\
    \bottomrule
  \end{tabular*}
\end{table}

\begin{table}[h!]
  \centering
  \caption{冲激响应实验记录表(填写结果)}
  \label{tab:impulse_response_filled}
  \begin{tabular*}{\columnwidth}{@{\extracolsep{\fill}}lccc@{}}
    \toprule
    \multirow{2}{*}{参数测量} & \multicolumn{3}{c}{状态} \\
    \cmidrule{2-4}
     & 欠阻尼状态\( R<2\sqrt{\frac{L}{C}} \) & 临界状态\( R=2\sqrt{\frac{L}{C}} \) & 过阻尼状态\( R>2\sqrt{\frac{L}{C}} \) \\
    \midrule
    \( R= \) &  &  &  \\
    TP11激励波形 & 冲激波形 & 冲激波形 & 冲激波形 \\
    TP14响应波形 & 振荡衰减波形 & 最快非振荡衰减波形 & 非振荡衰减波形 \\
    \bottomrule
  \end{tabular*}
\end{table}
\subsection{实验程序及运行结果}
\subsubsection{问题一}


\begin{lstlisting}[language=Matlab, 
                  caption={信号波形生成与绘图代码}, 
                        % 指定代码语言(此处为MATLAB)
  basicstyle=\ttfamily\small,  % 等宽字体+小字号(Markdown常用)
  backgroundcolor=\color{gray!5},  % 浅灰背景(接近多数Markdown渲染效果)
  frame=none,           % 无边框(Markdown代码块通常无框)
  keywordstyle=\color{blue},  % 关键字高亮(蓝色,可选)
  commentstyle=\color{green},  % 注释高亮(绿色,可选)
  showstringspaces=false,  % 不显示字符串中的空格标记
  numbers=none,         % 不显示行号(Markdown默认无行号)
  breaklines=true,      % 自动换行(避免溢出)
  columns=fullflexible  ]        

t = sym('t');

ut = heaviside(t);

f1 = (2 - exp(-2*t)) .* ut;

f2 = cos(pi * t /2)* (heaviside(t) - heaviside(t-4));

f3 = exp(t)* cos(t)* ut;

f4 = 2/3 *t *heaviside(t +2);

figure;

subplot(2,2,1);
fplot(f1);
title("signal 1");

subplot(2,2,2);
fplot(f2);
title("signal 2");

subplot(2,2,3);
fplot(f3);
title("signal 3");

subplot(2,2,4);
fplot(f4);
title("signal 4");

saveas(gcf, './p1_signals.png');
\end{lstlisting}

运行结果如下图\ref{fig:p1}

\begin{figure}[H]
  \centering
  \includegraphics[width=0.8\textwidth]{./matlab/p1_signals.png}
  \caption{问题一信号波形}
  \label{fig:p1}
\end{figure}

\subsubsection{问题二}

\begin{lstlisting}[language=Matlab, 
                  caption={信号波形生成与绘图代码}, 
                        % 指定代码语言(此处为MATLAB)
  basicstyle=\ttfamily\small,  % 等宽字体+小字号(Markdown常用)
  backgroundcolor=\color{gray!5},  % 浅灰背景(接近多数Markdown渲染效果)
  frame=none,           % 无边框(Markdown代码块通常无框)
  keywordstyle=\color{blue},  % 关键字高亮(蓝色,可选)
  commentstyle=\color{green},  % 注释高亮(绿色,可选)
  showstringspaces=false,  % 不显示字符串中的空格标记
  numbers=none,         % 不显示行号(Markdown默认无行号)
  breaklines=true,      % 自动换行(避免溢出)
  columns=fullflexible  ]  

t = sym('t');

f = 2*heaviside(t) - 2*heaviside(t-1) + heaviside (t-1) - heaviside(t-2);

f1 = subs(f, t, -t);

f2 = subs(f, t, t-2);

f3 = subs(f, t, 0.5*t);
f31 = subs(f, t, 2*t);

f4 = subs(f, t, 0.5*t +1);

figure;

subplot(2,2,1);
fplot(f1);
title("f(-t)");

subplot(2,2,2);
fplot(f2);
title("f(t-2)");

subplot(2,2,3);
fplot(f3);
hold on;
fplot(f31);
title("f(0.5t) and f(2t)");
hold off;

subplot(2,2,4);
fplot(f4);
title("f(0.5t +1)");

saveas(gcf, './p2_signals.png');

\end{lstlisting}



运行结果如下图\ref{fig:p2}
\begin{figure}[H]
  \centering
  \includegraphics[width=0.8\textwidth]{./matlab/p2_signals.png}
  \caption{问题二信号波形}
  \label{fig:p2}
\end{figure}


\subsubsection{问题三}

\begin{lstlisting}[language=Matlab, 
                  caption={信号波形生成与绘图代码}, 
                        % 指定代码语言(此处为MATLAB)
  basicstyle=\ttfamily\small,  % 等宽字体+小字号(Markdown常用)
  backgroundcolor=\color{gray!5},  % 浅灰背景(接近多数Markdown渲染效果)
  frame=none,           % 无边框(Markdown代码块通常无框)
  keywordstyle=\color{blue},  % 关键字高亮(蓝色,可选)
  commentstyle=\color{green},  % 注释高亮(绿色,可选)
  showstringspaces=false,  % 不显示字符串中的空格标记
  numbers=none,         % 不显示行号(Markdown默认无行号)
  breaklines=true,      % 自动换行(避免溢出)
  columns=fullflexible  ]  

t= sym('t');

ft = 2* exp(j*(t+ pi/4));

realf = real(ft);
imagf = imag(ft);
absf = abs(ft);
angf = angle(ft);

figure;

subplot(2,2,1);
fplot(t, realf);
title('Real Part of f(t)');

subplot(2,2,2);
fplot(t, imagf);
title('Imaginary Part of f(t)');

subplot(2,2,3);
fplot(t, absf);
title('Magnitude of f(t)');

subplot(2,2,4);
fplot(t, angf);
title('Phase of f(t)');

saveas(gcf, './p3.png');

\end{lstlisting}

运行结果如下图\ref{fig:p3}

\begin{figure}[H]
  \centering
  \includegraphics[width=0.8\textwidth]{./matlab/p3_signals.png}
  \caption{问题三信号波形}
  \label{fig:p3}
\end{figure}




\end{document}

