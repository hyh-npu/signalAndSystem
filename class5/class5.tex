% ! TEX root = ../report.tex

\documentclass[../report.tex]{subfile}  % 仅这一行关联主文件,无其他导言命令

\begin{document}
\section{抽样定理与信号恢复}
\subsection{实验目的}
%1、观察离散信号频谱,了解其频谱特点。
%
%2、验证抽样定理并恢复原信号。

\begin{enumerate}
  \item 观察抽样信号波形,了解同步与异步抽样的区别。
  \item 验证抽样定理,掌握信号恢复的方法。
\end{enumerate}

\subsection{实验器材}
\begin{enumerate}
    \item 双踪示波器:1台
    \item 信号源及频率计模块S2:1块
    \item 抽样定理及滤波器模块S3:1块
\end{enumerate}

\subsection{实验原理}
\subsubsection{抽样定理}
抽样信号在条件 \(f_s \geq 2B_f\) 时可以恢复,其中 \(f_s\) 为抽样频率,\(B_f\) 为原信号占有频带宽度。由于抽样信号频谱是原信号频谱的周期性延拓,只要通过截止频率为 \(f_c\)(满足 \(f_m \leq f_c \leq f_s - f_m\),\(f_m\) 是原信号频谱中的最高频率)的低通滤波器就能恢复出原信号。若 \(f_s < 2B_f\),则抽样信号频谱会出现混叠,将无法恢复。

在实际信号中,仅含有限频率成分的信号极少,大多信号的频率成分是无限的,且实际低通滤波器在截止频率附近的频率特性曲线不够理想(如图\ref{fig:5-4}所示)。若使 \(f_s = 2B_f\)、\(f_c = f_m = B_f\),恢复出的信号难免有失真。为减小失真,应将抽样频率 \(f_s\) 取高(\(f_s > 2B_f\)),低通滤波器需满足 \(f_m < f_c < f_s - f_m\)。

\begin{figure}[!htbp]
    \centering
    \includegraphics[width=0.7\textwidth]{./fig5-4.png}  % 请替换为实际图片路径
    \caption{实际低通滤波器在截止频率附近频率特性曲线}
    \label{fig:5-4}
\end{figure}

\subsubsection{S3上的自然抽样功能}
\paragraph{抽样及恢复流程}
信号自然抽样及恢复流程如图\ref{fig:5-5}所示。连续信号 \(F(t)\) 经抽样器进行抽样处理,抽样输出的信号 \(F_s(t)\) 再经低通滤波器,输出信号 \(F'(t)\)。当抽样时钟频率满足抽样定理时,就能从 \(F_s(t)\) 中恢复出原始的连续信号 \(F(t)\)。

\begin{figure}[!htbp]
    \centering
    \includegraphics[width=0.8\textwidth]{fig5-5.png}  % 请替换为实际图片路径
    \caption{信号自然抽样及恢复流程图}
    \label{fig:5-5}
\end{figure}

\paragraph{同步与异步抽样}
S3模块有同步和异步两种抽样方式:
- 同步抽样:抽样时钟与连续信号由同一晶振源产生,便于观察到稳定的抽样信号,可对比信号抽样前后及恢复信号的波形。
- 异步抽样:连续信号与抽样时钟是不同源的关系,贴近实际的信号抽样过程,抽样频率连续可调,但不便于用示波器观察到稳定的抽样信号。

\paragraph{有源低通滤波器}
实验中,模块采用8阶有源低通滤波器对抽样后的信号进行滤波恢复,滤波器电路原理图如图\ref{fig:5-6}所示。异步抽样频率范围为1.25kHz~35kHz。

\begin{figure}[!htbp]
    \centering
    \includegraphics[width=0.8\textwidth]{fig5-6.png}  % 请替换为实际图片路径
    \caption{S3模块有源低通滤波器电路原理图}
    \label{fig:5-6}
\end{figure}

\subsection{实验步骤}
\subsubsection{任务:自然抽样及恢复}
\paragraph{1、观察抽样信号波形}
选用正弦波作为被抽样信号进行实验,步骤如下:
\begin{enumerate}
    \item 将S2模块中的扫频开关S3置为“OFF”,调节模拟信号源上的“ROL1”旋钮和“模拟输出幅度调节”旋钮,使P2处输出频率1kHz、幅度2V的正弦波。
    \item 连接模拟信号源输出端P2与抽样定理模块S3的连续信号输入点P17。
    \item 将开关S2拨至“异步”,用示波器观察TP20处抽样信号输出波形,调整电位器W1改变抽样频率,观察抽样信号的变化情况。
    \item 将开关S2拨至“同步”,连接信号源及频率计模块S2中P5与抽样定理模块S3上外部开关输入点P18。用示波器的两通道分别观察模拟信号输出端P2、抽样信号输出波形TP20,调整按钮S7改变抽样频率,观察抽样信号的变化情况,完成下表:
\end{enumerate}

\begin{table}[!htbp]
    \centering
    \caption{抽样信号波形观察记录表}
    \label{tab:5-1}
    \begin{tabular}{|c|c|}
        \hline
        抽样频率 & $F_s(t)$抽样信号TP20的波形 \\
        \hline
        1kHz &  \\
        \hline
        2kHz &  \\
        \hline
        4kHz &  \\
        \hline
        8kHz &  \\
        \hline
    \end{tabular}
\end{table}

\paragraph{2、验证抽样定理与信号恢复}
\begin{enumerate}
    \item 连接模块S3的P20和P19。
    \item 用示波器接原始抽样信号输入点TP17、恢复信号输出点TP22。
    \item 改变抽样时钟信号,对比观察信号恢复情况,完成下表:
    \item 将开关S2拨至“异步”,用示波器观察TP20处抽样信号的波形,调整电位器W1改变抽样频率,观察抽样信号的变化情况。
\end{enumerate}

\begin{table}[!htbp]
    \centering
    \caption{抽样定理验证与信号恢复记录表}
    \label{tab:5-2}
    \begin{tabular}{|c|c|c|c|}
        \hline
        输入信号频率 & 抽样频率 & TP20抽样信号输出 & TP22恢复信号输出 \\
        \hline
        1kHz & 1kHz &  \\
        \hline
        1kHz & 2kHz &  \\
        \hline
        1kHz & 4kHz &  \\
        \hline
        1kHz & 8kHz &  \\
        \hline
    \end{tabular}
\end{table}

\subsection{实验结果}
\paragraph{1、抽样信号波形观察记录}
抽样信号的波形如下图\ref{fig:5-wave}所示:
\begin{figure}[H]
  \centering
  \begin{subfigure}
    {0.45\textwidth}
    \centering
    \includegraphics[width=\textwidth]{fig5-1.png}  % 请替换为实际图片路径
    \caption{抽样频率1kHz}
  \end{subfigure}
  \begin{subfigure}
    {0.45\textwidth}
    \centering
    \includegraphics[width=\textwidth]{fig5-2.png}  % 请替换为实际图片路径
    \caption{抽样频率2kHz}
  \end{subfigure}
  \begin{subfigure}
    {0.45\textwidth}
    \centering
    \includegraphics[width=\textwidth]{fig5-3.png}  % 请替换为实际图片路径
    \caption{抽样频率4kHz}
  \end{subfigure}
  \begin{subfigure}
    {0.45\textwidth}
    \centering
    \includegraphics[width=\textwidth]{fig5-4 (1).png}  % 请替换为实际图片路径
    \caption{抽样频率8kHz}
  \end{subfigure}
  \caption{抽样信号波形}
  \label{fig:5-wave}
\end{figure}

\paragraph{2、抽样定理验证与信号恢复记录}
抽样信号的恢复波形如下图\ref{fig:5-recover}所示:
\begin{figure}[H]
  \centering
  \begin{subfigure}
    {0.45\textwidth}
    \centering
    \includegraphics[width=\textwidth]{fig5-7.png}  % 请替换为实际图片路径
    \caption{抽样频率1kHz}
  \end{subfigure}
  \begin{subfigure}
    {0.45\textwidth}
    \centering
    \includegraphics[width=\textwidth]{fig5-8.png}  % 请替换为实际图片路径
    \caption{抽样频率2kHz}
  \end{subfigure}
  \begin{subfigure}
    {0.45\textwidth}
    \centering
    \includegraphics[width=\textwidth]{fig5-9.png}  % 请替换为实际图片路径
    \caption{抽样频率4kHz}
  \end{subfigure}
  \begin{subfigure}
    {0.45\textwidth}
    \centering
    \includegraphics[width=\textwidth]{fig5-10.png}  % 请替换为实际图片路径
    \caption{抽样频率8kHz}
  \end{subfigure}
  \caption{抽样信号恢复波形}
  \label{fig:5-recover}
\end{figure}

\subsection{实验程序及运行结果}
\subsubsection{任务一程序及运行结果}
\begin{lstlisting}[language=Matlab, 
                  caption={信号波形生成与绘图代码}, 
                        % 指定代码语言(此处为MATLAB)
  basicstyle=\ttfamily\small,  % 等宽字体+小字号(Markdown常用)
  backgroundcolor=\color{gray!5},  % 浅灰背景(接近多数Markdown渲染效果)
  frame=none,           % 无边框(Markdown代码块通常无框)
  keywordstyle=\color{blue},  % 关键字高亮(蓝色,可选)
  commentstyle=\color{green},  % 注释高亮(绿色,可选)
  showstringspaces=false,  % 不显示字符串中的空格标记
  numbers=none,         % 不显示行号(Markdown默认无行号)
  breaklines=true,      % 自动换行(避免溢出)
  columns=fullflexible  ] 

 Ts = 0.5; %采样周期
 ts = -10:Ts:10;
 t = -10:0.01:10;
 %原始信号和采样信号

 fs = sinc(ts).*(heaviside(ts+2*pi)-heaviside(ts-2*pi));
 f = sinc(t).*(heaviside(t+2*pi)-heaviside(t-2*pi));
 figure;
 subplot(2,2,1);
 plot(t,f);
 title(' Original Signal f(t)');
 subplot(2,2,2);
 scatter(ts,fs);
 title(' Sampled Signal fs(t)');

 %信号恢复
 wc = pi/Ts;         
 sinc_mat = sinc( wc/pi * (t'*ones(1,length(ts)) - ones(length(t),1)*ts) );
 fr = sum( fs .* sinc_mat, 2 ); 
 subplot(2,2,3);
 plot(t,fr);
 title(' Reconstructed Signal fr(t)');

 bias = fr - f';
 subplot(2,2,4);
 plot(t,bias);
 title(' Reconstruction Error fr(t)-f(t)');





 %保存图片
 handle = gcf;
 saveas(handle,'./class5_1png');

\end{lstlisting}

运行结果如图\ref{fig:5-task1}所示:

\begin{figure}[H]
  \centering
  \includegraphics[width=0.8\textwidth]{./matlab/class5_1_original_signal.png}  % 请替换为实际图片路径
  \caption{任务一运行结果}
  \label{fig:5-task1}
\end{figure}

\paragraph{任务二程序及运行结果}
\begin{lstlisting}[language=Matlab, 
                  caption={信号波形生成与绘图代码}, 
                        % 指定代码语言(此处为MATLAB)
  basicstyle=\ttfamily\small,  % 等宽字体+小字号(Markdown常用)
  backgroundcolor=\color{gray!5},  % 浅灰背景(接近多数Markdown渲染效果)
  frame=none,           % 无边框(Markdown代码块通常无框)
  keywordstyle=\color{blue},  % 关键字高亮(蓝色,可选)
  commentstyle=\color{green},  % 注释高亮(绿色,可选)
  showstringspaces=false,  % 不显示字符串中的空格标记
  numbers=none,         % 不显示行号(Markdown默认无行号)
  breaklines=true,      % 自动换行(避免溢出)
  columns=fullflexible  ] 

Ts = 0.3; %采样周期
ts = -4:Ts:4;
t = -4:0.01:4;
%原始信号和采样信号

fs = 1/2.*(1.+cos(ts)).*(heaviside(ts+2*pi)-heaviside(ts-2*pi));
f = 1/2.*(1.+cos(t)).*(heaviside(t+2*pi)-heaviside(t-2*pi));
figure;
subplot(2,2,1);
plot(t,f);
title(' Original Signal f(t)');
subplot(2,2,2);

stem(ts,fs);
title(' Sampled Signal fs(t)');

%信号恢复
wc = pi/Ts;         
sinc_mat = sinc( wc/pi * (t'*ones(1,length(ts)) - ones(length(t),1)*ts) );
fr = sum( fs .* sinc_mat, 2 ); 
subplot(2,2,3);
plot(t,fr);
title(' Reconstructed Signal fr(t)');

bias = fr - f';
subplot(2,2,4);
plot(t,bias);
title(' Reconstruction Error fr(t)-f(t)');





%保存图片
handle = gcf;
saveas(handle,'class5_2.png');
\end{lstlisting}

运行结果如图\ref{fig:5-task2}所示:
\begin{figure}[H]
  \centering
  \includegraphics[width=0.8\textwidth]{./matlab/class_5_original_signal.png}  % 请替换为实际图片路径
  \caption{任务二运行结果}
  \label{fig:5-task2}
\end{figure}

\paragraph{任务三程序及运行结果}
\begin{lstlisting}[language=Matlab, 
                  caption={信号波形生成与绘图代码}, 
                        % 指定代码语言(此处为MATLAB)
  basicstyle=\ttfamily\small,  % 等宽字体+小字号(Markdown常用)
  backgroundcolor=\color{gray!5},  % 浅灰背景(接近多数Markdown渲染效果)
  frame=none,           % 无边框(Markdown代码块通常无框)
  keywordstyle=\color{blue},  % 关键字高亮(蓝色,可选)
  commentstyle=\color{green},  % 注释高亮(绿色,可选)
  showstringspaces=false,  % 不显示字符串中的空格标记
  numbers=none,         % 不显示行号(Markdown默认无行号)
  breaklines=true,      % 自动换行(避免溢出)
  columns=fullflexible  ] 

Ts = 0.1;
t = -3:0.01:3;
T = -3:Ts:3;
ft = 2.*((t<0&t>-1)|(t>1&t<2))+1*(t>=0&t<=1);

%采样信号
fTs = 2.*((T<0&T>-1)|(T>1&T<2))+1*(T>=0&T<=1);

%数值傅里叶变换
dw = 0.01;
w = -100:dw:100;
FTw = dw.*fTs*exp(-1j*T'*w);

%抽样信号的还原

%低通滤波器
Hw = heaviside(w+pi/Ts)-heaviside(w-pi/Ts);
Ftw = FTw.*Hw;
%逆傅里叶变换
f_recon = (1/(2*pi)).*Ftw*exp(1j*w'*t)*dw;

figure;

subplot(4,1,1);
plot(t,ft);
title(' Original Signal f(t)');

subplot(4,1,2);
stem(T,fTs);
title(' Sampled Signal f_{Ts}(t)');

subplot(4,1,3);
plot(w,abs(FTw));
title('Magnitude Spectrum |F(w)|');

subplot(4,1,4);
plot(t,real(f_recon));
title(' Reconstructed Signal f_{recon}(t)');

saveas(gcf,'./class5_3.png');
\end{lstlisting}

运行结果如图\ref{fig:5-task3}所示:
\begin{figure}[H]
  \centering
  \includegraphics[width=0.8\textwidth]{./matlab/class5_3.png}  % 请替换为实际图片路径
  \caption{任务三运行结果}
  \label{fig:5-task3}
\end{figure}

\end{document}
