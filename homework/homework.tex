% ! TEX root=../report.tex

\documentclass[../report.tex]{subfiles}
\begin{document}
% 仅作为整体文档的一个Section,适配嵌入需求
\newpage
\section{噪声抑制算法设计与实现}
\subsection{设计思路}
本算法针对1800Hz窄带噪声构建高斯陷波滤波器,通过分帧频域处理实现噪声精准抑制,核心逻辑为将信号分帧后转换至频域,利用高斯型带阻特性衰减目标噪声频率,最终重构时域信号。

\subsubsection{背景知识}

\begin{enumerate}
    \item 音频信号的非平稳特性:音频/电信号属于典型的非平稳时变信号,全局处理无法精准针对局部噪声特征,需通过分帧将信号划分为若干短时帧(通常10~30ms),每帧可近似视为短时平稳信号,为频域处理提供前提。
    \item 窄带噪声抑制的通用思路:单一频率/窄带噪声(如电磁干扰、设备谐振噪声)无法通过时域滤波精准抑制,频域陷波滤波是主流方案;传统矩形陷波滤波器易引入吉布斯现象(频谱振铃),高斯型陷波滤波器凭借平滑过渡带可显著降低信号失真。
    \item 分帧加窗的通用准则:分帧会导致信号截断,引入频谱泄漏,汉明窗、汉宁窗等余弦类窗函数可通过平滑帧边缘抑制泄漏;重叠率(通常25\%~75\%)是平衡帧间连续性与计算效率的关键,高重叠率(如75\%)可避免重构后信号断裂。
    \item 频域处理的核心逻辑:快速傅里叶变换(FFT)将时域信号转换为频域,可分离幅值谱与相位谱,仅对幅值谱进行带阻/带通处理(相位保持),能避免音色失真,逆FFT(IFFT)可将处理后的频域信号转回时域。
    \item 信号重构的鲁棒性设计:重叠相加法是分帧处理后信号重构的通用方法,需对重叠区域幅值加权平均(权重为窗函数叠加值),补偿加窗导致的幅值衰减;同时需处理帧长超限、除零等异常,保证输出信号长度与输入一致。
\end{enumerate}


\subsubsection{技术原理}
\begin{enumerate}
    \item 分帧与加窗原理:将原始信号划分为若干重叠帧,帧长由采样率和10ms时间窗长确定,重叠率75\%保证帧间连续性。汉明窗表达式为:
    \[
    w(n) = 0.54 - 0.46\cos\left(\frac{2\pi n}{N-1}\right), \quad n=0,1,...,N-1
    \]
    其中$N$为帧长,加窗可抑制频域分析的频谱泄漏。
    \item 高斯陷波滤波器设计:针对1800Hz中心噪声频率,构造高斯型带阻增益函数:
    \[
    bandElimination = 1 - \exp\left(-\frac{d(f)^2}{2\sigma^2}\right)
    \]
    式中$d(f) = \min\left(\vert f - f_0 \vert, \vert f_s - f - f_0 \vert\right)$($f_0=1800\mathrm{Hz}$为噪声中心频率,$f_s$为采样率),保证频率轴对称特性;$\sigma=300\mathrm{Hz}$为高斯函数标准差,决定带阻宽度。
    \item 频域处理与信号重构:
    \begin{enumerate}
        \item 对每帧加窗信号做快速傅里叶变换(FFT),分离幅值谱$|X(k)|$和相位谱$\angle X(k)$;
        \item 幅值谱乘以高斯陷波增益,相位谱保持不变,重构频域信号:$X'(k) = |X(k)| \cdot bandElimination \cdot e^{j\angle X(k)}$;
        \item 对重构频域信号做逆FFT(IFFT)转回时域;
        \item 采用重叠叠加法重构完整信号,对重叠区域幅值加权平均(权重为窗函数叠加值),补偿加窗导致的幅值衰减。
    \end{enumerate}
\end{enumerate}

\subsection{具体实现过程}
算法通过MATLAB函数`noise\_suppress`实现,输入为待降噪信号$y$和采样率$fs$,输出为降噪后信号$Y$和采样率$Fs$,具体步骤如下:

\subsubsection{步骤1:输入校验与预处理}
\begin{enumerate}
    \item 校验输入信号$y$是否为向量,若非向量则抛出错误并提示当前维度;
    \item 将信号转换为列向量,记录信号长度$sig\_len$,将采样率赋值给输出变量$Fs$。
\end{enumerate}

\subsubsection{步骤2:分帧参数计算}
\begin{enumerate}
    \item 设定帧时间为10ms,计算帧长$frameLength = \text{round}(0.010 \times fs)$;若帧长大于信号长度,则调整为$\max(\text{floor}(sig\_len/2), 2)$,保证分帧有效性;
    \item 设定重叠率75\%,计算重叠点数$overlapLength = \text{floor}(frameLength \times 0.75)$,帧移$frameStep = frameLength - overlapLength$;
    \item 计算总帧数:$frameNumber = \text{ceil}((sig\_len - frameLength)/frameStep) + 1$,确保所有信号点被覆盖。
\end{enumerate}

\subsubsection{步骤3:信号分帧}
\begin{enumerate}
    \item 初始化帧矩阵$frames$(维度:帧数×帧长);
    \item 遍历每帧,计算帧起始索引$start\_idx = (i-1) \times frameStep + 1$和结束索引$end\_idx = start\_idx + frameLength - 1$;
    \item 若结束索引超出信号长度,对该帧补零;否则直接截取对应信号段,完成帧矩阵填充。
\end{enumerate}

\subsubsection{步骤4:加窗处理}
\begin{enumerate}
    \item 生成汉明窗$window$,维度与帧长一致;
    \item 将帧矩阵与窗函数按列相乘($windowedFrames = frames \cdot window'$),完成每帧的加窗操作,抑制频谱泄漏。
\end{enumerate}

\subsubsection{步骤5:频域降噪处理}
\begin{enumerate}
    \item 计算频率轴:$freq = (0:frameLength-1) \times (fs / frameLength)$;
    \item 遍历每帧执行以下操作:
    \begin{enumerate}
        \item 对加窗帧做FFT得到$frameFFT$,提取幅值$magnitude = \text{abs}(frameFFT)$和相位$phase = \text{angle}(frameFFT)$;
        \item 计算频率与1800Hz的距离$freqDist$,构造高斯陷波增益$bandElimination$;
        \item 重构频域信号$denoisedFrameFFT = magnitude \cdot bandElimination \cdot \exp(1j \times phase)$;
        \item 对重构频域信号做IFFT并取实部,得到降噪后的时域帧$denoisedFrame$。
    \end{enumerate}
\end{enumerate}

\subsubsection{步骤6:窗补偿与信号重构}
\begin{enumerate}
    \item 初始化输出信号$Y$和权重向量$weight$(均为全零列向量);
    \item 遍历每帧,将降噪帧叠加到$Y$的对应位置,同时将窗函数叠加到$weight$的对应位置;若帧结束索引超出信号长度,仅处理有效长度部分;
    \item 对权重非零区域,将$Y$除以权重完成幅值补偿;权重为零区域赋值为0,避免除零错误,保证输出信号长度与输入完全一致。
\end{enumerate}

\subsubsection{步骤7:滤波器特性可视化}
\begin{enumerate}
    \item 创建图形窗口,绘制高斯陷波滤波器的频率响应曲线(横轴为频率,纵轴为增益);
    \item 设置坐标轴范围(1000~2500Hz),添加网格和边框,保存图片至指定路径后关闭窗口;
    \item 实际滤波器如图\ref{fig:filter_response}
    \begin{figure}[htbp]
        \centering
        % 此处为配图占位符,实际使用时替换为真实图片路径
        \includegraphics[width=0.5\textwidth]{./matlab/noiseElimination/filter.png}
        \caption{高斯陷波滤波器频率响应(中心频率1800Hz)}
        \label{fig:filter_response}
    \end{figure}
\end{enumerate}

\subsubsection{算法核心特性}
\begin{enumerate}
    \item 鲁棒性:处理了帧长大于信号长度、帧索引越界、除零等异常情况,避免程序崩溃;
    \item 低失真:采用高斯陷波滤波器,过渡带平滑,保持相位不变,降低信号失真;
    \item 完整性:重构信号长度与输入完全一致,保证时序连续性。
\end{enumerate}

\subsection{程序代码及运行结果}
\subsubsection{程序代码}
代码分为三部分:
\begin{enumerate}
  \item `noise\_suppress`:包含一个函数,输入音频,返回降噪音频
  \item `draw\_spectrogram`:包含一个函数,输入音频,绘制语谱图并返回音频语谱图的句柄
  \item `runNoiseSuppress`:控制流,添加噪声,去噪,绘图,保存音频
\end{enumerate}

下面是代码:


\begin{lstlisting}[language=Matlab, 
                  caption={noise\_suppress}, 
                        % 指定代码语言(此处为MATLAB)
  basicstyle=\ttfamily\small,  % 等宽字体+小字号(Markdown常用)
  backgroundcolor=\color{gray!5},  % 浅灰背景(接近多数Markdown渲染效果)
  frame=none,           % 无边框(Markdown代码块通常无框)
  keywordstyle=\color{blue},  % 关键字高亮(蓝色,可选)
  commentstyle=\color{green},  % 注释高亮(绿色,可选)
  showstringspaces=false,  % 不显示字符串中的空格标记
  numbers=none,         % 不显示行号(Markdown默认无行号)
  breaklines=true,      % 自动换行(避免溢出)
  columns=fullflexible  ] 

function [Y, Fs] = noise_suppress(y, fs)
% 降噪函数:修复信号长度/维度/幅值异常问题
% 输入:y-输入信号(向量),fs-采样率
% 输出:Y-输出信号(列向量,长度与输入完全一致),Fs-采样率


    if ~isvector(y)
        error('输入y必须是向量!当前是矩阵,维度:%s', mat2str(size(y)));
    end
    y = y(:); 
    sig_len = length(y); 
    Fs = fs; 
    
    frameTime = 0.010; % 帧长 10ms
    frameLength = round(frameTime * fs); % 帧长(点数)
    if frameLength >= sig_len 
        frameLength = max(floor(sig_len/2), 2); 
    end
    overlapRate = 0.75; % 重叠率
    overlapLength = floor(frameLength * overlapRate); % 重叠点数
    frameStep = frameLength - overlapLength; % 帧移
     
    frameNumber = ceil((sig_len - frameLength) / frameStep) + 1; 

    
    frames = zeros(frameNumber, frameLength); % 帧矩阵:帧数×帧长
    for i = 1:frameNumber
        start_idx = (i-1)*frameStep + 1;
        end_idx = start_idx + frameLength - 1;
        
        if end_idx > sig_len
            frame_data = zeros(1, frameLength);
            valid_len = sig_len - start_idx + 1;
            frame_data(1:valid_len) = y(start_idx:end);
            frames(i, :) = frame_data;
        else
            frames(i, :) = y(start_idx:end_idx)'; 
        end
    end

    
    window = hamming(frameLength); 
    windowedFrames = frames .* window'; 

    noiseFreq = 1800; % 噪声中心频率
    eliminationRate = 300; % 带阻宽度
    denoisedFrames = zeros(size(windowedFrames)); % 降噪帧矩阵
    
    freq = (0:frameLength-1) * (fs / frameLength); 

    for i = 1:frameNumber
       
        frameFFT = fft(windowedFrames(i, :)); 
        magnitude = abs(frameFFT); % 幅值
        phase = angle(frameFFT);   % 相位
        
        freqDist = abs(freq - noiseFreq); 
        freqDist = min(freqDist, abs(fs - freq - noiseFreq)); 
        bandElimination = 1 - exp( -freqDist.^2 / (2 * eliminationRate^2) );
        % 频域降噪
        denoisedFrameFFT = magnitude .* bandElimination .* exp(1j * phase);
         
        denoisedFrame = real(ifft(denoisedFrameFFT)); 
        denoisedFrames(i, :) = denoisedFrame;
    end

    
    compensationWindow = hamming(frameLength); % 补偿窗
    
    denoisedFrames = denoisedFrames ./ (compensationWindow' + eps) * 0.5; 
    denoisedFrames(isnan(denoisedFrames) | isinf(denoisedFrames)) = 0; 

    
    Y = zeros(sig_len, 1); 
    weight = zeros(sig_len, 1); 
    for i = 1:frameNumber
        start_idx = (i-1)*frameStep + 1;
        end_idx = start_idx + frameLength - 1;
        
        if end_idx > sig_len
            end_idx = sig_len;
            valid_len = end_idx - start_idx + 1;
            Y(start_idx:end_idx) = Y(start_idx:end_idx) + denoisedFrames(i, 1:valid_len)';
            weight(start_idx:end_idx) = weight(start_idx:end_idx) + window(1:valid_len);
        else
            Y(start_idx:end_idx) = Y(start_idx:end_idx) + denoisedFrames(i, :)';
            weight(start_idx:end_idx) = weight(start_idx:end_idx) + window;
        end
    end
    Y(weight > 0) = Y(weight > 0) ./ weight(weight > 0);
    Y(weight == 0) = 0; 

    figure('Name', '1800Hz高斯陷波滤波器');
    plot(freq, bandElimination, 'LineWidth', 1.2);
    xlabel('频率 (Hz)'); ylabel('增益');
    title(sprintf('高斯陷波滤波器(中心频率:%d Hz)', noiseFreq));
    xlim([1000, 2500]); grid on; box on;
    saveas(gcf, './filter.png');
    close(gcf); 

end
\end{lstlisting}



\begin{lstlisting}[language=Matlab, 
                  caption={draw\_spectrogram}, 
                        % 指定代码语言(此处为MATLAB)
  basicstyle=\ttfamily\small,  % 等宽字体+小字号(Markdown常用)
  backgroundcolor=\color{gray!5},  % 浅灰背景(接近多数Markdown渲染效果)
  frame=none,           % 无边框(Markdown代码块通常无框)
  keywordstyle=\color{blue},  % 关键字高亮(蓝色,可选)
  commentstyle=\color{green},  % 注释高亮(绿色,可选)
  showstringspaces=false,  % 不显示字符串中的空格标记
  numbers=none,         % 不显示行号(Markdown默认无行号)
  breaklines=true,      % 自动换行(避免溢出)
  columns=fullflexible  ] 

function fig = draw_spectrogram(signal, fs, save_path)

    if ~isa(signal, 'double')
        signal = double(signal);
    end
    if ismatrix(signal)
        signal = signal(:, 1);
    end
    signal = signal(:); 

    % 语谱图参数
    window_size = 256; 
    overlap = 128;     
    nfft = 512;        

    
    [S, F, T] = spectrogram(signal, window_size, overlap, nfft, fs);
    S_abs = abs(S);
    S_abs(S_abs == 0) = eps;  
    S_dB = 10*log10(S_abs);

    
    fig_width = 1200;    
    fig_height = 800;    
    fig = figure(...
        'Name', 'Spectrogram', ...
        'Position', [100, 100, fig_width, fig_height], ... % [左偏移, 下偏移, 宽, 高]
        'Units', 'inches', ...  
        'Color', 'white' ... 
    );

      ax = axes(...
        'Parent', fig, ...
        'Position', [0.08, 0.1, 0.75, 0.8], ... % [左, 下, 宽, 高]
        'Units', 'normalized' ...
    );
    imagesc(ax, T, F, S_dB); 
    axis(ax, 'xy');  
    xlabel(ax, 'Time (s)', 'FontSize', 12); 
    ylabel(ax, 'Frequency (Hz)', 'FontSize', 12);
    title(ax, 'Spectrogram (Unified Scale & Width)', 'FontSize', 14);
    set(ax, 'FontSize', 10); %

    
    cb = colorbar(ax, 'Position', [0.87, 0.1, 0.03, 0.8]); 
    cb.Label.String = 'Amplitude (dB)';
    cb.Label.FontSize = 11;
    set(cb, 'FontSize', 10);

    
    caxis(ax, [-80 20]); 

        if nargin >= 3 && ~isempty(save_path)
        
        print(fig, save_path, '-dpng', '-r600', '-painters');
        fprintf('语谱图已保存至:%s(600dpi,尺寸:%d×%d英寸)\n', save_path, fig_width, fig_height);
    end
end

\end{lstlisting}

\begin{lstlisting}[language=Matlab, 
                  caption={runNoiseSuppress}, 
                        % 指定代码语言(此处为MATLAB)
  basicstyle=\ttfamily\small,  % 等宽字体+小字号(Markdown常用)
  backgroundcolor=\color{gray!5},  % 浅灰背景(接近多数Markdown渲染效果)
  frame=none,           % 无边框(Markdown代码块通常无框)
  keywordstyle=\color{blue},  % 关键字高亮(蓝色,可选)
  commentstyle=\color{green},  % 注释高亮(绿色,可选)
  showstringspaces=false,  % 不显示字符串中的空格标记
  numbers=none,         % 不显示行号(Markdown默认无行号)
  breaklines=true,      % 自动换行(避免溢出)
  columns=fullflexible  ] 
clc; close all; clear;
audioPath = '../name.wav'
outputPath = './denoised_name.wav'
outputNoisedPath = './noised_name.wav'

% 噪声参数
noiseAmplitude = 0.02; % 噪声幅度
noiseFrequency = 1800; % 噪声频率 Hz

[y, fs] = audioread(audioPath);

original_fig = draw_spectrogram(y,fs);
saveas(original_fig, 'original_spectrogram.png')

% 添加噪声
noise = noiseAmplitude * sin(2 * pi * noiseFrequency * (0:length(y)-1)' / fs);
noisedSignal = y + noise;
audiowrite(outputNoisedPath, noisedSignal, fs);

% 绘制语谱图分析
% fig = draw_spectrogram(y, fs);
noised_fig = draw_spectrogram(noisedSignal, fs);
% saveas(fig, 'original_spectrogram.png');
saveas(noised_fig, 'noised_spectrogram.png');

% 降噪处理
[Y, Fs] = noise_suppress(noisedSignal, fs);
denoised_fig = draw_spectrogram(Y, Fs);
saveas(denoised_fig, 'denoised_spectrogram.png');
audiowrite(outputPath, Y, Fs);

figure;
stem(y, 'b', 'filled');
hold on;
stem( noisedSignal, 'r');
stem( Y, 'g');
hold off;
disp('Noise suppression processing completed.');
\end{lstlisting}

\subsubsection{运行结果}

首先是原音频的语谱图\ref{fig:org_spec}:
\begin{figure}[htbp]
  \centering
  \includegraphics[width=0.7\textwidth]{./matlab/noiseElimination/original_spectrogram.png}
  \caption{原音频语谱图}
  \label{fig:org_spec}
\end{figure}

加入噪声后的音频语谱图如下图\ref{fig:noised_spec}

\begin{figure}[htbp]
  \centering
  \includegraphics[width=0.7\textwidth]{./matlab/noiseElimination/noised_spectrogram.png}
  \caption{加入噪声后音频语谱图}
  \label{fig:noised_spec}
\end{figure}

 在纵轴的1800HZ处可以明显看到噪声对应的频谱信号。

使用滤波器去噪后的音频语谱图如下 \ref{fig:denoised_spec}

\begin{figure}[htbp]
  \centering
  \includegraphics[width=0.7\textwidth]{./matlab/noiseElimination/denoised_spectrogram.png}
  \caption{去除噪声后的语谱图}
  \label{fig:denoised_spec}
\end{figure}

经过去噪后,可见噪声已经有明显衰减。


所有处理后的语音都在在文件夹`./matlab/noiseElimination/`中储存。包括:
\begin{itemize}
  \item `noised\_name.wav`:加入噪声后的音频
  \item `denoised\_name.wav`:去除噪声后的音频 
\end{itemize}

\newpage
\section{女声转男声音频处理算法设计与实现}
\subsection{设计思路}
本算法实现音频的变调不变速,以男女声的音频的转换为例。

\subsubsection{背景知识}
\begin{enumerate}
    \item 人声音高与音色的核心特征:基频(Fundamental Frequency)决定音高,女声基频(200~500Hz)显著高于男声(85~180Hz),调整基频是女声转男声的核心手段;共振峰(Formant)决定音色,是否保留共振峰直接影响变声后音色的自然度。
    \begin{figure}
      \centering
      \includegraphics[width=0.7\textwidth]{./genderConverse.png}
      \caption{不同性别音色频谱图}
    \end{figure}
    \item 短时傅里叶变换(STFT)的窗长适配:STFT窗长决定频率/时间分辨率,窗长过小(如512点)会导致频率分辨率不足,信号处理失真;1024点窗长是人声处理的经典取值,可平衡分辨率与处理效率。
    \item 高频增强的合理方式:人声清晰感依赖1000~5000Hz高频成分,硬高通滤波会滤除0~3kHz人声核心频段,导致声音丢失;“带通提取+温和叠加”的方式仅增强高频,不破坏核心频段,是人声增强的通用策略。
    \item 音频幅值归一化的必要性:音频处理后易出现幅值异常(过小/过大),将幅值归一化到[-1,1]可避免削波失真,增益调整(如0.9)可保证最终音量适中;同时需检测处理后信号丢失等异常,降级使用原始音频避免程序报错。
    \item 频谱分析的验证价值:通过FFT计算单侧边频谱是验证音频处理效果的通用手段,可直观对比处理前后核心频段(如0~3kHz)的幅值分布,确认核心信号未丢失。
\end{enumerate}

\subsubsection{技术原理}
\begin{enumerate}
    \item 基频调整原理:基于STFT的音高偏移算法(shiftPitch函数),通过调整每帧信号的基频实现音高降低(nsemitones为半音数,负值表示降频);设置汉明窗、75%重叠率、锁相(LockPhase)保证帧间信号连续性,避免破音/失真。
    \item 温和高频增强原理:替代硬高通滤波,采用2阶巴特沃斯带通滤波器提取1000~5000Hz高频成分,将其以小幅增益叠加回原信号,仅增强高频清晰度,不滤除0~3kHz人声核心频段。
    \item 频谱分析原理:通过快速傅里叶变换(FFT)计算音频单侧边频谱,公式如下:
    \[
    \begin{split}
    P_2 &= |\text{FFT}(audio)| / L, \quad (L为音频长度) \\
    P_1 &= P_2(1:L/2+1), \quad (\text{单侧边频谱}) \\
    P_1(2:\text{end}-1) &= 2 \cdot P_1(2:\text{end}-1), \quad (\text{幅值补偿}) \\
    f_p &= f_s \cdot (0:(L/2))/L, \quad (\text{频率轴})
    \end{split}
    \]
    其中$P_1$为单侧边幅值谱,$f_p$为频率轴,通过对比原始/处理后频谱验证核心频段信号存在。
    \item 音量归一化原理:将处理后信号幅值归一化到[-1,1]区间,再通过增益系数调整音量,同时增加防除零处理,避免信号丢失时程序报错。
\end{enumerate}

\subsection{具体实现过程}
算法通过MATLAB脚本实现女声转男声,核心解决“声音消失、音色闷/憨”问题,具体步骤如下:

\subsubsection{步骤1:环境初始化与音频读取}
\begin{enumerate}
    \item 执行环境清理:`clear; clc; close all;` 清空变量、清除命令行、关闭所有图形窗口;
    \item 读取音频文件:调用`audioread`读取`../name.wav`,返回音频数据`audio\_In`和采样率`fs`;
    \item 立体声转单声道:若音频为立体声(列数>1),通过`mean(audio\_In, 2)`取列均值转为单声道;
    \item 音频有效性校验:若音频最大幅值小于1e-6,抛出错误提示“原始音频文件无信号”。
\end{enumerate}

\subsubsection{步骤2:核心调整参数设置}
\begin{enumerate}
    \item 基频调整参数:`nsemitones = -9.5`(降9.5个半音,平衡男声感和声音保留);
    \item 共振峰参数:`preserve\_formants = false`(不保留共振峰,避免音色闷/憨);
    \item 音量参数:`gain = 0.9`(提升音量,避免声音过小);
    \item 高频增强参数:`high\_freq\_boost = 1.05`(温和高频增益,仅小幅增强)。
\end{enumerate}

\subsubsection{步骤3:基频调整(核心变声)}
\begin{enumerate}
    \item 设置窗长:`win\_len = 1024`(恢复合理窗长,避免512窗长导致的信号失真);
    \item 调用音高偏移函数:`audio\_Out = shiftPitch(audio\_In, nsemitones, ...)`,关键参数:
    \begin{enumerate}
        \item `'PreserveFormants'`:设为`false`,不保留共振峰,避免音色闷/憨;
        \item `'Window'`:设为`hamming(win\_len)`,汉明窗减少频谱泄漏;
        \item `'OverlapLength'`:设为`round(win\_len * 0.75)`,75%重叠率保证帧间连续性;
        \item `'LockPhase'`:设为`true`,锁相避免相位失真导致的破音。
    \end{enumerate}
\end{enumerate}

\subsubsection{步骤4:温和高频增强(修复声音消失)}
\begin{enumerate}
    \item 设计带通滤波器:调用`butter(2, [1000 5000]/(fs/2), 'bandpass')`设计2阶巴特沃斯带通滤波器,通带1000~5000Hz;
    \item 提取高频成分:调用`filtfilt(b, a, audio\_Out)`对变声后音频做零相位滤波,提取高频成分`audio\_high`;
    \item 温和叠加高频:`audio\_Out = audio\_Out + (audio\_high * (high\_freq\_boost - 1))`,仅将高频成分以0.05倍增益叠加,不滤除核心频段。
\end{enumerate}

\subsubsection{步骤5:音量归一化与异常处理}
\begin{enumerate}
    \item 信号丢失检测:若`audio\_Out`最大幅值小于1e-6,弹出警告并将`audio\_Out`重置为原始音频`audio\_In`;
    \item 幅值归一化:`audio\_Out = audio\_Out / max(abs(audio\_Out))`,将幅值归一化到[-1,1];
    \item 音量提升:`audio\_Out = audio\_Out * gain`,将归一化后的音频乘以增益系数0.9,提升音量。
\end{enumerate}

\subsubsection{步骤6:频谱对比绘制(验证信号)}
\begin{enumerate}
    \item 创建图形窗口:`figure('Color','w')`设置白色背景,开启`hold on`和网格;
    \item 绘制频谱:调用自定义子函数`plotSpectrum`分别绘制原始音频(蓝色)和处理后音频(红色)的单侧边频谱;
    \item 设置坐标轴:聚焦人声核心频段`xlim([0 3000])`,设置纵轴范围`ylim([0 0.03])`,添加标题、坐标轴标签和图例;

\end{enumerate}

\subsubsection{步骤7:音频保存与播放}
\begin{enumerate}
    \item 保存音频:调用`audiowrite('to\_male.wav', audio\_Out, fs)`将处理后音频保存为`to\_male.wav`;
    \item 输出提示:`disp(' 音频已保存为 to\_male.wav')`,提示保存完成。
\end{enumerate}

\subsubsection{子函数:plotSpectrum(绘制单侧边频谱)}
\begin{enumerate}
    \item 输入参数:`audio`(音频数据)、`fs`(采样率)、`label`(图例标签)、`color`(线条颜色);
    \item 计算音频长度:`L = length(audio)`;
    \item 计算FFT并转换为单侧边频谱:按1.2.2节的频谱分析公式计算$P_1$和$f_p$;
    \item 绘制频谱:`plot(fp, P1, color, 'LineWidth', 1.2, 'DisplayName', label)`,设置线宽和图例。
\end{enumerate}

\subsubsection{算法核心特性}
\begin{enumerate}
    \item 问题修复:替换硬高通滤波为温和高频增强,避免核心频段丢失导致的声音消失;优化窗长和共振峰参数,避免音色闷/憨;
    \item 鲁棒性:增加音频有效性校验、信号丢失异常处理,避免程序报错;
    \item 可验证性:绘制频谱对比图,直观验证处理后核心频段信号存在;
    \item 实用性:输出可播放的音频文件,参数可灵活调整以平衡“男声感”和“声音清晰度”。
\end{enumerate}

\subsection{程序代码及运行结果}

\subsubsection{程序代码}

程序源码如下:


\begin{lstlisting}[language=Matlab, 
                  caption={genderConversion}, 
                        % 指定代码语言(此处为MATLAB)
  basicstyle=\ttfamily\small,  % 等宽字体+小字号(Markdown常用)
  backgroundcolor=\color{gray!5},  % 浅灰背景(接近多数Markdown渲染效果)
  frame=none,           % 无边框(Markdown代码块通常无框)
  keywordstyle=\color{blue},  % 关键字高亮(蓝色,可选)
  commentstyle=\color{green},  % 注释高亮(绿色,可选)
  showstringspaces=false,  % 不显示字符串中的空格标记
  numbers=none,         % 不显示行号(Markdown默认无行号)
  breaklines=true,      % 自动换行(避免溢出)
  columns=fullflexible  ] 
clear; clc; close all;
%% 1. 读取音频(兼容立体声/单声道)
[audioIn, fs] = audioread('../name.wav');  
if size(audioIn, 2) > 1
    audioIn = mean(audioIn, 2);  % 转为单声道
end
% 检查原始音频是否有效
if max(abs(audioIn)) < 1e-6
    error('原始音频文件无信号,请检查name.wav是否存在且有效!');
end

%% 2. 核心调整参数(平衡"男声感"和"不憨+有声音")
nsemitones = -9.5;           % 基频:-8(适度降频,保留声音)
preserve_formants = false;  % 保留共振峰(核心:避免闷/憨,且不丢声音)
gain = 0.9;                % 提高音量(避免声音小)
high_freq_boost = 1.05;     % 温和高频增强(不滤掉核心频段)

%% 3. 核心:调整基频(恢复合理窗长,避免信号丢失)
win_len = 1024;  % 恢复1024窗长(512太小易导致信号处理失真)
audioOut = shiftPitch(audioIn, nsemitones, ...
    'PreserveFormants', preserve_formants, ...
    'Window', hamming(win_len), ...
    'OverlapLength', round(win_len * 0.75), ...
    'LockPhase', true);

%% 4. 温和高频增强(替代硬高通,不丢声音)
% 问题根源:2kHz高通滤掉了人声核心频段→声音消失!
% 修正:用"低通+高频增益",仅增强高频,不滤掉低频/中频
% 设计1000Hz~5000Hz的带通增强(人声清晰频段)
[b, a] = butter(2, [1000 5000]/(fs/2), 'bandpass'); % 2阶带通
audio_high = filtfilt(b, a, audioOut);              % 提取高频
audioOut = audioOut + (audio_high * (high_freq_boost - 1)); % 温和增强高频

%% 5. 音量归一化(避免失真+确保声音大小)
% 增加防除零:若信号全零,直接用原始音频(避免报错)
if max(abs(audioOut)) < 1e-6
    warning('处理后信号丢失,使用原始音频!');
    audioOut = audioIn;
end
audioOut = audioOut / max(abs(audioOut));  % 归一化到[-1,1]
audioOut = audioOut * gain;                % 提升音量

%% 6. 绘制频谱对比(验证信号存在)
figure('Color','w'); hold on; grid on;
plotSpectrum(audioIn, fs, '原始女声', 'b');
plotSpectrum(audioOut, fs, '调整后男声(有声音+不憨)', 'r');

title('音频频谱对比(修复声音消失问题)');
xlabel('频率 (Hz)'); ylabel('幅值');
xlim([0 3000]);  % 聚焦人声核心频段(0-3kHz)
ylim([0 0.03]);
legend('Location','best');
hold off;

%% 7. 保存+播放音频
audiowrite('to_male.wav', audioOut, fs);
disp('音频已保存为 to_male.wav');


%% 子函数:绘制单侧边频谱
function plotSpectrum(audio, fs, label, color)
    L = length(audio);
    ffta = fft(audio);
    P2 = abs(ffta/L);
    P1 = P2(1:L/2+1);
    P1(2:end-1) = 2*P1(2:end-1);
    fp = fs*(0:(L/2))/L;
    plot(fp, P1, color, 'LineWidth', 1.2, 'DisplayName', label);
end
\end{lstlisting}
\subsubsection{运行结果}

音频变调后结果如下图\ref{fig:audio_spectrum}:

\begin{figure}[H]
    \centering
    % 配图占位符,实际使用时替换为真实图片路径
    \includegraphics[width=0.9\textwidth]{./matlab/genderConverse/voiceContrast.png}
    \caption{女声转男声音频频谱对比(0~3kHz核心频段)}
    \label{fig:audio_spectrum}
\end{figure}

可见变调后频谱整体左移,低频分量增加,高频分量减小。

输出音频保存在`./matlab/genderConverse/to\_male.wav`

\newpage
\section{音频变速算法设计与实现}
\subsection{设计思路}
本算法基于时域分帧重叠相加原理实现音频变速,核心逻辑是通过调整输出帧的重叠长度改变帧移,在保持音频音调不变的前提下调整播放时长,避免频域处理的复杂度与失真问题。

\subsubsection{背景知识}

\subsubsection{背景知识}
\begin{enumerate}
    \item 音频变速的时域实现逻辑:变速的核心是改变帧移(帧间步进长度),而非修改采样率(采样率不变可保证音调不偏移);加速时增大帧移,减速时减小帧移,是时域变速的通用思路。
    \item 分帧参数的通用计算:帧长通常由固定时间窗(如10ms)与采样率换算($frame\_Length = frame\_Time \times fs$),重叠率(25\%~75\%)需根据场景适配——低重叠率(如25\%)提升计算效率,高重叠率提升信号连续性。
    \item 重叠相加法的重构原理:输出信号长度由帧长、帧数、输出重叠长度共同决定,每帧按输出帧移叠加到对应位置,重叠区域幅值累加可保证时域连续性,避免拼接毛刺。
    \item 窗函数的通用作用:汉明窗是音频分帧处理的标配窗函数,可抑制频谱泄漏,同时使帧间叠加更平滑;窗函数需与帧矩阵维度匹配(行/列向量对齐),避免维度错误。
    \item 变速算法的兼容性:时域重叠相加法无需频域变换,计算复杂度低,可兼容加速(rate>1)与减速(rate<1)场景,仅需调整输出重叠率即可适配不同变速比率。
    \begin{figure}
      \centering
      \includegraphics[width=0.8\textwidth]{./changeSpeed.png}
      \caption{分帧与复原}
    \end{figure}
\end{enumerate}


\subsubsection{技术原理}
\begin{enumerate}
    \item 分帧参数计算原理:
    \begin{enumerate}
        \item 输入帧长(点数):$frame\_Length = frame\_Time \times fs$(frame\_Time=0.010s为固定帧时间,fs为采样率);
        \item 输入重叠长度:$overlapLength = \text{floor}(frame\_Length \times overlapRate)$(overlapRate=0.25为输入重叠率);
        \item 输入帧数:$frameNumber = \text{floor}((\text{length}(y)-frame\_Length)/(frame\_Length-overlapLength)) + 1$,确保所有有效输入信号被分帧覆盖;
        \item 输出重叠率自适应:$outputOverlapRate = \frac{rate + overlapRate - 1}{rate}$,rate>1时输出重叠率降低,输出帧移增大,总时长缩短(实现加速);
        \item 输出重叠长度:$outputOverlapLength = \text{floor}(frame\_Length \times outputOverlapRate)$。
    \end{enumerate}
    \item 加窗原理:采用汉明窗抑制频谱泄漏,汉明窗公式为:
    \[
    w(n) = 0.54 - 0.46\cos\left(\frac{2\pi n}{N-1}\right), \quad n=0,1,...,N-1
    \]
    其中$N=frame\_Length$为帧长,窗函数与帧矩阵按元素相乘实现加窗。
    \item 重叠相加重构原理:输出信号长度为$frame\_Length + (frameNumber -1) \times (frame\_Length - outputOver\\lapLength)$,每帧信号按输出帧移($frame\_Length - outputOverlapLength$)叠加到对应位置,重叠区域幅值累加,保证信号连续。
    \item 变速核心逻辑:rate>1时,输出帧移($frame\_Length - outputOverlapLength$)大于输入帧移($frame\_Length - overlapLength$),相同帧数的信号在更短的时长内完成叠加,实现音频加速;采样率保持不变(Fs=fs),保证音调不发生偏移。
\end{enumerate}

\subsection{具体实现过程}
算法通过MATLAB函数`speed\_change`实现音频变速,具体步骤如下:

\subsubsection{步骤1:基本参数计算与初始化}
\begin{enumerate}
    \item 设定固定帧时间:$frame\_Time = 0.010$(单位:秒,对应10ms帧长);
    \item 计算输入帧长(点数):$frame\_Length = frame\_Time \times fs$;
    \item 设定输入重叠率:$overlapRate = 0.25$,计算输入重叠长度:$overlapLength = \text{floor}(frame\_Length \times overlapRate)$;
    \item 计算输入总帧数:$frameNumber = \text{floor}((\text{length}(y)-frame\_Length)/(frame\_Length-overlapLength)) + 1$;
    \item 计算输出重叠率:$outputOverlapRate = (rate + overlapRate - 1)/rate$,并推导输出重叠长度:$outputOverlapLength = \text{floor}(frame\_Length \times outputOverlapRate)$。
\end{enumerate}

\subsubsection{步骤2:输入信号分帧}
\begin{enumerate}
    \item 初始化帧矩阵$frames$(维度:$frameNumber \times frame\_Length$),初始值全为0;
    \item 遍历每帧($i=1:frameNumber$),计算当前帧的信号索引:
          起始索引:$(i-1) \times (frame\_Length - overlapLength) + 1$,
          结束索引:$(i-1) \times (frame\_Length - overlapLength) + frame\_Length$;
    \item 将输入信号$y$的对应索引段赋值给帧矩阵$frames(i, :)$,完成分帧。
\end{enumerate}

\subsubsection{步骤3:帧加窗处理}
\begin{enumerate}
    \item 生成汉明窗$window$:$window = \text{hamming}(frame\_Length)'$(转置为行向量,匹配帧矩阵维度);
    \item 帧矩阵与窗函数按元素相乘:$windowedFrames = frames .* window$,完成每帧的加窗处理,抑制频谱泄漏。
\end{enumerate}

\subsubsection{步骤4:重叠相加重构输出信号}
\begin{enumerate}
    \item 初始化输出信号$Y$:维度为$1 \times (frame\_Length + (frameNumber -1) \times (frame\_Length - outputOverlapLength))$,初始值全为0;
    \item 遍历每帧($i=1:frameNumber$),计算当前帧在输出信号中的起始索引:$startIndex = (i-1) \times (frame\_Length - outputOverlapLength) + 1$;
    \item 将加窗后的帧$windowedFrames(i, :)$叠加到$Y$的$startIndex : startIndex + frame\_Length -1$区间,重叠区域幅值累加;
    \item 遍历完成后,$Y$即为重构后的变速音频信号。
\end{enumerate}

\subsubsection{步骤5:采样率赋值}
\begin{enumerate}
    \item 将输出采样率$Fs$赋值为输入采样率$fs$,保证音频音调不随速度变化。
\end{enumerate}

\subsubsection{算法核心特性}
\begin{enumerate}
    \item 音调保持:采样率全程不变(Fs=fs),仅通过调整帧重叠长度改变播放时长,保证变速后音调不偏移;
    \item 低复杂度:基于时域分帧重叠相加实现,无需频域变换,计算效率高;
    \item 低失真:汉明窗抑制频谱泄漏,重叠相加保证帧间连续性,避免信号断裂/毛刺;
    \item 灵活性:变速比率$rate$可灵活调整,rate>1实现加速,rate<1可实现减速(算法兼容);
    \item 鲁棒性:帧数计算基于整数取整,确保分帧覆盖有效信号,无索引越界风险。
\end{enumerate}

\subsection{程序代码和运行结果}

\subsubsection{程序代码}

程序代码包含两部分:
\begin{enumerate}
  \item `speed\_change`:  包含一个函数,输入一个音频和一个标量(加速倍率)返回加减速后的音频
  \item `runSpeedChange`: 控制流,控制读入音频,加减速音频和保存结果
\end{enumerate}

\begin{lstlisting}[language=Matlab, 
                  caption={speed\_change}, 
                        % 指定代码语言(此处为MATLAB)
  basicstyle=\ttfamily\small,  % 等宽字体+小字号(Markdown常用)
  backgroundcolor=\color{gray!5},  % 浅灰背景(接近多数Markdown渲染效果)
  frame=none,           % 无边框(Markdown代码块通常无框)
  keywordstyle=\color{blue},  % 关键字高亮(蓝色,可选)
  commentstyle=\color{green},  % 注释高亮(绿色,可选)
  showstringspaces=false,  % 不显示字符串中的空格标记
  numbers=none,         % 不显示行号(Markdown默认无行号)
  breaklines=true,      % 自动换行(避免溢出)
  columns=fullflexible  ] 

function [Y, Fs] = speed_change(y, fs, rate)
%  Y 输出信号,向量形式
%  Fs 输出采样率
%  y 输入信号,向量形式
%  fs 输入采样率
%  rate 变数比率, >1 加速

% 基本参数设置
    
    frameTime = 0.010; % 帧长 ms
    frameLength = frameTime * fs; % 帧长 点数
    overlapRate = 0.25; % 重叠率
    overlapLength = floor(frameLength * overlapRate); % 重叠点数
    frameNumber = floor((length(y)- frameLength) / (frameLength - overlapLength)) + 1; % 帧数
    outputOverlapRate =  (rate + overlapRate - 1) / rate; % 输出重叠率
    outputOverlapLength = floor(frameLength * outputOverlapRate); % 输出重叠点数
    
    % 分帧
    frames = zeros(frameNumber, frameLength);
    for i = 1:frameNumber
        frames(i, :) = y((i-1)*(frameLength - overlapLength) + 1 : (i-1)*(frameLength - overlapLength) + frameLength);
    end
    
    % 加窗
    window = hamming(frameLength)';
    windowedFrames = frames .* window;
    
    % 重叠相加
    Y = zeros(1, frameLength+ (frameNumber -1)*(frameLength - outputOverlapLength));
    for i = 1:frameNumber
        startIndex = (i-1)*(frameLength - outputOverlapLength) + 1;
        Y(startIndex : startIndex + frameLength -1) = Y(startIndex : startIndex + frameLength -1) + windowedFrames(i, :);
    end
    
    
    Fs = fs;
    
end
\end{lstlisting}



\begin{lstlisting}[language=Matlab, 
                  caption={runSpeedChange}, 
                        % 指定代码语言(此处为MATLAB)
  basicstyle=\ttfamily\small,  % 等宽字体+小字号(Markdown常用)
  backgroundcolor=\color{gray!5},  % 浅灰背景(接近多数Markdown渲染效果)
  frame=none,           % 无边框(Markdown代码块通常无框)
  keywordstyle=\color{blue},  % 关键字高亮(蓝色,可选)
  commentstyle=\color{green},  % 注释高亮(绿色,可选)
  showstringspaces=false,  % 不显示字符串中的空格标记
  numbers=none,         % 不显示行号(Markdown默认无行号)
  breaklines=true,      % 自动换行(避免溢出)
  columns=fullflexible  ] 

% 基本信息
audiopath =  '../name.wav'; 
outputpathFast = './fastname.wav'
outputpathSlow = './slowname.wav'

[y,Fs] = audioread(audiopath); 
[Yf, Fsf] = speed_change(y, Fs, 2); 
audiowrite(outputpathFast, Yf, Fsf); 

[Ys, Fss] = speed_change(y, Fs, 0.5);
audiowrite(outputpathSlow, Ys, Fss);

%输出基本信息
disp('Speed change processing completed.');

\end{lstlisting}

\subsubsection{代码运行结果}
输出音频存储在`./matlab/changeSpeed/`
包含
\begin{itemize}
  \item `slowname.wav`: 减速后的音频
  \item `fastname.wav`:加速后的音频
\end{itemize}


% 整体作为文档的一个Section,适配层级要求
\newpage
\section{基于能量阈值的语音段自动分割与反转拼接算法实现}
\subsection{设计思路}
本算法针对数字语音音频(1\~10段)实现“自动分割-段起始前移-反转拼接”全流程处理,核心逻辑是通过短时能量分析识别语音段,自适应调整能量阈值匹配目标分割段数,再对语音段进行前移修正和反转拼接,最终输出反转后的音频并可视化验证。

\subsubsection{背景知识}

\subsubsection{背景知识}
\begin{enumerate}
    \item 语音端点检测的核心依据:语音段的短时能量远高于静音段,滑动窗口计算短时能量(均方根能量)是端点检测的经典方案,窗口时长(20ms)是语音处理的通用取值,可精准区分语音/静音。
    \item 阈值自适应的鲁棒性设计:固定能量阈值易受音频信噪比、说话人音量影响,导致分割结果偏离目标;通过小幅迭代调整阈值(如±10\%),并限制最大调整次数(如3次),是平衡效果与收敛性的通用策略。
    \item 语音段过滤的通用规则:需设置最小段长(如100ms)过滤误分割的短段,避免无效段干扰;段起始前移(如0.25秒)可补偿能量上升延迟,避免丢失语音段开头的有效信息。
    \item 音频拼接的时序完整性:拼接时需保留原始静音间隔,保证反转后音频的时序逻辑与原音频一致;静音间隔包括段间间隔与最后一段后的间隔,是拼接类算法的核心考量。
    \item 可视化验证的通用价值:绘制原始/处理后音频波形,标记关键特征点(如段起始),可直观验证分割与拼接效果;设置中文显示适配(如SimHei字体)是可视化的基础要求。
\end{enumerate}


\subsubsection{技术原理}
\begin{enumerate}
    \item 短时能量计算原理:对音频每个采样点,取以其为中心的20ms滑动窗口,计算窗口内音频的均方根能量,公式为:
    \[
    E(i) = \sqrt{\frac{1}{N} \sum_{n=start\_idx}^{end\_idx} x(n)^2}
    \]
    其中$N$为窗口内采样点数,$start\_idx = \max(1, i-\text{half\_win})$,$end\_idx = \min(\text{total\_samples}, i+\text{half\_win})$,$x(n)$为音频采样值;
    \item 能量阈值分割:设定能量阈值$\text{energy\_thresh}$,当$E(i) > \text{energy\_thresh}$时标记为语音段起始,当$E(i) \leq \text{energy\_thresh}$时标记为语音段结束,同时过滤长度小于$\text{min\_seg\_len}$的短段(避免误分割);
    \item 阈值自适应调整:若分割段数少于目标值,将阈值降低10\%;若多于目标值,将阈值升高10\%,最多调整3次保证算法收敛;
    \item 段起始前移:将每个语音段的起始索引前移$\text{shift\_samples} = \text{fs} \times \text{shift\_time}$个采样点(fs为采样率,shift\_time为前移时间),且保证前移后索引≥1;
    \item 反转拼接:先提取各语音段间的静音间隔,将语音段和间隔分别反转,再按“反转语音段+反转静音间隔”的顺序拼接,保证反转后音频的静音间隔与原音频一致。
\end{enumerate}

\subsection{具体实现过程}
算法通过MATLAB脚本实现语音段分割与反转拼接,输入为数字语音音频文件,输出为反转拼接后的音频文件及可视化结果,具体步骤如下:

\subsubsection{步骤1:环境初始化与参数配置}
\begin{enumerate}
    \item 环境清理:执行`clear; clc; close all;`清空变量、清除命令行、关闭所有图形窗口;
    \item 配置核心参数:构建结构体`cfg`,包含以下参数:
          \begin{enumerate}
              \item `cfg.audio\_path = '../number.wav'`:输入音频文件路径;
              \item `cfg.shift\_time = 0.25`:段起始前移时间(秒);
              \item `cfg.energy\_thresh = 0.05`:初始能量阈值;
              \item `cfg.win\_size\_ms = 20`:滑动窗口时长(毫秒);
              \item `cfg.min\_seg\_len\_ms = 100`:最小语音段长度(毫秒);
              \item `cfg.target\_seg\_num = 10`:目标分割段数(匹配1~10数字语音)。
          \end{enumerate}
\end{enumerate}

\subsubsection{步骤2:音频读取与预处理}
\begin{enumerate}
    \item 读取音频:调用`audioread(cfg.audio\_path)`返回音频数据`audio\_In`和采样率`fs`;
    \item 立体声转单声道:`audio\_In = mean(audio\_In, 2)`,保证音频为列向量;
    \item 幅值归一化:`audio\_In = audio\_In / max(abs(audio\_In))`,将幅值归一化到[-1,1];
    \item 计算音频基本信息:总采样数`total\_samples = length(audio\_In)`,总时长`total\_duration = total\_samples / fs`,并打印音频信息。
\end{enumerate}

\subsubsection{步骤3:短时能量计算}
\begin{enumerate}
    \item 窗口参数计算:滑动窗口采样数`win\_size = round(fs * cfg.win\_size\_ms / 1000)`,窗口半长`half\_win = floor(win\_size / 2)`;
    \item 初始化能量数组:`audio\_energy = zeros(total\_samples, 1)`,存储每个采样点的短时能量;
    \item 遍历每个采样点计算短时能量:
          \begin{enumerate}
              \item 计算窗口起始索引:`start\_idx = max(1, i - half\_win)`;
              \item 计算窗口结束索引:`end\_idx = min(total\_samples, i + half\_win)`;
              \item 截取窗口内音频:`window\_audio = audio\_In(start\_idx:end\_idx)`;
              \item 计算均方根能量:`audio\_energy(i) = sqrt(mean(window\_audio.\^2))`。
          \end{enumerate}
\end{enumerate}

\subsubsection{步骤4:语音段自动分割}
\begin{enumerate}
    \item 最小段长转换:`min\_seg\_len = round(fs * cfg.min\_seg\_len\_ms / 1000)`,将毫秒转换为采样点数;
    \item 初始分割语音段:
          \begin{enumerate}
              \item 初始化语音段矩阵`segments`(存储起始/结束索引)、起始标记`start\_idx = 0`;
              \item 遍历采样点,当能量大于阈值且无起始标记时,标记`start\_idx = i`;
              \item 当能量小于等于阈值且有起始标记时,标记结束索引`end\_idx = i`,若段长大于最小段长则存入`segments`;
              \item 遍历结束后,若仍有未结束的语音段且长度达标,补充存入`segments`。
          \end{enumerate}
    \item 自适应调整阈值:
          \begin{enumerate}
              \item 初始化调整次数`adjust\_count = 0`,最大调整次数`max\_adjust\_times = 3`;
              \item 若分割段数≠目标段数且未达最大调整次数:
                    - 段数不足:阈值×0.9,扩大识别范围;
                    - 段数过多:阈值×1.1,缩小识别范围;
              \item 重新分割语音段,更新段数`num\_seg = size(segments, 1)`;
              \item 调整结束后,若段数仍不匹配,输出警告及调整建议;匹配则打印成功提示。
          \end{enumerate}
\end{enumerate}

\subsubsection{步骤5:语音段起始前移修正}
\begin{enumerate}
    \item 计算前移采样数:`shift\_samples = round(fs * cfg.shift\_time)`;
    \item 初始化调整后段矩阵`segments\_adjusted = segments`;
    \item 遍历每个语音段,将起始索引前移:`segments\_adjusted(i, 1) = max(1, segments\_adjusted(i, 1) - shift\_samples)`,保证索引不越界。
\end{enumerate}

\subsubsection{步骤6:提取音频间隔(静音区)}
\begin{enumerate}
    \item 初始化间隔数组`intervals\_all`;
    \item 提取段间间隔:遍历1~num\_seg-1段,计算`interval\_between = segments\_adjusted(i+1, 1) - segments\_adjusted(i, 2)`,存入`intervals\_all`;
    \item 提取最后一段后的间隔:`interval\_after\_last = total\_samples - segments\_adjusted(end, 2)`,补充存入`intervals\_all`。
\end{enumerate}

\subsubsection{步骤7:反转拼接音频}
\begin{enumerate}
    \item 反转语音段和间隔:`reversed\_segments = flipud(segments\_adjusted)`,`reversed\_intervals = flipud(intervals\_all)`;
    \item 初始化输出音频`audioOut = []`;
    \item 遍历反转后的语音段:
          \begin{enumerate}
              \item 截取当前段音频:`seg\_audio = audio\_In(reversed\_segments(i, 1):reversed\_segments(i, 2))`;
              \item 拼接语音段:`audioOut = [audioOut; seg\_audio]`;
              \item 生成对应静音间隔:`silence = zeros(reversed\_intervals(i), 1)`;
              \item 拼接静音间隔:`audioOut = [audioOut; silence]`。
          \end{enumerate}
    \item 音量归一化:`audioOut = audioOut / max(abs(audioOut)) * 0.9`,避免失真且保证音量适中。
\end{enumerate}

\subsubsection{步骤8:音频保存与可视化}
\begin{enumerate}
    \item 保存音频:调用`audiowrite('numberConverse.wav', audioOut, fs)`,输出保存提示;
    \item 创建可视化窗口:`figure('Color','w','Position',[100 100 1000 700],'Name','音频反转结果可视化')`;
    \item 绘制原始音频波形:
          \begin{enumerate}
              \item 子图1(2,1,1):绘制原始音频时域波形,标记原段起始(红色虚线)和前移后段起始(绿色实线),标注段序号;
              \item 设置坐标轴、网格、图例,保证中文显示正常。
          \end{enumerate}
    \item 绘制反转后音频波形:
          \begin{enumerate}
              \item 子图2(2,1,2):绘制反转后音频时域波形,标记反转后段起始(蓝色虚线),标注反转后段序号;
              \item 设置坐标轴、网格,保证中文显示正常。
          \end{enumerate}
    \item 保存可视化图片:`saveas(gcf, 'numberConverse\_visualization.png')`;

\end{enumerate}

\subsubsection{算法核心特性}
\begin{enumerate}
    \item 自适应分割:通过能量阈值微调,自动匹配目标分割段数,降低人工调参成本;
    \item 鲁棒性强:段起始前移避免丢失语音开头信息,最小段长过滤误分割短段,阈值调整次数限制避免无限循环;
    \item 可视化验证:绘制原始/反转音频波形并标记语音段,直观验证分割和拼接效果;
    \item 低失真:全程仅做幅值归一化,无频域处理,保证语音音色完整性;
    \item 可扩展性:参数可灵活调整(如前移时间、窗口时长、目标段数),适配不同语音场景。
\end{enumerate}

\subsection{程序代码及运行结果}
\subsubsection{ 程序代码}

\begin{lstlisting}[language=Matlab, 
                  caption={信号波形生成与绘图代码}, 
                        % 指定代码语言(此处为MATLAB)
  basicstyle=\ttfamily\small,  % 等宽字体+小字号(Markdown常用)
  backgroundcolor=\color{gray!5},  % 浅灰背景(接近多数Markdown渲染效果)
  frame=none,           % 无边框(Markdown代码块通常无框)
  keywordstyle=\color{blue},  % 关键字高亮(蓝色,可选)
  commentstyle=\color{green},  % 注释高亮(绿色,可选)
  showstringspaces=false,  % 不显示字符串中的空格标记
  numbers=none,         % 不显示行号(Markdown默认无行号)
  breaklines=true,      % 自动换行(避免溢出)
  columns=fullflexible  ] 
clear; clc; close all;

%% 基础参数
cfg = struct();
cfg.audio_path     = '../number.wav';       % 输入音频文件
cfg.shift_time     = 0.25;              % 段起始前移时间(秒)
cfg.energy_thresh  = 0.05;              % 能量阈值,控制语音段识别
cfg.win_size_ms    = 20;                % 滑动窗口时长(毫秒)
cfg.min_seg_len_ms = 100;               % 最小语音段长度(毫秒)
cfg.target_seg_num = 10;                % 目标分割段数

[audioIn, fs] = audioread(cfg.audio_path);
audioIn = mean(audioIn, 2);            
audioIn = audioIn / max(abs(audioIn)); 
total_samples = length(audioIn);
total_duration = total_samples / fs;

fprintf(' 音频信息:\n');
fprintf(' 采样率:%d Hz | 总时长:%.2f 秒 | 总样本数:%d\n', fs, total_duration, total_samples);
fprintf('_______________________________________________________________\n');

win_size = round(fs * cfg.win_size_ms / 1000);  
half_win = floor(win_size / 2);
audio_energy = zeros(total_samples, 1);         

for i = 1:total_samples
    start_idx = max(1, i - half_win);
    end_idx   = min(total_samples, i + half_win);
    window_audio = audioIn(start_idx:end_idx);
    audio_energy(i) = sqrt(mean(window_audio.^2));
end

%%  语音段自动分割
min_seg_len = round(fs * cfg.min_seg_len_ms / 1000); 
segments = [];          % 语音段存储矩阵
start_idx = 0;          % 语音段起始标记

for i = 1:total_samples
    if audio_energy(i) > cfg.energy_thresh && start_idx == 0
        start_idx = i;
    elseif audio_energy(i) <= cfg.energy_thresh && start_idx ~= 0
        end_idx = i;
        if (end_idx - start_idx) > min_seg_len
            segments = [segments; start_idx, end_idx];
        end
        start_idx = 0;
    end
end
if start_idx ~= 0 && (total_samples - start_idx) > min_seg_len
    segments = [segments; start_idx, total_samples];
end

% 微调阈值以匹配目标段数
num_seg = size(segments, 1);
max_adjust_times = 3;  % 最大微调次数
adjust_count = 0;

while num_seg ~= cfg.target_seg_num && adjust_count < max_adjust_times
    adjust_count = adjust_count + 1;
    if num_seg < cfg.target_seg_num
        cfg.energy_thresh = cfg.energy_thresh * 0.9;
        fprintf(' 第%d次微调:段数不足(%d段),能量阈值降至%.3f\n', adjust_count, num_seg, cfg.energy_thresh);
    else
        cfg.energy_thresh = cfg.energy_thresh * 1.1;
        fprintf(' 第%d次微调:段数过多(%d段),能量阈值升至%.3f\n', adjust_count, num_seg, cfg.energy_thresh);
    end
    
    segments = []; start_idx = 0;
    for i = 1:total_samples
        if audio_energy(i) > cfg.energy_thresh && start_idx == 0, start_idx = i; end
        if audio_energy(i) <= cfg.energy_thresh && start_idx ~= 0
            if (i - start_idx) > min_seg_len, segments = [segments; start_idx, i]; end
            start_idx = 0;
        end
    end
    if start_idx ~= 0 && (total_samples - start_idx) > min_seg_len
        segments = [segments; start_idx, total_samples];
    end
    num_seg = size(segments, 1);
end

% 最终段数检查
if num_seg ~= cfg.target_seg_num
    if num_seg < cfg.target_seg_num
        tip = '调小energy_thresh(如0.04)';
    else
        tip = '调大energy_thresh(如0.06)';
    end
    warning(' 最终分割出%d段(目标%d段)→ %s', num_seg, cfg.target_seg_num, tip);
else
    fprintf(' 成功分割出%d段语音(匹配1~10)\n', cfg.target_seg_num);
end

%%  语音段起始前移
shift_samples = round(fs * cfg.shift_time);  
segments_adjusted = segments;
for i = 1:num_seg
    segments_adjusted(i, 1) = max(1, segments_adjusted(i, 1) - shift_samples);
end

%%  提取原音频所有间隔
intervals_all = [];
for i = 1:num_seg-1
    interval_between = segments_adjusted(i+1, 1) - segments_adjusted(i, 2);
    intervals_all = [intervals_all; interval_between];
end
interval_after_last = total_samples - segments_adjusted(end, 2);
intervals_all = [intervals_all; interval_after_last];

%%  拼接音频 
reversed_segments = flipud(segments_adjusted);
reversed_intervals = flipud(intervals_all);

audioOut = [];
for i = 1:num_seg
    seg_audio = audioIn(reversed_segments(i, 1):reversed_segments(i, 2));
    audioOut = [audioOut; seg_audio];
    silence = zeros(reversed_intervals(i), 1);
    audioOut = [audioOut; silence];
end

audioOut = audioOut / max(abs(audioOut)) * 0.9;

%%  保存最终音频 
output_path = 'numberConverse.wav';
audiowrite(output_path, audioOut, fs);
fprintf('音频已保存:%s\n', output_path);

%% 可视化绘图
figure('Color','w','Position',[100 100 1000 700],'Name','音频反转结果可视化');

% 原始音频
subplot(2,1,1);
t_raw = (0:total_samples-1)/fs;
plot(t_raw, audioIn, 'Color', [0.2 0.4 0.6], 'LineWidth', 1); hold on;
% 标记原始段和前移后的段
for i = 1:num_seg
    plot([segments(i,1)/fs, segments(i,1)/fs], ylim, 'Color', [0.8 0.2 0.2], 'LineStyle', '--', 'LineWidth', 1);
    plot([segments_adjusted(i,1)/fs, segments_adjusted(i,1)/fs], ylim, 'Color', [0.2 0.8 0.2], 'LineStyle', '-', 'LineWidth', 1.5);
    text(segments_adjusted(i,1)/fs, 0.8, num2str(i), 'Color', [0.2 0.8 0.2], 'FontSize', 9, 'FontWeight', 'bold');
end
title('原始音频', 'FontSize', 12, 'FontWeight', 'bold');
xlabel('时间(秒)', 'FontSize', 10); ylabel('幅值', 'FontSize', 10);
xlim([0, total_duration]); ylim([-1.1, 1.1]);
grid on; 
ax1 = gca;
ax1.LineWidth = 0.5;  
box on;
legend('原始音频','原段起始','前移后段起始','Location','best','FontSize',9);

% 反转音频
subplot(2,1,2);
t_rev = (0:length(audioOut)-1)/fs;
plot(t_rev, audioOut, 'Color', [0.8 0.4 0.2], 'LineWidth', 1); hold on;
rev_seg_times = [];
current_pos = 1;
for i = 1:num_seg
    seg_len = reversed_segments(i,2) - reversed_segments(i,1) + 1;
    seg_start = (current_pos - 1)/fs;
    seg_end = (current_pos + seg_len - 1)/fs;
    rev_seg_times = [rev_seg_times; seg_start, seg_end];
    plot([seg_start, seg_start], ylim, 'Color', [0.2 0.4 0.6], 'LineStyle', '--', 'LineWidth', 1);
    text((seg_start+seg_end)/2, 0.8, num2str(11-i), 'Color', [0.2 0.4 0.6], 'FontSize', 9, 'FontWeight', 'bold');
    current_pos = current_pos + seg_len + reversed_intervals(i);
end
title('反转后音频', 'FontSize', 12, 'FontWeight', 'bold');
xlabel('时间(秒)', 'FontSize', 10); ylabel('幅值', 'FontSize', 10);
xlim([0, t_rev(end)]); ylim([-1.1, 1.1]);
grid on;  
ax2 = gca;
ax2.LineWidth = 0.5; 
box on;

set(gcf, 'DefaultAxesFontName', 'SimHei');
set(gcf, 'DefaultTextFontName', 'SimHei');

saveas(gcf, 'numberConverse_visualization.png');

\end{lstlisting}


\subsubsection{运行结果}
程序运行结果如下图\ref{fig:number_converse_vis}
    \begin{figure}[htbp]
        \centering
        % 配图占位符,实际使用时替换为真实图片路径
        \includegraphics[width=0.9\textwidth]{./matlab/countConverse/numberConverse_visualization.png}
        \caption{语音段分割与反转拼接可视化结果}
        \label{fig:number_converse_vis}
    \end{figure}

图中标注了代码语音分割后每一段音频以及对应音频段所读的数字。

输出音频保存在`./matlab/countConverse/numberConverse.wav`

同时,另有一份代码和音频展示直接倒放音频的结果作为对比,保存在`./matlab/countConverse/directConverse.m`和`./matlab/countConverse/directConverse.wav`中。
其原理为直接反转离散音频序列并输出,此处不再赘述。
\end{document}

