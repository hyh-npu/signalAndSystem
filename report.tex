
% ! TEX root = ./report.tex
\documentclass[a4paper,oneside]{ctexbook}
\usepackage{subfiles}
\usepackage[left=2cm,right=2cm,top=2cm,bottom=2cm]{geometry} 
\usepackage{graphicx}   
\usepackage{listings}
\usepackage{xcolor}
\usepackage{amsmath}
\usepackage{booktabs}
\usepackage{subcaption}
\usepackage{caption}
\usepackage{multirow} 
\usepackage{float}
\usepackage{fancyhdr}
\usepackage{setspace}
\usepackage{textpos} 

\makeatletter
\def\addcontentsline#1#2#3{%
  \protected@write\@auxout{}%
    {\string\@writefile{#1}{\string\contentsline{#2}{#3}{\thepage}}}%
}
\makeatother

% ========== 调整section编号(无chapter前缀) ==========
\renewcommand{\thesection}{\arabic{section}}
\renewcommand{\thesubsection}{\thesection.\arabic{subsection}}

% ========== 原有配置保留 ==========
\counterwithin{figure}{section}
\counterwithin{table}{section}

\captionsetup[figure]{
    labelformat=simple,
    labelsep=period
}
\captionsetup[table]{
    labelformat=simple,
    labelsep=period
}

% ========== 页眉配置 ==========
\pagestyle{fancy}
\fancyhf{}
\renewcommand{\chaptermark}[1]{\markboth{#1}{}}
\renewcommand{\sectionmark}[1]{\markright{\thesection\quad#1}{}}
\fancyhead[R]{\bfseries\rightmark}
\fancyfoot[C]{\thepage}
\renewcommand{\headrulewidth}{0.4pt}
\renewcommand{\footrulewidth}{0pt}

\fancypagestyle{plain}{%
    \fancyhf{}%
    \fancyfoot[C]{\thepage}%
    \renewcommand{\headrulewidth}{0pt}%
}

% ========== hyperref ==========
\usepackage{hyperref}
\hypersetup{
    colorlinks=true,
    linkcolor=blue,
    anchorcolor=blue,
    citecolor=blue,
    urlcolor=blue,
    pdftitle={信号与系统实验报告}, 
    pdfauthor={韩耀辉}
}

\title{信号与系统实验报告}
\author{韩耀辉 2024302022}
\date{\today}

\begin{document}

% ========== 左上角logo ==========
\begin{titlepage}
    \pagestyle{empty} % 隐藏封面页码
    \centering
    \setstretch{1.2} % 全局行距

    % 左上角logo
    \begin{textblock*}{8cm}(0cm,0cm) 
        \includegraphics[width=6cm]{logo.png} 
    \end{textblock*}

    \vspace*{0.1\textheight}

    {\fontsize{16}{20}\kaishu 电子信息学院实验课程报告}
    \vspace{0.05\textheight}

    {\fontsize{36}{42}\heiti \textbf{信号与系统实验报告}}
    \vspace{0.03\textheight}

    \rule{0.5\textwidth}{1pt}
    \vspace{0.12\textheight}

    {\fontsize{12}{20}\selectfont
    \begin{tabular}{|l@{\hspace{1em}:\hspace{1em}}r|}
        \hline
        \textbf{姓名} & 此处填写姓名 \\
        \hline
        \textbf{学号} & 此处填写学号 \\
        \hline
        \textbf{班级} & U08P2108.0X \\
        \hline
        \textbf{学院} & X X X X学院 \\
        \hline
    \end{tabular}
    }
    \vspace{0.12\textheight}

    % 底部日期(楷体,适中字号)
    {\fontsize{14}{18}\kaishu 提交日期:\today}
    \vspace{0.05\textheight}

    % 底部装饰线(与标题线呼应)
    \rule{0.3\textwidth}{0.8pt}

    % 底部空行(压底)
    \vfill
\end{titlepage}

% ========== 目录生成:先清空旧目录,再生成 ==========
\newpage
\clearpage % 强制换页,避免目录写入冲突
\tableofcontents
\clearpage % 强制换页,闭合目录写入

% ========== 课程报告(无编号+安全的目录项) ==========
\chapter*{课程报告}
\addcontentsline{toc}{chapter}{课程报告} % 简化目录项,无特殊字符
\clearpage % 避免子文件干扰目录
\subfile{class1/class1.tex}
\newpage
\subfile{class2/class2.tex}
\newpage
\subfile{class3/class3.tex}
\newpage
\subfile{class4/class4.tex}
\newpage
\subfile{class5/class5.tex}
\newpage
\subfile{class6/class6.tex}
\newpage
\subfile{class7/class7.tex}

% ========== 大作业报告(无编号+安全的目录项) ==========
\chapter*{大作业报告(第八次实验)}
\addcontentsline{toc}{chapter}{大作业报告(第八次实验)} % 简化目录项,无特殊字符
\clearpage % 避免子文件干扰目录
\subfile{homework/homework.tex}

% ========== 强制闭合所有写入流 ==========
\clearpage
\end{document}

