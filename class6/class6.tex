% ! TEX root = ../report.tex

\documentclass[../report.tex]{subfile} 

\begin{document}

\section{信号卷积实验}

\subsection{实验目的}
1. 理解卷积的概念及物理意义。\\
2. 通过实验的方法加深对卷积运算的图解方法及结果的理解。

\subsection{实验器材}
\begin{itemize}
    \item 双踪示波器:1台
    \item 信号源及频率计模块(S2模块):1块
    \item 数字信号处理模块(S4模块):1块
\end{itemize}

\subsection{实验原理}
\subsubsection{两个矩形脉冲信号的卷积过程}
两信号均为矩形脉冲信号,其卷积运算可通过图解法完成,最终结果需与实验结果进行对比。图6-1展示了两矩形脉冲的卷积积分运算过程与结果。

\begin{figure}[H]
    \centering
    \includegraphics[width=0.7\textwidth]{./6-1.png}
    \caption{两矩形脉冲的卷积积分的运算过程与结果}
    \label{fig:rect_conv_process}
\end{figure}

\paragraph{图解法的一般步骤}
1. 置换:将变量 \(t\) 替换为 \(\tau\),即 \(f_1(t) \to f_1(\tau)\),\(f_2(t) \to f_2(\tau)\);\\
2. 反褶:将 \(\tau\) 替换为 \(-\tau\),即 \(f_2(\tau) \to f_2(-\tau)\);\\
3. 平移:将 \(-\tau\) 替换为 \(t-\tau\),即 \(f_2(-\tau) \to f_2(t-\tau)\);\\
4. 相乘:计算 \(f_1(\tau) \cdot f_2(t-\tau)\);\\
5. 积分:计算 \(\int_{-\infty}^{+\infty} f_1(\tau) f_2(t-\tau) d\tau\)。

\paragraph{不同占空比矩形波的自卷积过程}
1. 占空比50\%的矩形波自卷积过程:通过上述图解法步骤,最终得到的卷积结果为三角形脉冲信号;\\
2. 占空比25\%的矩形波自卷积过程:同样遵循图解法步骤,卷积结果的波形形态随占空比变化而改变。

\subsubsection{矩形脉冲信号与锯齿波信号的卷积}
信号 \(f_1(t)\) 为锯齿波信号,\(f_2(t)\) 为矩形脉冲信号,如图6-2所示。根据卷积积分的运算方法,可得到两者的卷积积分结果 \(y(t)\),具体波形如图6-2(c)所示。

\begin{figure}[htbp]
    \centering
    \includegraphics[width=0.7\textwidth]{./6-2.png}
    \caption{矩形脉冲信号与锯齿波信号的卷积积分结果}
    \label{fig:rect_saw_conv}
\end{figure}

\paragraph{占空比50\%的矩形波与锯齿波的互卷积过程}
按照卷积图解法的五个步骤依次运算,最终得到的卷积结果为分段线性的连续信号,其波形特征与矩形波和锯齿波的参数相关。

\subsection{实验步骤}
\subsubsection{矩形脉冲信号的自卷积}
1. 模块关电,连接信号源及频率计模块S2的模拟输出P2和数字信号处理模块S4的P9;\\
2. 模块开电,调节信号源模块使P2输出方波信号:扫频开关拨至“OFF”,调节“ROL1”使方波频率为500Hz,调节“模拟输出幅度调节W1”使幅度为1V;然后长按“ROL1”旋钮约2秒钟,旋转调节“ROL1”,使数码管上显示数据“50”(即占空比为50\%);\\
3. 将拨动开关SW1调整为“00000010”,设置为自卷积功能;\\
4. 按下复位键S2;\\
5. 将示波器的探头CH1接于TP2,探头CH2接于TP1,对比观察占空比为50\%的输入信号与卷积后输出信号波形,照片记录在表\ref{tab:self_conv_data}中;\\
6. 改变矩形波占空比:长按信号源模块的“频率调节”旋钮切换到方波占空比设置功能,旋转“频率调节”旋钮,将P2输出矩形波的占空比调整至25\%和75\%;用示波器探头1观测信号源模块的P2,确认占空比变化;用示波器探头2观测数字信号处理模块的TP1,记录卷积信号输出的变化情况,照片填入表\ref{tab:self_conv_data};\\
7. 改变矩形波的幅度:调节信号源模块的“模拟输出幅度调节W1”使P9输入幅度为2V;用示波器分别观测信号源模块的P2和数字信号处理模块的TP1,记录自卷积输出变化情况,照片填入表\ref{tab:self_conv_data}。

\begin{table}[H]
    \centering
    \caption{矩形脉冲信号自卷积实验数据记录}
    \label{tab:self_conv_data}
    \begin{tabular}{ccc}
        \toprule
        信号条件 & TP2波形 & TP1波形 \\
        \midrule
        占空比50\%,幅度1V &  &  \\
        占空比25\%,幅度1V &  &  \\
        占空比75\%,幅度1V &  &  \\
        占空比50\%,幅度2V &  &  \\
        \bottomrule
    \end{tabular}
\end{table}

\subsubsection{矩形信号与矩形信号的互卷积}
激励信号为幅度1V、频率500Hz、占空比约为50\%的方波信号(由信号源模块提供并输入到S4模块的P9);将S4模块的S3开关第8位拨为1,此时S4模块内部产生频率500Hz、占空比50\%的方波信号(测试点为TP2),卷积输出测试点为TP1。具体步骤如下:
1. 模块关电,连接信号源及频率计模块S2的模拟输出P2和数字信号处理模块S4的P9;\\
2. 模块开电,调节信号源模块使P2输出方波信号:扫频开关拨至“OFF”,调节“ROL1”使方波频率为500Hz,调节“模拟输出幅度调节W1”使幅度为1V;长按“ROL1”旋钮约2秒钟,旋转调节“ROL1”,使数码管显示“50”(占空比50\%);\\
3. 将拨动开关SW1调整为“00000011”(互卷积功能),将拨码开关S3拨为“00000001”(TP2输出矩形波信号);\\
4. 按下复位键S2;\\
5. 将示波器探头CH1接于TP2,探头CH2接于TP1,对比观察两占空比50\%矩形信号的卷积输出波形,照片记录在表\ref{tab:rect_rect_conv_data}中;\\
6. 改变矩形波占空比:长按信号源模块的“频率调节”旋钮切换到占空比设置功能,旋转旋钮将P2输出矩形波的占空比调整至25\%和75\%;用示波器探头1观测S2模块的P2,确认占空比变化;用探头2观测S4模块的TP1,记录卷积输出变化情况,照片填入表\ref{tab:rect_rect_conv_data};\\
7. 改变矩形波的幅度:调节信号源模块的“模拟输出幅度调节W1”使P9输入幅度为2V;用示波器分别观测S2模块的P2和S4模块的TP1,记录互卷积输出变化情况,照片填入表\ref{tab:rect_rect_conv_data}。

\begin{table}[H]
    \centering
    \caption{矩形信号互卷积实验数据记录}
    \label{tab:rect_rect_conv_data}
    \begin{tabular}{ccc}
        \toprule
        信号条件 & TP2波形 & TP1波形 \\
        \midrule
        两信号占空比均为50\%,幅度1V &  &  \\
        输入信号占空比25\%,内部信号占空比50\%,幅度1V &  &  \\
        输入信号占空比75\%,内部信号占空比50\%,幅度1V &  &  \\
        两信号占空比均为50\%,幅度2V &  &  \\
        \bottomrule
    \end{tabular}
\end{table}

\subsubsection{矩形信号与锯齿波信号的互卷积}
激励信号为幅度1V、频率500Hz、占空比50\%的方波信号(由S2模块提供并输入到S4模块的P9);将S4模块的S3开关第8位拨为0,此时S4模块内部产生频率500Hz、占空比50\%的锯齿波信号(测试点为TP2),卷积输出测试点为TP1。具体步骤如下:
1. 模块关电,连接信号源及频率计模块S2的P2与数字信号处理模块S4的P9;\\
2. 模块开电,调节信号源上相应旋钮,使P2输出幅度1V、频率500Hz、占空比50\%的矩形波;\\
3. 将S4模块的拨动开关SW1调整为“00000011”(互卷积功能),将拨码开关S3拨为“00000000”(TP2输出锯齿波信号),按下复位键S2;\\
4. 用示波器探头连接S4模块的TP2,观测锯齿波波形;\\
5. 用示波器探头连接S4模块的TP1,观测卷积后输出信号的波形,照片记录在表\ref{tab:rect_saw_conv_data}中;\\
6. 改变矩形波占空比:长按信号源模块的“频率调节”旋钮切换到占空比设置功能,旋转旋钮将P2输出矩形波的占空比调整至25\%和75\%;观测P9输出矩形波的占空比变化,记录卷积输出变化情况,照片填入表\ref{tab:rect_saw_conv_data};\\
7. 改变锯齿波占空比:改变拨码开关S3的前7位,设置锯齿波输出为不同占空比信号;用示波器探头1观测S4模块的TP2,确认锯齿波占空比变化;用探头2观测S4模块的TP1,记录不同占空比下锯齿波与矩形波的互卷积输出变化情况,照片填入表\ref{tab:rect_saw_conv_data}。

\begin{table}[H]
    \centering
    \caption{矩形信号与锯齿波互卷积实验数据记录}
    \label{tab:rect_saw_conv_data}
    \begin{tabular}{ccc}
        \toprule
        信号条件 & TP2波形(S4模块) & TP1波形 \\
        \midrule
        矩形波、锯齿波占空比均为50\%,幅度1V &  &  \\
        矩形波占空比25\%,锯齿波占空比50\%,幅度1V &  &  \\
        矩形波占空比75\%,锯齿波占空比50\%,幅度1V &  &  \\
        矩形波占空比50\%,锯齿波占空比25\%,幅度1V &  &  \\
        \bottomrule
    \end{tabular}
\end{table}

\subsection{实验结果}


\subsubsection{矩形脉冲信号的自卷积}
自卷积信号的波形如下图\ref{fig:6-4-1}所示:
\begin{figure}[H]  % 补全[H]固定位置,与文档风格统一
  \centering
  % 每个subfigure必须指定宽度参数,这里用0.45\textwidth避免排版拥挤
  \begin{subfigure}{0.45\textwidth}
    \centering
    \includegraphics[width=\textwidth]{./6-4-11.png}  % 图片宽度适配subfigure
    \caption{占空比50\%,幅度1V}
  \end{subfigure}
  \hfill  % 子图之间添加水平间距
  \begin{subfigure}{0.45\textwidth}
    \centering
    \includegraphics[width=\textwidth]{./6-4-12.png}
    \caption{占空比25\%,幅度1V}
  \end{subfigure}
  
  \vspace{0.5cm}  % 换行添加垂直间距
  \begin{subfigure}{0.45\textwidth}
    \centering
    \includegraphics[width=\textwidth]{./6-4-21.png}
    \caption{占空比75\%,幅度1V}
  \end{subfigure}
  \hfill
  \begin{subfigure}{0.45\textwidth}
    \centering
    \includegraphics[width=\textwidth]{./6-4-22.png}
    \caption{占空比50\%,幅度2V}
  \end{subfigure}
  \caption{矩形脉冲信号自卷积波形}
  \label{fig:6-4-1}
\end{figure}

\subsubsection{矩形信号与矩形信号的互卷积}
互卷积信号的波形如下图\ref{fig:6-4-2}所示:
\begin{figure}[H]
  \centering
  \begin{subfigure}{0.45\textwidth}
    \centering
    \includegraphics[width=\textwidth]{./6-4-21.png}
    \caption{两信号占空比均为50\%,幅度1V}
  \end{subfigure}
  \hfill
  \begin{subfigure}{0.45\textwidth}
    \centering
    \includegraphics[width=\textwidth]{./6-4-22.png}
    \caption{输入信号占空比25\%,内部信号占空比50\%,幅度1V}
  \end{subfigure}
  
  \vspace{0.5cm}
  \begin{subfigure}{0.45\textwidth}
    \centering
    \includegraphics[width=\textwidth]{./6-4-23.png}
    \caption{输入信号占空比75\%,内部信号占空比50\%,幅度1V}
  \end{subfigure}
  \hfill
  \begin{subfigure}{0.45\textwidth}
    \centering
    \includegraphics[width=\textwidth]{./6-4-24.png}  % 修正笔误:6-4-44.png → 6-4-24.png
    \caption{两信号占空比均为50\%,幅度2V}
  \end{subfigure}
  \caption{矩形信号互卷积波形}
  \label{fig:6-4-2}
\end{figure}

\subsubsection{矩形信号与锯齿波信号的互卷积}
互卷积信号的波形如下图\ref{fig:6-4-3}所示:
\begin{figure}[H]
  \centering
  \begin{subfigure}{0.45\textwidth}
    \centering
    \includegraphics[width=\textwidth]{./6-4-31.png}
    \caption{矩形波、锯齿波占空比均为50\%,幅度1V}
  \end{subfigure}
  \hfill
  \begin{subfigure}{0.45\textwidth}
    \centering
    \includegraphics[width=\textwidth]{./6-4-32.png}
    \caption{矩形波占空比25\%,锯齿波占空比50\%,幅度1V}
  \end{subfigure}
  
  \vspace{0.5cm}
  \begin{subfigure}{0.45\textwidth}
    \centering
    \includegraphics[width=\textwidth]{./6-4-33.png}
    \caption{矩形波占空比75\%,锯齿波占空比50\%,幅度1V}
  \end{subfigure}
  \hfill
  \begin{subfigure}{0.45\textwidth}
    \centering
    \includegraphics[width=\textwidth]{./6-4-34.png}
    \caption{矩形波占空比50\%,锯齿波占空比25\%,幅度1V}
  \end{subfigure}
  \caption{矩形信号与锯齿波互卷积波形}
  \label{fig:6-4-3}
\end{figure}

\subsection{实验程序及运行结果}
\subsubsection{任务一程序及运行结果}
\begin{lstlisting}[language=Matlab, 
                  caption={信号波形生成与绘图代码}, 
                        % 指定代码语言(此处为MATLAB)
  basicstyle=\ttfamily\small,  % 等宽字体+小字号(Markdown常用)
  backgroundcolor=\color{gray!5},  % 浅灰背景(接近多数Markdown渲染效果)
  frame=none,           % 无边框(Markdown代码块通常无框)
  keywordstyle=\color{blue},  % 关键字高亮(蓝色,可选)
  commentstyle=\color{green},  % 注释高亮(绿色,可选)
  showstringspaces=false,  % 不显示字符串中的空格标记
  numbers=none,         % 不显示行号(Markdown默认无行号)
  breaklines=true,      % 自动换行(避免溢出)
  columns=fullflexible  ] 

delta = 0.01;
t = -10:delta:10;
f1 = heaviside(t) - heaviside(t-2);
f2 =  heaviside(t+3) - heaviside(t);

c12 = conv(f1, f2) * delta;
c11 = conv(f1, f1) * delta;
c22 = conv(f2, f2) * delta;
ctime = -20:delta:20;

figure;
subplot(2,3,1);
plot(t, f1);
title('f1(t)');

subplot(2,3,2);
plot(t, f2);
title('f2(t)');

subplot(2,3,4);
plot(ctime, c11);
title('f1(t) * f1(t)');

subplot(2,3,5);
plot(ctime, c12);
title('f1(t) * f2(t)');

subplot(2,3,6);
plot(ctime, c22);
title('f2(t) * f2(t)');

saveas(gcf, './p1.png');


\end{lstlisting}

运行结果如图\ref{fig:6-5-1}所示:
\begin{figure}[H]
  \centering
  \includegraphics[width=0.8\textwidth]{./matlab/p1.png}
  \caption{任务一运行结果}
  \label{fig:6-5-1}
\end{figure}

\subsubsection{任务二程序及运行结果}
\begin{lstlisting}[language=Matlab, 
                  caption={信号波形生成与绘图代码}, 
                        % 指定代码语言(此处为MATLAB)
  basicstyle=\ttfamily\small,  % 等宽字体+小字号(Markdown常用)
  backgroundcolor=\color{gray!5},  % 浅灰背景(接近多数Markdown渲染效果)
  frame=none,           % 无边框(Markdown代码块通常无框)
  keywordstyle=\color{blue},  % 关键字高亮(蓝色,可选)
  commentstyle=\color{green},  % 注释高亮(绿色,可选)
  showstringspaces=false,  % 不显示字符串中的空格标记
  numbers=none,         % 不显示行号(Markdown默认无行号)
  breaklines=true,      % 自动换行(避免溢出)
  columns=fullflexible  ] 

delta = 0.01;
t = -5:delta:5;
f1 =heaviside(t) - heaviside(t-2);
f2 =  heaviside(t) - heaviside(t-3) + heaviside(t-1) - heaviside(t-2);

c = conv(f1, f2) * delta;
ctime = -10:delta:10;

figure;
subplot(3,1,1);
plot(t, f1);
title('f1(t)');

subplot(3,1,2);
plot(t, f2);
title('f2(t)');

subplot(3,1,3);
plot(ctime, c);
title('f1(t) * f2(t)');

saveas(gcf, './p2.png');

\end{lstlisting}

运行结果如图\ref{fig:6-5-2}所示:

\begin{figure}[H]
  \centering
  \includegraphics[width=0.6\textwidth]{./matlab/p2.png}
  \caption{任务二运行结果}
  \label{fig:6-5-2}
\end{figure}

\subsubsection{任务三程序及运行结果}
\begin{lstlisting}[language=Matlab, 
                  caption={信号波形生成与绘图代码}, 
                        % 指定代码语言(此处为MATLAB)
  basicstyle=\ttfamily\small,  % 等宽字体+小字号(Markdown常用)
  backgroundcolor=\color{gray!5},  % 浅灰背景(接近多数Markdown渲染效果)
  frame=none,           % 无边框(Markdown代码块通常无框)
  keywordstyle=\color{blue},  % 关键字高亮(蓝色,可选)
  commentstyle=\color{green},  % 注释高亮(绿色,可选)
  showstringspaces=false,  % 不显示字符串中的空格标记
  numbers=none,         % 不显示行号(Markdown默认无行号)
  breaklines=true,      % 自动换行(避免溢出)
  columns=fullflexible  ] 

delta = 0.01;
t = -5:delta:5;
f1 = heaviside(t+1) - heaviside(t-1);
f2 = (t + 3).*(t>-3&t<-2) + 1.*(t>=-2&t<=0) + (1-t).*(t>0&t<1);

c = conv(f1, f2) * delta;
ctime = -10:delta:10;

figure;
subplot(3,1,1);
plot(t, f1);
title('f1(t)');

subplot(3,1,2);
plot(t, f2);
title('f2(t)');
title('f2(t)');

subplot(3,1,3);
plot(ctime, c);
title('f1(t) * f2(t)');

saveas(gcf, './p3.png');

\end{lstlisting}

运行结果如图\ref{fig:6-5-3}所示:

\begin{figure}[H]
  \centering
  \includegraphics[width=0.6\textwidth]{./matlab/p3.png}
  \caption{任务三运行结果}
  \label{fig:6-5-3}
\end{figure}

\subsubsection{任务四程序及运行结果}

\begin{lstlisting}[language=Matlab, 
                  caption={信号波形生成与绘图代码}, 
                        % 指定代码语言(此处为MATLAB)
  basicstyle=\ttfamily\small,  % 等宽字体+小字号(Markdown常用)
  backgroundcolor=\color{gray!5},  % 浅灰背景(接近多数Markdown渲染效果)
  frame=none,           % 无边框(Markdown代码块通常无框)
  keywordstyle=\color{blue},  % 关键字高亮(蓝色,可选)
  commentstyle=\color{green},  % 注释高亮(绿色,可选)
  showstringspaces=false,  % 不显示字符串中的空格标记
  numbers=none,         % 不显示行号(Markdown默认无行号)
  breaklines=true,      % 自动换行(避免溢出)
  columns=fullflexible  ] 

delta = 0.01;
t = -5:delta:5;
f1 = (t-1).*(heaviside(t-1)-heaviside(t-3));
f2 = heaviside(t+2) - 2*heaviside(t-2);

c = conv(f1, f2) * delta;
ctime = -10:delta:10;

figure;
subplot(3,1,1);
plot(t, f1);
title('f1(t)');

subplot(3,1,2);
plot(t, f2);
title('f2(t)');

subplot(3,1,3);
plot(ctime, c);
title('f1(t) * f2(t)');

saveas(gcf, './p4.png');

\end{lstlisting}

运行结果如图\ref{fig:6-5-4}所示:

\begin{figure}[H]
  \centering
  \includegraphics[width=0.6\textwidth]{./matlab/p4.png}
  \caption{任务四运行结果}
  \label{fig:6-5-4}
\end{figure}



\end{document}
