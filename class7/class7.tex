% ! TEX root=../report.tex

\documentclass[../report.tex]{subfiles}
\begin{document}
\section{波形的分解与合成实验}
\subsection{实验目的}
1. 了解和熟悉波形分解与合成原理;
2. 了解和掌握用傅里叶级数进行谐波分析的方法。

\subsection{实验器材}
\begin{itemize}
    \item 双踪示波器:1台
    \item 数字万用表:1台
    \item 信号源及频率计模块(S2模块):1块
    \item 数字信号处理模块(S4模块):1块
\end{itemize}

\subsection{实验原理}
\subsubsection{信号的频谱与测量}
信号的时域特性和频域特性是对信号的两种不同描述方式。对于满足狄利克莱(Dirichlet)条件的周期信号 \(f(t)\),可展开为三角形式的傅里叶级数,在区间 \([-T/2, T/2]\) 内表示为:
\[f(t)=a_0+\sum_{n=1}^{\infty}\left(a_n \cos n \Omega t+b_n \sin n \Omega t\right)\]
其中 \(a_0\) 为直流分量,\(a_n\)、\(b_n\) 为各次谐波的系数,\(\Omega = 2\pi/T\) 为基波角频率。

信号的时域与频域特性的内在联系如图7-1所示:图(a)为幅度-时间-频率三维图形,图(b)为时域波形图,图(c)为振幅频谱图。周期信号的振幅频谱具有离散性、谐波性、收敛性三大性质。

\begin{figure}[H]
    \centering
    \includegraphics{./7-1.png}
    \caption{信号的时域特性和频域特性}
    \label{fig:time_freq_char}
\end{figure}

测量方法分为同时分析法和顺序分析法,本实验采用同时分析法,其原理是利用多个滤波器,将中心频率分别调至被测信号的各次谐波频率,可在信号发生的实际时间内同时测得所有频率分量,原理框图如图7-2所示。

\begin{figure}[H]
    \centering
    \includegraphics{./7-2.png}
    \caption{用同时分析法进行频谱分析的原理框图}
    \label{fig:spectrum_analysis_principle}
\end{figure}

\subsubsection{方波信号的分解(占空比50\%)}
占空比50\%的方波信号在一个周期内的解析式为:
\[f(t)=\begin{cases}A & 0 < t \leq \frac{T}{2} \\ -A & \frac{T}{2} < t \leq T\end{cases}\]
其傅里叶级数系数推导如下:
\[
\begin{aligned}
B_{k\omega} &= \frac{4}{T} \int_{0}^{T/2} A \sin k\omega t \, dt \\
&= \frac{4A}{T k\omega} \left[ -\cos k\omega t \right]_{0}^{T/2} \\
&= \frac{4A}{k\pi} \quad (k=1,3,5,7,\cdots)
\end{aligned}
\]
因此,傅里叶级数展开式为:
\[f(t)=\frac{4 A}{\pi}\left(\sin \omega t+\frac{1}{3} \sin 3 \omega t+\frac{1}{5} \sin 5 \omega t+\frac{1}{7} \sin 7 \omega t+\cdots\right)\]
可见该信号仅含1、3、5、7……等奇次谐波分量,偶次谐波分量为0。当信号峰峰值为2V(即 \(A=1\))时,各次谐波分量的峰值理论值如下表所示:

\begin{table}[H]
    \centering
    \caption{占空比50\%方波信号各次谐波峰值理论值(峰峰值2V)}
    \label{tab:50duty_square_harmonic}
    \begin{tabular}{cc}
        \toprule
        谐波次数 & 峰值理论值(V) \\
        \midrule
        1次(基波) & 1.2732395 \\
        3次 & 0.4244131 \\
        5次 & 0.2546479 \\
        7次 & 0.1818914 \\
        \bottomrule
    \end{tabular}
\end{table}

\subsubsection{矩形信号的分解(占空比40\%)}
矩形信号占空比为40%时,满足 \(\tau = 2T/5\)(\(\tau\) 为脉冲宽度,\(T\) 为周期),该信号为偶函数,其角频率 \(\omega = \frac{2\pi}{T}\)。

傅里叶级数相关系数推导:
\[a_0 = \frac{1}{T} \int_{-\tau/2}^{\tau/2} A \, dt = \frac{A\tau}{T}\]
\[a_n = \frac{2}{T} \int_{-\tau/2}^{\tau/2} A \cos(n\omega t) \, dt = \frac{2A}{n\pi} \sin\left(\frac{n\pi\tau}{T}\right) = \frac{2A}{n\pi} \sin\left(\frac{2n\pi}{5}\right)\]

三角级数形式:
\[f(t)=\frac{2}{5} A+\frac{2 A T}{5 \pi} \sum_{n=1}^{\infty} S a\left(\frac{n \omega T}{5}\right) \cos(n\omega t)\]

指数形式:
\[f(t)=\frac{2A}{5} \sum_{n=-\infty}^{\infty} S a\left(\frac{n\omega T}{5}\right) e^{jn\omega t}\]

当信号峰峰值为2V(\(A=1\))时,各次谐波分量的峰值理论值如下表所示:

\begin{table}[H]
    \centering
    \caption{占空比40\%矩形信号各次谐波峰值理论值(峰峰值2V)}
    \label{tab:40duty_rect_harmonic}
    \begin{tabular}{cc}
        \toprule
        谐波次数 & 峰值理论值(V) \\
        \midrule
        基波(1次) & 1.2109 \\
        2次 & 0.3742 \\
        3次 & 0.2495(相位与基波相反) \\
        4次 & 0.3027(相位与基波相反) \\
        5次 & 0 \\
        6次 & 0.2018 \\
        7次 & 0.1069 \\
        \bottomrule
    \end{tabular}
\end{table}

\subsubsection{三角波信号的分解}
三角波信号在一个周期内的解析式为:
\[f(t)= \begin{cases}\frac{4 A}{T} t & 0 \leq t \leq \frac{T}{4} \\ -\frac{4 A}{T} t+2 A & \frac{T}{4} \leq t \leq \frac{T}{2}\end{cases}\]

利用积分公式 \(\int x \sin ax \, dx = \frac{1}{a^2} \sin ax - \frac{1}{a} x \cos ax\),推导谐波系数:
\[
B_{k\omega} = 
\begin{cases}
\frac{8A}{k^2\pi^2} & k=1,5,9,\cdots \\
-\frac{8A}{k^2\pi^2} & k=3,7,11,\cdots
\end{cases}
\]

傅里叶级数展开式为:
\[f(t)=\frac{8 A}{\pi^2}\left(\sin \omega t - \frac{1}{3^2} \sin 3\omega t + \frac{1}{5^2} \sin 5\omega t - \frac{1}{7^2} \sin 7\omega t + \cdots\right)\]

当 \(A=1\) 时,各次谐波分量的峰值理论值如下表所示:

\begin{table}[H]
    \centering
    \caption{三角波信号各次谐波峰值理论值(\(A=1\))}
    \label{tab:triangular_harmonic}
    \begin{tabular}{cc}
        \toprule
        谐波次数 & 峰值理论值(V) \\
        \midrule
        1次(基波) & 0.81056941 \\
        3次 & 0.09006321 \\
        5次 & 0.03242261 \\
        7次 & 0.0165422 \\
        \bottomrule
    \end{tabular}
\end{table}

\subsubsection{信号的分解提取与合成}
\paragraph{信号分解提取}
进行信号分解和提取是滤波系统的一项基本任务。当对信号的某些分量感兴趣时,可利用选频滤波器提取有用部分,滤除其他部分。数字信号处理模块(S4模块)内置8个滤波器(1个低通、6个带通、1个高通),可将复杂信号分解为各次谐波分量,分解后的8路信号分别从测试点TP1~TP8输出(TP1~TP7对应1~7次谐波,TP8对应8次以上高次谐波)。

\paragraph{信号合成}
矩形脉冲信号通过8路滤波器输出的各次谐波分量,DSP将选中的谐波分量相加后从TP8输出。谐波的叠加组合通过S4模块的8位拨码开关S3控制(闭合为参与合成):第1位对应基波,第2位对应二次谐波,依此类推,8位开关均闭合时各次谐波全部参与合成。分解前的原信号可通过TP9观测,此时合成波形应与原信号一致。

图7-3为MATLAB仿真的方波谐波合成结果,直观展示了不同谐波叠加次数对合成波形的影响:

\begin{figure}[H]
    \centering
    \includegraphics{7-3}
    \caption{方波(500Hz、2V、50\%占空比)谐波合成仿真结果}
    \label{fig:square_synthesis_simulation}
    % 替换tablenotes,用普通文本注释,避免依赖threeparttable包
    \small\centering  % 小号字体+居中对齐
    左1图:1次谐波;右1图:1+3次谐波;左2图:1+3+5次谐波;右2图:1+3+5+7次谐波;左3图:1+3+5+7+9次谐波;右3图:1+3+5+7+9+11次谐波
\end{figure}

\subsection{实验步骤}
\subsubsection{方波的分解}
1. 模块关电,连接信号源及频率计模块S2的信号输出端口P2与数字信号处理模块S4的P9端口;
2. 模块开电,设置S2模块,使P2输出幅度2V、频率500Hz的方波(占空比调为50%);
3. 将S4模块的拨动开关SW1调整为“00000101”,按下复位键S2,选择矩形信号分解及合成功能;将拨码开关S3拨为“00000000”;
4. 用示波器分别观测S4模块TP1~TP7输出的1~7次谐波波形及TP8输出的8次以上高次谐波波形,拍照记录;
5. 保持信号幅度2V、频率500Hz不变,将方波占空比调整为40%,重复步骤4的观测与记录;
6. 按表7-4、表7-5要求,记录两种占空比下方波信号各次谐波的测量值(电压峰峰值)。

\begin{table}[H]
    \centering
    \caption{占空比50\%(\(\tau/T=1/2\))矩形脉冲信号的频谱(\(f=500Hz\),\(E=2V\))}
    \label{tab:50duty_spectrum_data}
    \begin{tabular}{ccccccccc}
        \toprule
        谐波频率 & \(1f\) & \(2f\) & \(3f\) & \(4f\) & \(5f\) & \(6f\) & \(7f\) & \(8f\)以上 \\
        \midrule
        理论值(电压峰峰值,V) &  &  &  &  &  &  &  &  \\
        测量值(电压峰峰值,V) &  &  &  &  &  &  &  &  \\
        \bottomrule
    \end{tabular}
\end{table}

\begin{table}[H]
    \centering
    \caption{占空比40\%(\(\tau/T=2/5\))矩形脉冲信号的频谱(\(f=500Hz\),\(E=2V\))}
    \label{tab:40duty_spectrum_data}
    \begin{tabular}{ccccccccc}
        \toprule
        谐波频率 & \(1f\) & \(2f\) & \(3f\) & \(4f\) & \(5f\) & \(6f\) & \(7f\) & \(8f\)以上 \\
        \midrule
        理论值(电压峰峰值,V) &  &  &  &  &  &  &  &  \\
        测量值(电压峰峰值,V) &  &  &  &  &  &  &  &  \\
        \bottomrule
    \end{tabular}
\end{table}

\subsubsection{方波的合成}
本任务仅做占空比50%、频率400Hz的方波合成:
1. 保持S4模块SW1为“00000101”,将S2模块设置为输出幅度2V、频率400Hz、占空比50%的方波;
2. 用示波器观测S4模块TP8(合成输出)和S2模块P2(原信号),按表7-6要求设置拨码开关S3,观察并记录各组合的合成波形;
3. 每次切换S3状态后,对比合成波形与原信号的差异,重点观察谐波叠加次数对合成波形逼近原信号的影响。

\begin{table}[H]
    \centering
    \caption{矩形脉冲信号的各次谐波合成要求}
    \label{tab:harmonic_synthesis_requirements}
    \begin{tabular}{cc}
        \toprule
        拨码开关S3状态 & 合成要求 \\
        \midrule
        10000000 & 基波单独合成 \\
        11000000 & 基波+二次谐波合成 \\
        10100000 & 基波+三次谐波合成 \\
        10010000 & 三次+五次谐波合成 \\
        10001000 & 基波+五次谐波合成 \\
        10101000 & 基波+三次+五次谐波合成 \\
        11111111 & 所有谐波合成 \\
        11011111 & 无三次谐波的其他谐波合成 \\
        11110111 & 无五次谐波的其他谐波合成 \\
        01111111 & 无八次以上高次谐波的其他谐波合成 \\
        \bottomrule
    \end{tabular}
\end{table}

\subsubsection{三角波的分解与合成}
1. 模块关电,保持P2与P9的连接,S4模块SW1仍设为“00000101”;
2. 模块开电,设置S2模块,使P2输出幅度2V、频率400Hz的三角波;
3. 三角波分解:将S3拨为“00000000”,用示波器观测TP1~TP7的1、3、5、7次谐波波形,拍照记录,按表7-7记录各次谐波的测量值;
4. 三角波合成:按表7-8要求设置拨码开关S3,观测TP8的合成波形,拍照记录,对比不同合成组合与原三角波的差异。

\begin{table}[H]
    \centering
    \caption{三角波信号的频谱(\(f=400Hz\),\(A=1V\))}
    \label{tab:triangular_spectrum_data}
    \begin{tabular}{ccccccccc}
        \toprule
        谐波频率 & \(1f\) & \(2f\) & \(3f\) & \(4f\) & \(5f\) & \(6f\) & \(7f\) & \(8f\)以上 \\
        \midrule
        理论值(电压峰峰值,V) &  &  & &  &  &  &  &  \\
        测量值(电压峰峰值,V) &  &  &  &  &  &  &  &  \\
        \bottomrule
    \end{tabular}
\end{table}

\begin{table}[H]
    \centering
    \caption{三角波信号的各次谐波合成要求}
    \label{tab:triangular_synthesis_requirements}
    \begin{tabular}{cc}
        \toprule
        拨码开关S3状态 & 合成要求 \\
        \midrule
        10000000 & 基波单独合成 \\
        10100000 & 基波+三次谐波合成 \\
        10001000 & 基波+五次谐波合成 \\
        10101000 & 基波+三次+五次谐波合成 \\
        10101010 & 基波+三次+五次+七次谐波合成 \\
        11111111 & 所有谐波合成 \\
        11011111 & 无三次谐波的其他谐波合成 \\
        11110111 & 无五次谐波的其他谐波合成 \\
        11111101 & 无七次谐波的其他谐波合成 \\
        \bottomrule
    \end{tabular}
\end{table}

\subsection{实验结果}
% 和各个实验步骤对应
\subsubsection{方波的分解}
\paragraph{占空比50\%方波分解结果}
分解后各次谐波波形如图\ref{fig:7-5-1-1}所示:
\begin{figure}[H]
    \centering
    \begin{subfigure}
      {0.45\textwidth}
      \centering
      \includegraphics[width=\textwidth]{7-5-1-1-1.png}
      \caption{1次谐波}
    \end{subfigure}
    \begin{subfigure}
      {0.45\textwidth}
      \centering
      \includegraphics[width=\textwidth]{7-5-1-1-2.png}
      \caption{3次谐波}
    \end{subfigure}
    \begin{subfigure}
      {0.45\textwidth}
      \centering
      \includegraphics[width=\textwidth]{7-5-1-1-3.png}
      \caption{5次谐波}
    \end{subfigure}
    \begin{subfigure}
      {0.45\textwidth}
      \centering
      \includegraphics[width=\textwidth]{7-5-1-1-4.png}
      \caption{7次谐波}
    \end{subfigure}
    \caption{占空比50\%方波分解后各次谐波波形}
    \label{fig:7-5-1-1}
\end{figure}
各次谐波测量值记录在表\ref{tab:7-5-1-1}中。


\begin{table}[H]
    \centering
    \caption{占空比50\%(\(\tau/T=1/2\))矩形脉冲信号的频谱(\(f=500Hz\),\(E=2V\))}
    \label{tab:7-5-1-1}
    \begin{tabular}{ccccccccc}
        \toprule
        谐波频率 & \(1f\) & \(2f\) & \(3f\) & \(4f\) & \(5f\) & \(6f\) & \(7f\) & \(8f\)以上 \\
        \midrule
        理论值(电压峰峰值,V) & 2.546 & 0 & 0.849 & 0 & 0.509 & 0 & 0.364 & - \\
        测量值(电压峰峰值,V) & 2.060 & 0.320 & 0.940 & 0.140 & 0.620 & 0.140 & 0.500 & - \\
        \bottomrule
    \end{tabular}
\end{table}

\paragraph{占空比40\%方波分解结果}
分解后各次谐波波形如图\ref{fig:7-5-1-2}所示:
\begin{figure}
  \centering
  \begin{subfigure}
    {0.4\textwidth}
    \centering
    \includegraphics[width=\textwidth]{7-5-1-2-1.png}
    \caption{1次谐波}
  \end{subfigure}
  \begin{subfigure}
    {0.4\textwidth}
    \centering
    \includegraphics[width=\textwidth]{7-5-1-2-2.png}
    \caption{2次谐波}
  \end{subfigure}
  \begin{subfigure}
    {0.4\textwidth}
    \centering
    \includegraphics[width=\textwidth]{7-5-1-2-2.png}
    \caption{3次谐波}
  \end{subfigure}
  \begin{subfigure}
    {0.4\textwidth}
    \centering
    \includegraphics[width=\textwidth]{7-5-1-2-4.png}
    \caption{4次谐波}
  \end{subfigure}
  \begin{subfigure}
    {0.4\textwidth}
    \centering
    \includegraphics[width=\textwidth]{7-5-1-2-3.png}
    \caption{5次谐波}
  \end{subfigure}
  \begin{subfigure}
    {0.4\textwidth}
    \centering
    \includegraphics[width=\textwidth]{7-5-1-2-6.png}
    \caption{6次谐波}
  \end{subfigure}
  \begin{subfigure}
    {0.4\textwidth}
    \centering
    \includegraphics[width=\textwidth]{7-5-1-2-4.png}
    \caption{7次谐波}
  \end{subfigure}
  \begin{subfigure}
    {0.4\textwidth}
    \centering
    \includegraphics[width=\textwidth]{7-5-1-2-8.png}
    \caption{8次以上高次谐波}
  \end{subfigure}
  \caption{占空比40\%方波分解后各次谐波波形}
  \label{fig:7-5-1-2}
\end{figure}

各次谐波测量值记录在表\ref{tab:7-5-1-2}中。

\begin{table}[H]
    \centering
    \caption{占空比40\%(\(\tau/T=2/5\))矩形脉冲信号的频谱(\(f=500Hz\),\(E=2V\))}
    \label{tab:7-5-1-2}
    \begin{tabular}{ccccccccc}
        \toprule
        谐波频率 & \(1f\) & \(2f\) & \(3f\) & \(4f\) & \(5f\) & \(6f\) & \(7f\) & \(8f\)以上 \\
        \midrule
        理论值(电压峰峰值,V) & 2.421 & 0.748 & 0.499 & 0.606 & 0.000 & 0.404 & 0.214 & - \\
        测量值(电压峰峰值,V) & 2.240 & 0.740 & 0.620 & 0.660 & 0.180 & 0.500 & 0.380 & - \\
        \bottomrule
    \end{tabular}
\end{table}

\subsubsection{方波的合成}
方波合成结果如图\ref{fig:7-5-2}所示:

% 对应表格     \label{tab:triangular_synthesis_requirements}
\begin{figure}
    \centering
    \begin{subfigure}
      {0.4\textwidth}
      \centering
      \includegraphics[width=\textwidth]{7-5-2-1.png}
      \caption{基波单独合成}
    \end{subfigure}
    \begin{subfigure}
      {0.4\textwidth}
      \centering
      \includegraphics[width=\textwidth]{7-5-2-2.png}
      \caption{基波+二次谐波合成}
    \end{subfigure}
    \begin{subfigure}
      {0.4\textwidth}
      \centering
      \includegraphics[width=\textwidth]{7-5-2-3.png}
      \caption{基波+三次谐波合成}
    \end{subfigure}
    \begin{subfigure}
      {0.4\textwidth}
      \centering
      \includegraphics[width=\textwidth]{7-5-2-4.png}
      \caption{三次+五次谐波合成}
    \end{subfigure}
    \begin{subfigure}
      {0.4\textwidth}
      \centering
      \includegraphics[width=\textwidth]{7-5-2-5.png}
      \caption{基波+五次谐波合成}
    \end{subfigure}
    \begin{subfigure}
      {0.4\textwidth}
      \centering
      \includegraphics[width=\textwidth]{7-5-2-6.png}
      \caption{基波+三次+五次谐波合成}
    \end{subfigure}
    \begin{subfigure}
      {0.4\textwidth}
      \centering
      \includegraphics[width=\textwidth]{7-5-2-7.png}
      \caption{所有谐波合成}
    \end{subfigure}
    \begin{subfigure}
      {0.4\textwidth}
      \centering
      \includegraphics[width=\textwidth]{7-5-2-8.png}
      \caption{无三次谐波的其他谐波合成}
    \end{subfigure}
    \caption{方波各次谐波合成结果}
    \label{fig:7-5-2}
\end{figure}

\subsubsection{三角波的分解与合成}
\paragraph{三角波分解结果}
三角波分解后各次谐波波形如图\ref{fig:7-5-3-1}所示:
\begin{figure}
  \centering
  \begin{subfigure}
    {0.45\textwidth}
    \centering
    \includegraphics[width=\textwidth]{7-5-3-1-1.png}
    \caption{1次谐波}
  \end{subfigure}
  \begin{subfigure}
    {0.45\textwidth}
    \centering
    \includegraphics[width=\textwidth]{7-5-3-1-2.png}
    \caption{3次谐波}
  \end{subfigure}
  \begin{subfigure}
    {0.45\textwidth}
    \centering
    \includegraphics[width=\textwidth]{7-5-3-1-3.png}
    \caption{5次谐波}
  \end{subfigure}
  \begin{subfigure}
    {0.45\textwidth}
    \centering
    \includegraphics[width=\textwidth]{7-5-3-1-4.png}
    \caption{7次谐波}
  \end{subfigure}
  \caption{三角波分解后各次谐波波形}
  \label{fig:7-5-3-1}
\end{figure}

各次谐波测量值记录在表\ref{tab:7-5-3-1}中。

\begin{table}[H]
    \centering
    \caption{三角波信号的频谱(\(f=400Hz\),\(A=2V\))}
    \label{tab:7-5-3-1}
    \begin{tabular}{ccccccccc}
        \toprule
        谐波频率 & \(1f\) & \(2f\) & \(3f\) & \(4f\) & \(5f\) & \(6f\) & \(7f\) & \(8f\)以上 \\
        \midrule
        理论值(电压峰峰值,V) & 1.621 & 0 & 0.180 & 0 & 0.065 & 0 & 0.033 & - \\
        测量值(电压峰峰值,V) & 1.180 & & 0.300 & & 0.220 & & 0.180 & - \\
        \bottomrule
    \end{tabular}
\end{table}

\paragraph{三角波合成结果}
% 对应表格      \label{tab:triangular_synthesis_requirements}
三角波合成结果如图\ref{fig:7-5-3-2}所示:
\begin{figure}
  \centering
  \begin{subfigure}
    {0.4\textwidth}
    \centering
    \includegraphics[width=\textwidth]{7-5-3-2-1.png}
    \caption{基波单独合成}
  \end{subfigure}
  \begin{subfigure}
    {0.4\textwidth}
    \centering
    \includegraphics[width=\textwidth]{7-5-3-2-2.png}
    \caption{基波+三次谐波合成}
  \end{subfigure}
  \begin{subfigure}
    {0.4\textwidth}
    \centering
    \includegraphics[width=\textwidth]{7-5-3-2-3.png}
    \caption{基波+五次谐波合成}
  \end{subfigure}
  \begin{subfigure}
    {0.4\textwidth}
    \centering
    \includegraphics[width=\textwidth]{7-5-3-2-4.png}
    \caption{基波+三次+五次谐波合成}
  \end{subfigure}
  \begin{subfigure}
    {0.4\textwidth}
    \centering
    \includegraphics[width=\textwidth]{7-5-3-2-5.png}
    \caption{基波+三次+五次+七次谐波合成}
  \end{subfigure}
  \begin{subfigure}
    {0.4\textwidth}
    \centering
    \includegraphics[width=\textwidth]{7-5-3-2-6.png}
    \caption{所有谐波合成}
  \end{subfigure}
  \begin{subfigure}
    {0.4\textwidth}
    \centering
    \includegraphics[width=\textwidth]{7-5-3-2-7.png}
    \caption{无三次谐波的其他谐波合成}
  \end{subfigure}
  \begin{subfigure}
    {0.4\textwidth}
    \centering
    \includegraphics[width=\textwidth]{7-5-3-2-8.png}
    \caption{无五次谐波的其他谐波合成}
  \end{subfigure}
  \caption{三角波各次谐波合成结果}
  \label{fig:7-5-3-2}
\end{figure}

\newpage
\subsection{实验程序及运行结果}

\textbf{以下程序中,采样点数N均取64,频谱图中,传统方法计算结果用红色曲线表示,FFT计算结果用蓝色杆状图表示。}

\subsubsection{任务一}
任务一代码如下所示:

\begin{lstlisting}[language=Matlab, 
                  caption={信号波形生成与绘图代码}, 
                        % 指定代码语言(此处为MATLAB)
  basicstyle=\ttfamily\small,  % 等宽字体+小字号(Markdown常用)
  backgroundcolor=\color{gray!5},  % 浅灰背景(接近多数Markdown渲染效果)
  frame=none,           % 无边框(Markdown代码块通常无框)
  keywordstyle=\color{blue},  % 关键字高亮(蓝色,可选)
  commentstyle=\color{green},  % 注释高亮(绿色,可选)
  showstringspaces=false,  % 不显示字符串中的空格标记
  numbers=none,         % 不显示行号(Markdown默认无行号)
  breaklines=true,      % 自动换行(避免溢出)
  columns=fullflexible  ] 
t_length = 20;
N = 64;

T = t_length/N;
t = (0:N-1)*T - t_length/2;

x = sin(2 .* pi .*(t - 1))./(pi .* (t - 1));

w_length = 2*pi/T;
W = w_length/N;
w = (0:N-1)*W;
disp('Computing FFT...');
tic
X = T*fft(x,N);
X = fftshift(X);
toc
x_num = x;
w_num = linspace(-w_length/2, w_length/2, N);
disp('Computing Numerical FT...');
tic
W_num = x_num .* exp(-1j * w_num' * t);
toc

figure;
subplot(2,1,1);
stem(t,x);
title('Signal x(t)');

subplot(2,1,2);
stem(w-w_length/2,abs(X));
hold on;
plot(w_num, abs(sum(W_num,2))*T, 'r');
title('Magnitude Spectrum of x(t)');
hold off;


saveas(gcf, './p1.png');
\end{lstlisting}

运行结果如图\ref{fig:7-task1_result}所示:
\begin{figure}[H]
    \centering
    \includegraphics[width=0.8\textwidth]{./matlab/p1.png}
    \caption{任务一运行结果}
    \label{fig:7-task1_result}
\end{figure}

\subsubsection{任务二}

任务二代码如下所示:


\begin{lstlisting}[language=Matlab, 
                  caption={信号波形生成与绘图代码}, 
                        % 指定代码语言(此处为MATLAB)
  basicstyle=\ttfamily\small,  % 等宽字体+小字号(Markdown常用)
  backgroundcolor=\color{gray!5},  % 浅灰背景(接近多数Markdown渲染效果)
  frame=none,           % 无边框(Markdown代码块通常无框)
  keywordstyle=\color{blue},  % 关键字高亮(蓝色,可选)
  commentstyle=\color{green},  % 注释高亮(绿色,可选)
  showstringspaces=false,  % 不显示字符串中的空格标记
  numbers=none,         % 不显示行号(Markdown默认无行号)
  breaklines=true,      % 自动换行(避免溢出)
  columns=fullflexible  ] 

t_length = 20;
N = 64;

T = t_length/N;
t = (0:N-1)*T - t_length/2;

x = (t + 2).*(t>=-2&t<-1)+1.*(t>=-1&t<=1)+(2 - t).*(t>1&t<=2);

w_length = 2*pi/T;
W = w_length/N;
w = (0:N-1)*W;
disp('Computing FFT...');
tic
X = T*fft(x,N);
X = fftshift(X);
toc
time_of_fft = toc - tic;

x_num = x;
w_num = linspace(-w_length/2, w_length/2, N);
disp('Computing Numerical FT...');
tic
W_num = x_num .* exp(-1j * w_num' * t);
toc
toc_of_num = toc - tic;
figure;
subplot(2,1,1);
stem(t,x);
title('Signal x(t)');

subplot(2,1,2);
stem(w-w_length/2,abs(X));
hold on;
plot(w_num, abs(sum(W_num,2))*T, 'r');
title('Magnitude Spectrum of x(t)');
hold off;

saveas(gcf, './p2.png');
\end{lstlisting}

运行结果如图\ref{fig:7-task2_result}所示:
\begin{figure}[H]
    \centering
    \includegraphics[width=0.8\textwidth]{./matlab/p2.png}
    \caption{任务二运行结果}
    \label{fig:7-task2_result}
\end{figure}


\subsubsection{任务三}
任务三代码如下所示:


\begin{lstlisting}[language=Matlab, 
                  caption={信号波形生成与绘图代码}, 
                        % 指定代码语言(此处为MATLAB)
  basicstyle=\ttfamily\small,  % 等宽字体+小字号(Markdown常用)
  backgroundcolor=\color{gray!5},  % 浅灰背景(接近多数Markdown渲染效果)
  frame=none,           % 无边框(Markdown代码块通常无框)
  keywordstyle=\color{blue},  % 关键字高亮(蓝色,可选)
  commentstyle=\color{green},  % 注释高亮(绿色,可选)
  showstringspaces=false,  % 不显示字符串中的空格标记
  numbers=none,         % 不显示行号(Markdown默认无行号)
  breaklines=true,      % 自动换行(避免溢出)
  columns=fullflexible  ] 

t_length = 10;
N = 64;

T = t_length/N;
t = (0:N-1)*T - t_length/2;

x = heaviside(t+0.5) - heaviside(t-0.5);
w_length = 2*pi/T;
W = w_length/N;
w = (0:N-1)*W;
disp('f(t) Computing FFT...');
tic
X = T*fft(x,N);
X = fftshift(X);
toc

x_num = x;
w_num = linspace(-w_length/2, w_length/2, N);
disp('Computing Numerical FT...');
tic
W_num = x_num .* exp(-1j * w_num' * t);
toc

figure;
subplot(3,2,1);
stem(t,x);
title('Signal x(t)');

subplot(3,2,2);
stem(w-w_length/2,abs(X));
hold on;
plot(w_num, abs(sum(W_num,2))*T, 'r');
title('Magnitude Spectrum of x(t)');
hold off;

x = heaviside(t/2+0.5) - heaviside(t/2-0.5);
w_length = 2*pi/T;
W = w_length/N;
w = (0:N-1)*W;
disp('f(t/2) Computing FFT...');
tic
X = T*fft(x,N);
X = fftshift(X);
toc

x_num = x;
w_num = linspace(-w_length/2, w_length/2, N);
disp('Computing Numerical FT...');
tic
W_num = x_num .* exp(-1j * w_num' * t);
toc

subplot(3,2,3);
stem(t,x);
title('Signal x(t)');

subplot(3,2,4);
stem(w-w_length/2,abs(X));
hold on;
plot(w_num, abs(sum(W_num,2))*T, 'r');
title('Magnitude Spectrum of x(t)');
hold off;


x = heaviside(2.*t+0.5) - heaviside(2.*t-0.5);
w_length = 2*pi/T;
W = w_length/N;
w = (0:N-1)*W;
disp('f(2t) Computing FFT...');
tic
X = T*fft(x,N);
X = fftshift(X);
toc

x_num = x;
w_num = linspace(-w_length/2, w_length/2, N);
disp('Computing Numerical FT...');
tic
W_num = x_num .* exp(-1j * w_num' * t);
toc

subplot(3,2,5);
stem(t,x);
title('Signal x(t)');

subplot(3,2,6);
stem(w-w_length/2,abs(X));
hold on;
plot(w_num, abs(sum(W_num,2))*T, 'r');
title('Magnitude Spectrum of x(t)');
hold off;


saveas(gcf, './p3.png');
\end{lstlisting}

运行结果如图\ref{fig:7-task3_result}所示:
\begin{figure}[H]
    \centering
    \includegraphics[width=0.8\textwidth]{./matlab/p3.png}
    \caption{任务三运行结果}
    \label{fig:7-task3_result}
\end{figure}

终端输出结果如下所示:
\begin{lstlisting}[language=Matlab, 
                  caption={任务三终端输出结果}, 
                        % 指定代码语言(此处为MATLAB)
  basicstyle=\ttfamily\small,  % 等宽字体+小字号(Markdown常用)
  backgroundcolor=\color{gray!5},  % 浅灰背景(接近多数Markdown渲染效果)
  frame=none,           % 无边框(Markdown代码块通常无框)
  keywordstyle=\color{blue},  % 关键字高亮(蓝色,可选)
  commentstyle=\color{green},  % 注释高亮(绿色,可选)
  showstringspaces=false,  % 不显示字符串中的空格标记
  numbers=none,         % 不显示行号(Markdown默认无行号)
  breaklines=true,      % 自动换行(避免溢出)
  columns=fullflexible  ]

f(t) Computing FFT...
历时 0.000569 秒。
Computing Numerical FT...
历时 0.000585 秒。
f(t/2) Computing FFT...
历时 0.000154 秒。
Computing Numerical FT...
历时 0.000244 秒。
f(2t) Computing FFT...
历时 0.000144 秒。
Computing Numerical FT...
历时 0.000244 秒。
\end{lstlisting}
\end{document}
