%! TEX root = ../report.tex

\documentclass{article}
\usepackage{ctex}       
\usepackage[left=1.5cm,right=1.5cm,head=1.5cm,bottom=1.5cm]{geometry}
\usepackage{graphicx}   
\usepackage{listings}
\usepackage{xcolor}
\usepackage{subcaption}
\usepackage{hyperref}
\usepackage{catoption}
\begin{document}
\section{常用信号的分类与观察}
\subsection{实验目的:}
\begin{itemize}
  \item 观察常用信号的波形,了解其特点及产生方法。
  \item 学会使用示波器测量常用波形的基本参数,了解信号及信号的特性。
\end{itemize}

\subsection{实验器材:}
\begin{itemize}
  \item 数字信号处理模块 S4 \\

    {\centering
    \includegraphics[width=0.5\textwidth]{moduleS4.png}
    \captionof{figure}{数字信号处理模块 S4}
    }

  \item 双踪示波器
\end{itemize}
\subsection{实验原理:}

对于一个系统特性的研究,其中重要的一个方面是研究它的输入输出关系,即在一特定的输入信号下,系
统对应的输出响应信号。因而对信号的研究是对系统研究的出发点,是对系统特性观察的基本手段与方法。
在本实验中,将对常用信号和特性进行分析、研究。

信号可以表示为一个或多个变量的函数,在这里仅对一维信号进行研究,自变量为时间。常用信号有:指
数信号、正弦信号、指数衰减正弦信号、抽样信号、钟形信号、脉冲信号等。
\subsubsection{指数信号:}
指数信号可以表示为:\[x(t)=Ae^{\alpha t}\]其中,A为幅值,$\alpha$为指数衰减系数。
\subsubsection{指数正弦信号:}
指数正弦信号的表达式为:\[x(t)=Ae^{\alpha t}sin(2\pi f t+\phi)U(t)\]其中,A为幅值,$\alpha$为指数衰减系数,f为频率,$\phi$为初相位。
\subsubsection{抽样信号:}
抽样信号表达式为:\[Sa(t)=\frac{\sin t}{t}\] 
$Sa(t)$ 是一个偶函数,在t为$\pi$的整数倍时,函数数值为0,此函数在很多应用场合有独特的应用。
\subsubsection{钟形信号:}
钟形信号表达式为:\[x(t)=Ae^{-\frac{(t-t_0)^2}{2\sigma^2}}\]其中,A为幅值,$t_0$为中心位置,$\sigma$为宽度参数。
\subsubsection{脉冲信号:}
脉冲信号表达式为:\[p(t)=U(t-\frac{\tau}{2})-U(t+\frac{\tau}{2})\]其中,$\tau$为脉冲宽度,U(t)为单位阶跃函数。
\subsubsection{方波信号:}
方波信号表达式为:\[x(t)=A \cdot square(2\pi f t)\]其中,A为幅值,f为频率。

\subsection{实验步骤:}
将拨码开关SW1置为“0000 0001”(开关拨上为1,拨下为0),打开实验箱及模块电源,按下复位键S2加载常用信号观测功能。将拨码开关S3拨为“0000 0000”,点击查看设置。

\begin{enumerate}
  \item 用示波器观测指数信号波形,并分析测量其所对应的 \(a\)、\(K\) 参数。
    \begin{enumerate}
      \item 拨码开关S3第1位拨为“1”(从左到右),其他开关拨为“0”,用示波器在TP1观察输出的指数信号,并分析测量其对应的频率 \(a\)、\(K\) 参数。
      \item 拨码开关S3第2位拨为“1”(从左到右),其他开关拨为“0”,观察指数信号波形的变化,分析原因。
    \end{enumerate}

  \item 指数正弦信号观察(正频率信号)。
    \begin{enumerate}
      \item 拨码开关S3第3位拨为“1”(从左到右),其他开关拨为“0”,用示波器在TP1观察输出的指数增长正弦信号。
      \item 拨码开关S3第4位拨为“1”(从左到右),其他开关拨为“0”,注意波形变化情况,分析原因。
    \end{enumerate}

  \item 抽样信号的观察。
    \begin{enumerate}
      \item 拨码开关S3第5位拨为“1”(从左到右),其他开关拨为“0”,用示波器在TP1处观察输出的抽样信号。
    \end{enumerate}

  \item 钟形信号的观察。
    \begin{enumerate}
      \item 拨码开关S3第6位拨为“1”(从左到右),其他开关拨为“0”,用示波器在TP1观察输出的钟形信号,观测波形。
    \end{enumerate}
\end{enumerate}

\textbf{注意:}该实验不要将拨码开关S3的第7位和第8位拨为“1”。

\subsection{实验实验结果及结论:}
\subsubsection{通过示波器观察常见信号:}

以下是通过示波器观察到的常见的信号。
% 六个子图(2×3布局),含独立标题(需导言区加载graphicx和subcaption包)
\begin{figure}[htbp]
  \centering
  % 第1行子图
  \begin{subfigure}[b]{0.31\textwidth}
    \centering
    \includegraphics[width=\linewidth]{exponential1.png}
    \caption{指数增加信号}
  \end{subfigure}
  \hfill
  \begin{subfigure}[b]{0.31\textwidth}
    \centering
    \includegraphics[width=\linewidth]{exponential2.png}
    \caption{指数衰减信号}
  \end{subfigure}
  \hfill
  \begin{subfigure}[b]{0.31\textwidth}
    \centering
    \includegraphics[width=\linewidth]{exponential_sine1.png}
    \caption{指数增长正弦信号}
  \end{subfigure}
  
  % 第2行子图
  \vspace{0.5cm} % 上下行间距
  \begin{subfigure}[b]{0.31\textwidth}
    \centering
    \includegraphics[width=\linewidth]{exponential_sine2.png}
    \caption{指数衰减正弦信号}
  \end{subfigure}
  \hfill
  \begin{subfigure}[b]{0.31\textwidth}
    \centering
    \includegraphics[width=\linewidth]{sampling_signal.png}
    \caption{抽样信号}
  \end{subfigure}
  \hfill
  \begin{subfigure}[b]{0.31\textwidth}
    \centering
    \includegraphics[width=\linewidth]{bell_signal.png}
    \caption{钟形信号}
  \end{subfigure}
  
  \caption{常用信号观测波形图} % 总标题(可选,可删除)
\end{figure}
\subsubsection{指数信号数据记录}

以下是原始数据记录,如表\ref{tab:fit_data}所示。

\begin{table}[htbp]
  \centering
  \caption{指数信号采集数据表}
  \begin{tabular}{l p{14cm}}  % 左列变量名,右列数据(自动换行)
    \hline
    \textbf{$x_1/ms$} & 0.752, 2.304, 2.240, 2.148, 2.004, 1.892, 1.772, 1.628, 1.500, 1.392, 1.276, 1.180, 1.032 \\
    \textbf{$y_1/V$} & 0.160, 5.040, 4.240, 3.360, 2.360, 1.800, 1.320, 0.920, 0.680, 0.520, 0.400, 0.320, 0.240 \\
    \textbf{$x_2/ms$} & 1.560, 1.652, 1.728, 1.804, 1.900, 1.984, 2.068, 2.176, 2.292, 2.412, 2.500, 2.636, 2.832 \\
    \textbf{$y_2/V$} & 5.840, 4.120, 3.400, 2.840, 2.200, 1.800, 1.440, 1.080, 0.840, 0.600, 0.520, 0.400, 0.280 \\
    \hline
  \end{tabular}
  \label{tab:fit_data}
\end{table}

\subsection{实验程序及运行结果:}
\subsubsection{指数信号拟合}
通过MATLAB曲线拟合器对采集指数信号数据进行指数拟合,得到拟合曲线,如图\ref{fig:exp_fit}所示。

\begin{figure}
  \centering
  \begin{subfigure}[b]{0.45\linewidth}
    \centering
    \includegraphics[width=0.8\linewidth]{exp_increase_fit.png}
    \caption{指数增加信号拟合曲线}
  \end{subfigure}
  \begin{subfigure}[b]{0.45\linewidth}
    \centering
    \includegraphics[width=0.8\linewidth]{exp_decrease_fit.png}
    \caption{指数衰减信号拟合曲线}
  \end{subfigure}
  \caption{指数信号拟合结果}
  \label{fig:exp_fit}
\end{figure}

拟合参数参见表\ref{tab:exp_curves_fit}。

% 需在导言区加载subcaption包:\usepackage{subcaption}
\begin{table}[htbp]
  \centering
  \caption{指数增长与衰减曲线拟合结果对比}
  % 子表1:指数增长曲线(居左)
  \begin{subtable}[t]{0.48\textwidth}
    \centering
    \caption{指数增长曲线}
    \begin{tabular}{c c c c}
      \hline
      \multicolumn{4}{c}{\textbf{指数曲线拟合(exp1)}} \\
      \multicolumn{4}{c}{$f(x) = a \cdot \exp(b \cdot x)$} \\
      \hline
      \multicolumn{4}{c}{\textbf{系数和 95\% 置信边界}} \\
      \hline
      & 值 & 下限 & 上限 \\
      \hline
      $a$ & 0.0161 & 0.0147 & 0.0175 \\
      $b$ & 2.4915 & 2.4514 & 2.5315 \\
      \hline
      \multicolumn{4}{c}{\textbf{拟合优度}} \\
      \hline
      &  & & 值  \\
      \hline
      SSE &  & & 0.0077  \\
      $R^2$  & & & 0.9998  \\
      DFE  & & & 11  \\
      调整 $R^2$ & & & 0.9997  \\
      RMSE & & & 0.0265  \\
      \hline
    \end{tabular}
    \label{tab:sub_exp_growth}
  \end{subtable}
  \hfill % 两子表间距
  % 子表2:指数衰减曲线(居右)
  \begin{subtable}[t]{0.48\textwidth}
    \centering
    \caption{指数衰减曲线}
    \begin{tabular}{c c c c}
      \hline
      \multicolumn{4}{c}{\textbf{指数曲线拟合(exp1)}} \\
      \multicolumn{4}{c}{$f(x) = a \cdot \exp(b \cdot x)$} \\
      \hline
      \multicolumn{4}{c}{\textbf{系数和 95\% 置信边界}} \\
      \hline
      & 值 & 下限 & 上限 \\
      \hline
      $a$ & 387.63 & 266 & 509.27 \\
      $b$ & -2.7171 & -2.901 & -2.5333 \\
      \hline
      \multicolumn{4}{c}{\textbf{拟合优度}} \\
      \hline
      &  & & 值  \\
      \hline
      SSE &  & & 0.1782  \\
      $R^2$   & & & 0.9948  \\
      DFE &  & & 11  \\
      调整 $R^2$ &  & & 0.9943  \\
      RMSE &  & & 0.1273  \\
      \hline
    \end{tabular}
    \label{tab:sub_exp_decay}
  \end{subtable}
  \label{tab:exp_curves_fit}
\end{table}


\newpage
\subsubsection{常见信号的绘制}
  使用 MATLAB 绘制常见信号的代码如下:

\begin{lstlisting}[language=Matlab, 
                  caption={信号波形生成与绘图代码}, 
                        % 指定代码语言(此处为MATLAB)
  basicstyle=\ttfamily\small,  % 等宽字体+小字号(Markdown常用)
  backgroundcolor=\color{gray!5},  % 浅灰背景(接近多数Markdown渲染效果)
  frame=none,           % 无边框(Markdown代码块通常无框)
  keywordstyle=\color{blue},  % 关键字高亮(蓝色,可选)
  commentstyle=\color{green},  % 注释高亮(绿色,可选)
  showstringspaces=false,  % 不显示字符串中的空格标记
  numbers=none,         % 不显示行号(Markdown默认无行号)
  breaklines=true,      % 自动换行(避免溢出)
  columns=fullflexible  ]        

    x = linspace(-8,8,200); % 生成时间序列
    % 信号处理
    y1 = exp(0.3 .* x); % 指数增长信号
    y2 = exp(-0.3 .* x); % 指数衰减信号
    y3 = sin(x).*exp(0.15.*x); % 正弦指数增长信号
    y4 = sin(x).*exp(-0.15*x); % 正弦指数衰减信号
    y5 = sinc(1.5.*x./pi); % 采样信号
    y6 = exp(-(0.3.*x).^2); % 钟形信号
    % 绘图
    figure;
    subplot(3,2,1);
    plot(x,y1);
    title('指数增长信号'); 
    subplot(3,2,2);
    plot(x,y2);
    title('指数衰减信号'); 
    subplot(3,2,3);
    plot(x, y3);
    title('正弦指数增长信号');
    subplot(3,2,4);
    plot(x, y4);
    title('正弦指数衰减信号'); 
    subplot(3,2,5);
    plot(x, y5);
    title('采样信号'); 
    subplot(3,2,6);
    plot(x, y6);
    title('钟形信号'); 
    % 保存图像
    handle = gcf; % 当前图形句柄
    saveas(handle,'class1_1.jpg')

    
\end{lstlisting}

导出图像如图\ref{fig:common_signals}所示。
\begin{figure}[htbp]
  \centering
  \includegraphics[width=0.8\textwidth]{sixSignal.jpg}
  \caption{常见信号波形图}
  \label{fig:common_signals}
\end{figure}


\end{document}

