
% ! TEX root = ../report.tex
%

\documentclass[a4paper]{article}
\usepackage{ctex}       
% 修正:geometry参数head→top(head是页眉高度,此处应为上边距)
\usepackage[left=1.5cm,right=1.5cm,top=1.5cm,bottom=1.5cm]{geometry}
\usepackage{graphicx}   
\usepackage{listings}
\usepackage{xcolor}
\usepackage{booktabs}
\usepackage{subcaption}
\usepackage{caption}
\usepackage{multirow} 
% 移除冲突包:subcaption与subfigure冲突,且subfigure是旧包,此处用subcaption足够
% \usepackage{subfigure}
% 修正:hyperref放在最后加载,避免与其他包冲突
\usepackage{hyperref}
\usepackage{float}

\begin{document}
\section{连续时间系统的模拟}
\subsection{实验目的}
  

\begin{enumerate}
    \item 了解基本运算器——比例放大器、加法器和积分器的电路结构和运算功能。
    \item 掌握用基本运算单元模拟连续时间一阶系统原理与测试方法。
\end{enumerate}

\subsection{实验器材}

\begin{enumerate}
    \item 双踪示波器 \hfill 1台
    \item 数字万用表 \hfill 1块
    \item 电压表及直流信号源模块\textcircled{S1} \hfill 1块
    \item 信号源及频率计模块\textcircled{S2} \hfill 1块
    \item 基本运算单元与连续系统模块\textcircled{S9} \hfill 1块
\end{enumerate}

\subsection{实验原理}

\subsubsection{1、线性系统的模拟}
利用运算放大器组成基本的运算单元(放大器、加法器、积分器等)来模拟实际系统传输特性。

\subsubsection{2、三种基本运算电路}
\paragraph{a、比例放大器}
比例放大器的输入输出关系为:
\[
u_0 = -\frac{R_2}{R_1} \cdot u_1
\]
\begin{figure}[htbp]
    \centering
    \includegraphics{./fig3-1.png} % 替换为实际图片文件名(如fig3-1.png)
    \caption{比例放大电路连线示意图}
    \label{fig:amp-circuit} % 定义标签,命名可自定义
\end{figure}
比例放大器的电路连线示意如图\ref{fig:amp-circuit}所示。

\paragraph{b、加法器}
加法器的输入输出关系为:
\[
u_0 = -\frac{R_2}{R_1}(u_1 + u_2)
\]
当 \( R_1 = R_2 \) 时,简化为:
\[
u_0 = -(u_1 + u_2)
\]
\begin{figure}[htbp]
    \centering
    \includegraphics{fig3-2.png} % 替换为实际图片文件名(如fig3-2.png)
    \caption{加法器电路连线示意图}
    \label{fig:adder-circuit}
\end{figure}
加法器的电路连线示意如图\ref{fig:adder-circuit}所示。

\paragraph{c、积分器}
积分器的输入输出关系为:
\[
u_0 = -\frac{1}{RC} \int u_1 dt
\]
\begin{figure}[htbp]
    \centering
    \includegraphics{fig3-3.png} % 替换为实际图片文件名(如fig3-3.png)
    \caption{积分器电路连接示意图}
    \label{fig:integrator-circuit}
\end{figure}
积分器的电路连接示意如图\ref{fig:integrator-circuit}所示。

\subsubsection{3、一阶系统的模拟}
图\ref{fig:first-order-system}(\ref{subfig:first-order-rc})为一阶RC电路,可用微分方程描述:
\[
\frac{dy(t)}{dt} + \frac{1}{RC} y(t) = \frac{1}{RC} x(t)
\]
其模拟框图如图\ref{fig:first-order-system}(\ref{subfig:first-order-block1})和图\ref{fig:first-order-system}(\ref{subfig:first-order-block2})所示,二者数学关系等效,对应的积分关系分别为:
\[
\int \left[ x(t)\frac{1}{RC} - y(t)\frac{1}{RC} \right] dt = y(t)
\]
\[
\int \left[ -x(t) + y(t) \right] \left( -\frac{1}{RC} \right) dt = y(t)
\]
一阶系统模拟实验电路如图\ref{fig:first-order-system}(\ref{subfig:first-order-exp-circuit})所示。

\begin{figure}[htbp]
    \centering
    \begin{subfigure}[b]{0.2\textwidth}
        \centering
        \includegraphics[width=\linewidth]{fig3-4a.png} % 替换为实际子图文件名
        \caption{一阶RC电路}
        \label{subfig:first-order-rc}
    \end{subfigure}
    \hfill
    \begin{subfigure}[b]{0.2\textwidth}
        \centering
        \includegraphics[width=\linewidth]{fig3-4b.png} % 替换为实际子图文件名
        \caption{模拟框图1}
        \label{subfig:first-order-block1}
    \end{subfigure}
    \hfill
    \begin{subfigure}[b]{0.2\textwidth}
        \centering
        \includegraphics[width=\linewidth]{fig3-4c.png} % 替换为实际子图文件名
        \caption{模拟框图2}
        \label{subfig:first-order-block2}
    \end{subfigure}
    \hfill
    \begin{subfigure}[b]{0.2\textwidth}
        \centering
        \includegraphics[width=\linewidth]{fig3-4d.png} % 替换为实际子图文件名
        \caption{模拟实验电路}
        \label{subfig:first-order-exp-circuit}
    \end{subfigure}
    \caption{一阶系统的电路与模拟框图}
    \label{fig:first-order-system}
\end{figure}

\subsection{实验步骤}

\subsubsection{任务一 基本运算器——加法器的观测}
\begin{figure}[htbp]
    \centering
    \includegraphics[width=0.4\textwidth]{./fig3-6.png} % 替换为图3-6的实际文件名
    \caption{加法器实验电路图}
    \label{fig:adder-exp-circuit}
\end{figure}

①模块关电,按图\ref{fig:adder-exp-circuit}所示连接实验电路。

②将模块\textcircled{S1}的直流输出1(P1)和直流输出2(P2)分别接至加法器的$u_1$和$u_2$。模块开电,适当调节模块\textcircled{S1}中的W1、W2,并记录P1和P2的电压值。

③用万用表或示波器测量输出$u_0$端电压,验证反相加法器“输出为两路输入之和取反”的特性,完成下表:
\begin{center}
\begin{tabular}{|c|c|c|}
\hline
输入一 & 输入二 & 输出 \\
\hline
电压(V) & 电压(V) & 电压(V) \\
\hline
 &  &  \\
\hline
 &  &  \\
\hline
 &  &  \\
\hline
\end{tabular}
\end{center}

④将输入信号改为“幅度2V、频率500Hz的方波”,观察输入/输出波形并补充表格。


\subsubsection{任务二 基本运算器——比例放大器的观测}
\begin{figure}[htbp]
    \centering
    \includegraphics[width=0.4\textwidth]{./fig3-7.png} % 替换为图3-7的实际文件名
    \caption{比例放大器实验电路图}
    \label{fig:amp-exp-circuit}
\end{figure}

①模块关电,连接如图\ref{fig:amp-exp-circuit}所示实验电路;$R_1、R_2$可选2组不同电阻值以改变放大比例。

②模块开电,将“幅度1V、频率1KHz的正弦波”送入输入端,用示波器观测输入/输出波形,完成下表:
\begin{center}
\begin{tabular}{|c|c|c|c|c|c|}
\hline
\multicolumn{2}{|c|}{电阻} & \multicolumn{2}{c|}{输入} & \multicolumn{2}{c|}{输出} \\
\hline
\multicolumn{2}{|c|}{} & 电压(V) & 波形 & 电压(V) & 波形 \\
\hline
\multirow{2}{*}{①} & $R1=1\mathrm{K}\Omega$ &  &  &  &  \\
%\cline{2-6}
 & $R2=5.1\mathrm{K}\Omega$ &  &  &  &  \\
\hline
\multirow{2}{*}{②} & $R1=$ &  &  &  &  \\
%\cline{2-6}
 & $R2=$ &  &  &  &  \\
\hline
\end{tabular}
\end{center}


\subsubsection{任务三 基本运算器——积分器的观测}
\begin{figure}[htbp]
    \centering
    \includegraphics[width=0.4\textwidth]{./fig3-8.png} % 替换为图3-8的实际文件名
    \caption{积分器实验电路图}
    \label{fig:integrator-exp-circuit}
\end{figure}

①模块关电,连接如图\ref{fig:integrator-exp-circuit}所示实验电路($20\mathrm{K}\Omega$电阻可由两个$10\mathrm{K}\Omega$电阻串联代替)。

②模块开电,用模块\textcircled{S2}产生$f=1\mathrm{KHz}$的方波送入输入端,用示波器观测输入/输出波形并记录数据。


\subsubsection{任务四 一阶RC电路的模拟}
图\ref{fig:first-order-system}(\ref{subfig:first-order-rc})为一阶RC电路,模块关电后,按图\ref{fig:first-order-system}(\ref{subfig:first-order-exp-circuit})连接其一阶模拟电路($0.022\mathrm{\mu F}$电容可由两个$0.01\mathrm{\mu F}$电容并联代替)。

\begin{figure}[htbp]
    \centering
    \includegraphics[width=0.6\textwidth]{fig3-9.png} % 替换为图3-9的实际文件名
    \caption{模块实验电路图}
    \label{fig:rc-module-circuit}
\end{figure}

模块开电,将“幅度2V、频率1KHz的方波”送入一阶模拟电路输入端,用示波器观测输出电压波形,验证模拟效果。

\subsection{实验结果}
\subsubsection{任务一:基本运算器——加法器的观测}
结果如下表\ref{lab:adder-circuit-result}
\begin{table}[H]
\caption{加法器实验结果表}
\label{lab:adder-circuit-result}
\centering
\begin{tabular}{|c|c|c|}
\hline
输入一 & 输入二 & 输出 \\
\hline
电压(V) & 电压(V) & 电压(V) \\
\hline
 1& 1 & -2 \\
\hline
 1& -0.5 & -0.5 \\
\hline
 1& 0.5 & -1.5 \\
\hline
\end{tabular}
\end{table}


\subsubsection{任务二:基本运算器——比例放大器的观测}
结果如下表\ref{lab:amp-circuit-result}
\begin{table}[H]
\caption{比例放大器实验结果表}
\label{lab:amp-circuit-result}
\centering
\begin{tabular}{|c|c|c|c|c|c|}
\hline
\multicolumn{2}{|c|}{电阻} & \multicolumn{2}{c|}{输入} & \multicolumn{2}{c|}{输出} \\
\hline
\multicolumn{2}{|c|}{} & 电压(V) & 波形 & 电压(V) & 波形 \\
\hline
\multirow{2}{*}{①} & $R1=1\mathrm{K}\Omega$ & 1 V  & 正弦波 & 2 V & 正弦波(反相) \\
%\cline{2-6}
 & $R2=5.1\mathrm{K}\Omega$ &  &  &  &  \\
\hline
\multirow{2}{*}{②} & $R1=10\mathrm{K}\Omega$ & 1.92 V & 正弦波 & 9.6 V & 正弦波(反相) \\
%\cline{2-6}
 & $R2=5.1\mathrm{K}\Omega$ &  &  &  &  \\
\hline
\end{tabular}
\end{table}



波形如图\ref{fig:amp-waveform}所示。

\begin{figure}[H]
    \centering
    \begin{subfigure}[b]{0.4\textwidth}
        \centering
        \includegraphics[width=\linewidth]{fig-result-2-1.png} % 替换为实际子图文件名
        \caption{R1=1KΩ,R2=5.1KΩ}
    \end{subfigure}
    \begin{subfigure}[b]{0.4\textwidth}
        \centering
        \includegraphics[width=\linewidth]{fig-result-2-2.png} % 替换为实际子图文件名
        \caption{R1=10KΩ,R2=5.1KΩ}
    \end{subfigure}
    \caption{比例放大器输入输出波形}
    \label{fig:amp-waveform}
\end{figure}

\subsubsection{任务三:基本运算器——积分器的观测}

积分电路波形如下图\ref{fig:integrator-waveform}所示。

\begin{figure}[H]
    \centering
    \includegraphics[width=0.4\textwidth]{fig-result-3.png} % 替换为实际图片文件名
    \caption{积分器输入输出波形}
    \label{fig:integrator-waveform}
\end{figure}

\subsubsection{任务四:一阶RC电路的模拟}

电路波形如下图\ref{fig:first-order-waveform}所示。

\begin{figure}[H]
    \centering
    \includegraphics[width=0.4\textwidth]{fig-result-4.png} % 替换为实际图片文件名
    \caption{一阶RC电路模拟输入输出波形}
    \label{fig:first-order-waveform}
\end{figure}

\subsection{实验程序及运行结果}
\subsubsection{问题一运行结果}

\begin{lstlisting}[language=Matlab, 
                  caption={信号波形生成与绘图代码}, 
                        % 指定代码语言(此处为MATLAB)
  basicstyle=\ttfamily\small,  % 等宽字体+小字号(Markdown常用)
  backgroundcolor=\color{gray!5},  % 浅灰背景(接近多数Markdown渲染效果)
  frame=none,           % 无边框(Markdown代码块通常无框)
  keywordstyle=\color{blue},  % 关键字高亮(蓝色,可选)
  commentstyle=\color{green},  % 注释高亮(绿色,可选)
  showstringspaces=false,  % 不显示字符串中的空格标记
  numbers=none,         % 不显示行号(Markdown默认无行号)
  breaklines=true,      % 自动换行(避免溢出)
  columns=fullflexible  ]  
sys = tf([0,3,2],[1,5,6]);
t = -3:0.01:10;
yd = impulse(sys,t);
yu = step(sys,t);

figure;
subplot(3,1,1);
plot(t,yd);
title('Impulse Response');

subplot(3,1,2);
plot(t,yu);
title('Step Response');


f= exp(-2.*t).*heaviside(t);
yf = lsim(sys,f,t);

subplot(3,1,3);
plot(t,yf);
title('Response to f(t) = e^{-2t}u(t)');


saveas(gcf,'./p1.png');

\end{lstlisting}

问题一运行结果如下图\ref{fig:problem1-result}所示。
\begin{figure}[H]
    \centering
    \includegraphics[width=0.6\textwidth]{./matlab/p1.png} % 替换为实际图片文件名
    \caption{问题一运行结果图}
    \label{fig:problem1-result}
\end{figure}

\subsubsection{问题二运行结果}


\begin{lstlisting}[language=Matlab, 
                  caption={信号波形生成与绘图代码}, 
                        % 指定代码语言(此处为MATLAB)
  basicstyle=\ttfamily\small,  % 等宽字体+小字号(Markdown常用)
  backgroundcolor=\color{gray!5},  % 浅灰背景(接近多数Markdown渲染效果)
  frame=none,           % 无边框(Markdown代码块通常无框)
  keywordstyle=\color{blue},  % 关键字高亮(蓝色,可选)
  commentstyle=\color{green},  % 注释高亮(绿色,可选)
  showstringspaces=false,  % 不显示字符串中的空格标记
  numbers=none,         % 不显示行号(Markdown默认无行号)
  breaklines=true,      % 自动换行(避免溢出)
  columns=fullflexible  ] 

sys = tf([0,0,4],[0,1,12]);
t = -3:0.01:10;

yd = impulse(sys,t);
yu = step(sys,t);

f = 12 .* heaviside(t);

yf = lsim(sys,f,t);

figure;
subplot(3,1,1);
plot(t,yd);
title('Impulse Response');

subplot(3,1,2);
plot(t,yu);
title('Step Response');

subplot(3,1,3);
plot(t,yf);
title('Response to f(t) = 12u(t)');

saveas(gcf,'./p2.png');
\end{lstlisting}

问题二运行结果如下图\ref{fig:problem2-result}所示。
\begin{figure}[H]
  \centering
  \includegraphics[width=0.6\textwidth]{./matlab/p2.png} % 替换为实际图片文件名
  \caption{问题二运行结果图}
  \label{fig:problem2-result}
\end{figure}

\subsubsection{问题三运行结果}


\begin{lstlisting}[language=Matlab, 
                  caption={信号波形生成与绘图代码}, 
                        % 指定代码语言(此处为MATLAB)
  basicstyle=\ttfamily\small,  % 等宽字体+小字号(Markdown常用)
  backgroundcolor=\color{gray!5},  % 浅灰背景(接近多数Markdown渲染效果)
  frame=none,           % 无边框(Markdown代码块通常无框)
  keywordstyle=\color{blue},  % 关键字高亮(蓝色,可选)
  commentstyle=\color{green},  % 注释高亮(绿色,可选)
  showstringspaces=false,  % 不显示字符串中的空格标记
  numbers=none,         % 不显示行号(Markdown默认无行号)
  breaklines=true,      % 自动换行(避免溢出)
  columns=fullflexible  ] 
sys = tf([0,1,0],[1,1,100]);

t = -3:0.01:10;

yd = impulse(sys,t);
yu = step(sys,t);

figure;
subplot(2,1,1);
plot(t,yd);
title('Impulse Response');

subplot(2,1,2);
plot(t,yu);
title('Step Response');

saveas(gcf,'./p3.png');
\end{lstlisting}

问题三运行结果如下图\ref{fig:problem3-result}所示。
\begin{figure}[H]
  \centering
  \includegraphics[width=0.6\textwidth]{./matlab/p3.png} % 替换为实际图片文件名
  \caption{问题三运行结果图}
  \label{fig:problem3-result}
\end{figure}

\end{document}
