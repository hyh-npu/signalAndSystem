%  ! TEX root = ../report.tex 
\documentclass[../report.tex]{subfile}  % 仅这一行关联主文件,无其他导言命令
\begin{document}
\section{实验4:滤波器特性测量}
\subsection{实验目的}
1、熟悉滤波器构成及其特性。
2、学会测量滤波器幅频特性的方法。

\subsection{实验器材}
\begin{enumerate}
    \item 双踪示波器:1台
    \item 信号源及频率计模块:1块
    \item 抽样定理及滤波器模块:1块
\end{enumerate}

\subsection{实验原理}
滤波器是一种能使有用频率信号通过而抑制(或大为衰减)无用频率信号的电子装置,工程上常用于信号处理、数据传送和抑制干扰等,本实验主要讨论模拟滤波器。

以往滤波电路主要采用无源元件R、L和C组成;60年代以来,集成运放迅速发展,由其与R、C组成的有源滤波电路,具有不用电感、体积小、重量轻等优点。此外,集成运放的开环电压增益和输入阻抗均很高,输出阻抗又低,构成的有源滤波电路还具有一定的电压放大和缓冲作用,但集成运放的带宽有限,导致有源滤波电路的工作频率难以做得很高。

\subsubsection{初步定义}
滤波电路的一般结构如图\ref{fig:4-1}所示,图中$v_I(t)$表示输入信号,$v_O(t)$为输出信号。假设滤波器是线性时不变网络,则在复频域内有:
\[A(s) = \frac{V_O(s)}{V_I(s)}\]
式中$A(s)$是滤波电路的电压传递函数(一般为复数)。对于实际频率($s = j\omega$),则有:
\[A(j\omega) = \left|A(j\omega)\right| e^{j\varphi(\omega)}\]

其中,$\left|A(j\omega)\right|$为传递函数的模,$\varphi(\omega)$为其相位角。

% 改用figure浮动体(保留浮动体,避免subcaption报错)
\begin{figure}[htbp]
    \centering
    \includegraphics[width=0.6\textwidth]{fig4-1.png}
    \caption{滤波器电路的一般结构}
    \label{fig:4-1}
\end{figure}

二阶RC滤波器的传输函数如下表所示:
\begin{center}
    \begin{tabular}{|c|c|}
        \hline
        类型 & 传输函数 \\
        \hline
        低通 & $A(s) = \frac{A_0 \omega_0}{s^2 + \frac{\omega_0}{Q} s + \omega_0^2}$ \\
        \hline
        高通 & $A(s) = \frac{A_0 s^2}{s^2 + \frac{\omega_0}{Q} s + \omega_0^2}$ \\
        \hline
        带通 & $A(s) = \frac{A_0 \frac{\omega_0}{Q} s}{s^2 + \frac{\omega_0}{Q} s + \omega_0^2}$ \\
        \hline
        带阻 & $A(s) = \frac{A_V (s^2 + \omega_0^2)}{s^2 + \frac{\omega_0}{Q} s + \omega_0^2}$ \\
        \hline
    \end{tabular}
    \\[6pt]
    \textbf{备注:} 电压增益 $A_V$;$\omega_c$——低、高通滤波器的截止角频率;$\omega_0$——带通、带阻滤波器的中心角频率;$Q$——品质因数,$Q \approx \omega_0 / BW$ 或 $f_0 / BW$(当 $BW \ll \omega_0$ 时);$BW$——带通、带阻滤波器的带宽。
\end{center}

此外,滤波电路中需关注时延$\tau(\omega)$,其定义为:
\[ \tau(\omega) = -\frac{d\varphi(\omega)}{d\omega} \quad (4-2) \]
通常用幅频响应表征滤波电路特性,欲使信号通过滤波器的失真很小,需考虑相位和时延响应。当相位响应$\varphi(\omega)$呈线性变化(即时延响应$\tau(\omega)$为常数)时,输出信号可避免失真。

\subsubsection{滤波电路的分类}
对于幅频响应,能通过的信号频率范围定义为通带,受阻或衰减的信号频率范围称为阻带,通带和阻带的界限频率叫做截止频率$f_c$。

理想滤波电路在通带内应具有零衰减的幅频响应和线性的相位响应,而在阻带内应具有无限大的幅度衰减($|A(j\omega)| = 0$)。根据通带和阻带的相互位置,滤波电路通常分为以下四类,其幅频响应如图\ref{fig:4-2}所示:

% 子图改用minipage+figure浮动体,避免subcaption报错
\begin{figure}[htbp]
    \centering
    \begin{minipage}{0.45\textwidth}
        \centering
        \includegraphics[width=\textwidth]{fig4-2a.png}
        \footnotesize (a) 低通滤波电路(LPF)
    \end{minipage}
    \begin{minipage}{0.45\textwidth}
        \centering
        \includegraphics[width=\textwidth]{fig4-2b.png}
        \footnotesize (b) 高通滤波电路(HPF)
    \end{minipage}
    \\[6pt]
    \begin{minipage}{0.45\textwidth}
        \centering
        \includegraphics[width=\textwidth]{fig4-2c.png}
        \footnotesize (c) 带通滤波电路(BPF)
    \end{minipage}
    \begin{minipage}{0.45\textwidth}
        \centering
        \includegraphics[width=\textwidth]{fig4-2d.png}
        \footnotesize (d) 带阻滤波电路(BEF)
    \end{minipage}
    \caption{各种滤波电路的幅频响应}
    \label{fig:4-2}
\end{figure}

\begin{itemize}
    \item 低通滤波电路:通过从零到截止角频率$\omega_H$的低频信号,对大于$\omega_H$的频率完全衰减,带宽$BW = \omega_H$。
    \item 高通滤波电路:阻带为$0 < \omega < \omega_L$,通带为$\omega > \omega_L$,理论带宽$BW = \infty$,实际受有源器件带宽限制。
    \item 带通滤波电路:有两个阻带($0 < \omega < \omega_L$和$\omega > \omega_H$),带宽$BW = \omega_H - \omega_L$,$\omega_0$为中心角频率。
    \item 带阻滤波电路:有两个通带($0 < \omega < \omega_L$和$\omega > \omega_H$)和一个阻带($\omega_L < \omega < \omega_H$),功能是衰减$\omega_L$到$\omega_H$间的信号,$\omega_0$为中心角频率,实际通带$\omega > \omega_H$受有源器件限制。
\end{itemize}

\subsubsection{滤波器特性测量方法}
研究滤波器特性需考察幅频特性曲线和相频特性曲线,常用测量方法有两种:

\begin{enumerate}
    \item 点频法(逐点测量法):保持输入电压不变,人工逐点改变信号发生器的频率,记录各点对应的输出电压幅度,在直角坐标平面描绘幅度-频率曲线。
          \begin{itemize}
              \item 优点:测量准确度高,不需要特殊仪器。
              \item 缺点:操作繁琐,容易漏掉某些细节,不能反应被测网络动态特性。
          \end{itemize}
    \item 扫频法:利用扫频信号发生器产生电压恒定、频率随时间线性变化的信号,实现输入信号频率的自动调节。
          \begin{itemize}
              \item 优点:操作简单,速度快,频率连续变化能覆盖更多频点,可反应网络动态特性,能观察到系统脉冲干扰等冲激变化。
              \item 缺点:测量准确度比点频法低。
          \end{itemize}
\end{enumerate}

正弦扫频信号(Sine Sweep)可作为系统激励和测取系统传递函数的有效方法,广泛应用于科研及生产中,在滤波器设计中常用其测量系统频率特性。

\subsubsection{模块参数与扫频设置}
\paragraph{扫频设置方法}
信号源及频率计模块的扫频设置步骤如下:
\begin{enumerate}
    \item 将模块中扫频开关S3拨至“ON”,按下“扫频设置”按钮S5,此时“下限”指示灯亮,调节“ROL1”旋钮设置扫频下限频率;
    \item 再次按下“扫频设置”按钮,“上限”指示灯亮,调节“ROL1”旋钮设置扫频上限频率;
    \item 扫频范围设置完成后,再按一下“扫频设置”按钮,“分辨率”指示灯亮,此时扫频范围上方“上限”和“下限”指示灯亮,频率计数码管右方的“MHz”“Hz”指示灯熄灭;
    \item 调节“ROL1”设置“下限频率”和“上限频率”之间的频点数(频点数越少,扫频速度越快;反之则越慢);
    \item 扫频参数设置完成后,按下“扫频设置”即可输出扫频信号。
\end{enumerate}

\paragraph{滤波器固有参数}
各滤波器的固有截止频率/中心频率参数如下:
\begin{itemize}
    \item 低通:无源20kHz,有源17kHz;
    \item 高通:无源14.5kHz,有源14.5kHz;
    \item 带通:无源$f_L=1.3\mathrm{kHz}$、$f_H=18.5\mathrm{kHz}$;有源$f_L=2.4\mathrm{kHz}$、$f_H=20.8\mathrm{kHz}$;
    \item 带阻:无源$f_L=4.1\mathrm{kHz}$、$f_H=65.2\mathrm{kHz}$;有源$f_L=6.5\mathrm{kHz}$、$f_H=38\mathrm{kHz}$。
\end{itemize}

\subsection{实验步骤}
实验中信号源的输入信号均为4V左右的正弦波,设置模块:切换波形开关S4,使“SIN”指示灯亮,调节“模拟输出幅度调节”旋钮,使信号幅度为4V。

\subsubsection{任务一:测量低通滤波器的频响特性}
\paragraph{逐点测量法}
\begin{enumerate}
    \item 模块关电,连接模块2中模拟信号输出端P2与模块3中P1(低通无源),保持输入信号幅度为4V不变(如图\ref{fig:4-3}a);
    \item 模块开电,逐渐改变输入信号频率,用示波器观测TP2处信号波形的峰峰值;
    \item 将数据填入表\ref{tab:4-1a}中;
    \item 模块关电,连接模块2中P2与模块3中P5(低通有源)(如图\ref{fig:4-3}b);
    \item 模块开电,逐渐改变输入信号频率,用示波器观测TP6处信号波形的峰峰值;
    \item 将数据填入表\ref{tab:4-1b}中。
\end{enumerate}

% 低通滤波器子图
\begin{figure}[htbp]
    \centering
    \begin{minipage}{0.45\textwidth}
        \centering
        \includegraphics[width=\textwidth]{fig4-3a.png}
        \footnotesize (a) 无源低通滤波器
        \label{fig:4-3a}
    \end{minipage}
    \begin{minipage}{0.45\textwidth}
        \centering
        \includegraphics[width=\textwidth]{fig4-3b.png}
        \footnotesize (b) 有源低通滤波器
        \label{fig:4-3b}
    \end{minipage}
    \caption{低通滤波器连接示意图}
    \label{fig:4-3}
\end{figure}

% 表格改用table浮动体,确保引用正确
\begin{table}[htbp]
    \centering
    \caption{低通无源滤波器逐点测量法}
    \label{tab:4-1a}
    \begin{tabular}{|c|c|c|c|c|c|c|c|c|c|c|c|}
        \hline
        $V_I(\mathrm{V})$ & 4 & 4 & 4 & 4 & 4 & 4 & 4 & 4 & 4 & 4 & 截止频率 \\
        \hline
        $f(\mathrm{Hz})$ &  &  &  &  &  &  &  &  &  &  &  \\
        \hline
        $V_O(\mathrm{V})$ &  &  &  &  &  &  &  &  &  &  &  \\
        \hline
    \end{tabular}
\end{table}

\begin{table}[htbp]
    \centering
    \caption{低通有源滤波器逐点测量法}
    \label{tab:4-1b}
    \begin{tabular}{|c|c|c|c|c|c|c|c|c|c|c|}
        \hline
        $V_I(\mathrm{V})$ & 4 & 4 & 4 & 4 & 4 & 4 & 4 & 4 & 4 & 截止频率 \\
        \hline
        $f(\mathrm{Hz})$ &  &  &  &  &  &  &  &  &  &  \\
        \hline
        $V_O(\mathrm{V})$ &  &  &  &  &  &  &  &  &  &  \\
        \hline
    \end{tabular}
\end{table}

\paragraph{扫频法测量}
\begin{enumerate}
    \item 将扫频范围设置为100Hz~25kHz;
    \item 把示波器CH1连接到信号源输出P2处(示波器调为直流测试档),此时P2输出扫频信号;
    \item 分别将扫频信号输入到无源低通、有源低通滤波器的输入端;
    \item 模块开电,对比观察输入输出信号,拍照记录波形。
\end{enumerate}

\subsubsection{任务二:测量高通滤波器的频响特性}
\paragraph{逐点测量法}
\begin{enumerate}
    \item 模块关电,保持信号源输出正弦信号幅度不变,连接模块2中P2与模块3中P3(高通无源)端口(如图\ref{fig:4-4}a);
    \item 模块开电,逐渐改变输入信号频率,用示波器观测TP4处信号波形的峰峰值;
    \item 将数据填入表\ref{tab:4-2a}中;
    \item 模块关电,连接模块2中P2与模块3中P7(高通有源)(如图\ref{fig:4-4}b);
    \item 模块开电,逐渐改变输入信号频率,用示波器观测TP8处信号波形的峰峰值;
    \item 将数据填入表\ref{tab:4-2b}中。
\end{enumerate}

% 高通滤波器子图
\begin{figure}[htbp]
    \centering
    \begin{minipage}{0.45\textwidth}
        \centering
        \includegraphics[width=\textwidth]{fig4-4a.png}
        \footnotesize (a) 无源高通滤波器
        \label{fig:4-4a}
    \end{minipage}
    \begin{minipage}{0.45\textwidth}
        \centering
        \includegraphics[width=\textwidth]{fig4-4b.png}
        \footnotesize (b) 有源高通滤波器
        \label{fig:4-4b}
    \end{minipage}
    \caption{高通滤波器连接示意图}
    \label{fig:4-4}
\end{figure}

\begin{table}[htbp]
    \centering
    \caption{高通无源滤波器逐点测量法}
    \label{tab:4-2a}
    \begin{tabular}{|c|c|c|c|c|c|c|c|c|c|c|c|}
        \hline
        $V_I(\mathrm{V})$ & 4 & 4 & 4 & 4 & 4 & 4 & 4 & 4 & 4 & 4 & 截止频率 \\
        \hline
        $f(\mathrm{Hz})$ &  &  &  &  &  &  &  &  &  &  &  \\
        \hline
        $V_O(\mathrm{V})$ &  &  &  &  &  &  &  &  &  &  &  \\
        \hline
    \end{tabular}
\end{table}

\begin{table}[htbp]
    \centering
    \caption{高通有源滤波器逐点测量法}
    \label{tab:4-2b}
    \begin{tabular}{|c|c|c|c|c|c|c|c|c|c|c|c|}
        \hline
        $V_I(\mathrm{V})$ & 4 & 4 & 4 & 4 & 4 & 4 & 4 & 4 & 4 & 4 & 截止频率 \\
        \hline
        $f(\mathrm{Hz})$ &  &  &  &  &  &  &  &  &  &  &  \\
        \hline
        $V_O(\mathrm{V})$ &  &  &  &  &  &  &  &  &  &  &  \\
        \hline
    \end{tabular}
\end{table}

\paragraph{扫频法测量}
\begin{enumerate}
    \item 将扫频范围设置为100Hz~25kHz;
    \item 把示波器连接到信号源输出P2处(示波器调为直流测试档),此时P2输出扫频信号;
    \item 分别将扫频信号输入到无源高通、有源高通滤波器的输入端;
    \item 模块开电,对比观察输入输出信号,拍照记录波形。
\end{enumerate}

\subsubsection{任务三:测量带通滤波器的频响特性}
\paragraph{逐点测量法}
\begin{enumerate}
    \item 模块关电,保持信号源输出正弦波幅度为4V不变,连接模块2中P2与模块3中P9(带通无源)(如图\ref{fig:4-5}a);
    \item 模块开电,逐渐改变输入信号频率,用示波器观测TP10处信号波形的峰峰值;
    \item 将数据填入表\ref{tab:4-3a}中;
    \item 模块关电,连接模块2中P2与模块3中P13(带通有源)(如图\ref{fig:4-5}b);
    \item 模块开电,逐渐改变输入信号频率,用示波器观测TP14处信号波形的峰峰值;
    \item 将数据填入表\ref{tab:4-3b}中。
\end{enumerate}

% 带通滤波器子图
\begin{figure}[htbp]
    \centering
    \begin{minipage}{0.45\textwidth}
        \centering
        \includegraphics[width=\textwidth]{fig4-5a.png}
        \footnotesize (a) 无源带通滤波器
        \label{fig:4-5a}
    \end{minipage}
    \begin{minipage}{0.45\textwidth}
        \centering
        \includegraphics[width=\textwidth]{fig4-5b.png}
        \footnotesize (b) 有源带通滤波器
        \label{fig:4-5b}
    \end{minipage}
    \caption{带通滤波器连接示意图}
    \label{fig:4-5}
\end{figure}

\begin{table}[htbp]
    \centering
    \caption{带通无源滤波器逐点测量法}
    \label{tab:4-3a}
    \begin{tabular}{|c|c|c|c|c|c|c|c|c|c|c|c|}
        \hline
        $V_I(\mathrm{V})$ & 4 & 4 & 4 & 4 & 4 & 4 & 4 & 4 & 4 & 4 & 截止频率($f_L, f_H$) \\
        \hline
        $f(\mathrm{Hz})$ &  &  &  &  &  &  &  &  &  &  &  \\
        \hline
        $V_O(\mathrm{V})$ &  &  &  &  &  &  &  &  &  &  &  \\
        \hline
    \end{tabular}
\end{table}

\begin{table}[htbp]
    \centering
    \caption{带通有源滤波器逐点测量法}
    \label{tab:4-3b}
    \begin{tabular}{|c|c|c|c|c|c|c|c|c|c|c|c|}
        \hline
        $V_I(\mathrm{V})$ & 4 & 4 & 4 & 4 & 4 & 4 & 4 & 4 & 4 & 4 & 截止频率($f_L, f_H$) \\
        \hline
        $f(\mathrm{Hz})$ &  &  &  &  &  &  &  &  &  &  &  \\
        \hline
        $V_O(\mathrm{V})$ &  &  &  &  &  &  &  &  &  &  &  \\
        \hline
    \end{tabular}
\end{table}

\paragraph{扫频法测量}
\begin{enumerate}
    \item 将扫频范围设置为100Hz~80kHz;
    \item 把示波器连接到信号源输出P2处(示波器调为直流测试档),此时P2输出扫频信号;
    \item 分别将扫频信号输入到无源带通、有源带通滤波器的输入端;
    \item 模块开电,对比观察输入输出信号,拍照记录波形。
\end{enumerate}

\subsubsection{任务四:测量带阻滤波器的频响特性}
\paragraph{逐点测量法}
\begin{enumerate}
    \item 模块关电,保持信号源输出正弦波幅度为4V不变,连接模块2中P2与模块3中P11(带阻无源)(如图\ref{fig:4-6}a);
    \item 模块开电,逐渐改变输入信号频率,用示波器观测TP12处信号波形的峰峰值;
    \item 将数据填入表\ref{tab:4-4a}中;
    \item 模块关电,连接模块2中P2与模块3中P15(带阻有源)(如图\ref{fig:4-6}b);
    \item 模块开电,逐渐改变输入信号频率,用示波器观测TP16处信号波形的峰峰值;
    \item 将数据填入表\ref{tab:4-4b}中。
\end{enumerate}

% 带阻滤波器子图
\begin{figure}[htbp]
    \centering
    \begin{minipage}{0.45\textwidth}
        \centering
        \includegraphics[width=\textwidth]{fig4-6a.png}
        \footnotesize (a) 无源带阻滤波器
        \label{fig:4-6a}
    \end{minipage}
    \begin{minipage}{0.45\textwidth}
        \centering
        \includegraphics[width=\textwidth]{fig4-6b.png}
        \footnotesize (b) 有源带阻滤波器
        \label{fig:4-6b}
    \end{minipage}
    \caption{带阻滤波器连接示意图}
    \label{fig:4-6}
\end{figure}

\begin{table}[htbp]
    \centering
    \caption{带阻无源滤波器逐点测量法}
    \label{tab:4-4a}
    \begin{tabular}{|c|c|c|c|c|c|c|c|c|c|c|c|}
        \hline
        $V_I(\mathrm{V})$ & 4 & 4 & 4 & 4 & 4 & 4 & 4 & 4 & 4 & 4 & 截止频率($f_L, f_H$) \\
        \hline
        $f(\mathrm{Hz})$ &  &  &  &  &  &  &  &  &  &  &  \\
        \hline
        $V_O(\mathrm{V})$ &  &  &  &  &  &  &  &  &  &  &  \\
        \hline
    \end{tabular}
\end{table}

\begin{table}[htbp]
    \centering
    \caption{带阻有源滤波器逐点测量法}
    \label{tab:4-4b}
    \begin{tabular}{|c|c|c|c|c|c|c|c|c|c|c|c|}
        \hline
        $V_I(\mathrm{V})$ & 4 & 4 & 4 & 4 & 4 & 4 & 4 & 4 & 4 & 4 & 截止频率($f_L, f_H$) \\
        \hline
        $f(\mathrm{Hz})$ &  &  &  &  &  &  &  &  &  &  &  \\
        \hline
        $V_O(\mathrm{V})$ &  &  &  &  &  &  &  &  &  &  &  \\
        \hline
    \end{tabular}
\end{table}

\paragraph{扫频法测量}
\begin{enumerate}
    \item 将扫频范围设置为100Hz~80kHz;
    \item 把示波器连接到信号源输出P2处(示波器调为直流测试档),此时P2输出扫频信号;
    \item 分别将扫频信号输入到无源带阻、有源带阻滤波器的输入端;
    \item 模块开电,对比观察输入输出信号,拍照记录波形。
\end{enumerate}

\subsection{实验结果}
\subsubsection{任务一:低通滤波器频响特性测量结果}
实验结果如下表\ref{tab:4-1a-res}和表\ref{tab:4-1b-res}所示。
\begin{table}[htbp]
    \centering
    \caption{低通无源滤波器逐点测量法}
    \label{tab:4-1a-res}
    \begin{tabular}{|c|c|c|c|c|c|c|c|c|c|}
        \hline
        $V_I(\mathrm{V})$ & 4 & 4 & 4 & 4 & 4 & 4 & 4 & 4 & 截止频率 \\
        \hline
        $f(\mathrm{Hz})$ &1k  &2k  &4k  &8k  &16k  &32k  &64k  &128k  & 19.8k  \\
        \hline
        $V_O(\mathrm{V})$ &3.920  &3.920  &3.920  &3.600  &3.040  &2.080  &1.120  &0.640  &2.828  \\
        \hline
    \end{tabular}
\end{table}

\begin{table}[htbp]
    \centering
    \caption{低通有源滤波器逐点测量法}
    \label{tab:4-1b-res}
    \begin{tabular}{|c|c|c|c|c|c|c|c|c|c|c|}
        \hline
        $V_I(\mathrm{V})$ & 4 & 4 & 4 & 4 & 4 & 4 & 4 & 4  & 截止频率 \\
        \hline
        $f(\mathrm{Hz})$ &1k  &2k  &4k  &8k  &16k  &32k  &64k  &128k  &17.1k  \\
        \hline
        $V_O(\mathrm{V})$ &4  &4  &4  &3.920  &2.960  &1.120  &0.480  &0.240  &2.828 \\
        \hline
    \end{tabular}
\end{table}

\subsubsection{任务二:高通滤波器频响特性测量结果}
实验结果如下表\ref{tab:4-2a-res}和表\ref{tab:4-2b-res}所示。
\begin{table}[htbp]
    \centering
    \caption{高通无源滤波器逐点测量法}
    \label{tab:4-2a-res}
    \begin{tabular}{|c|c|c|c|c|c|c|c|c|c|}
        \hline
        $V_I(\mathrm{V})$ & 4 & 4 & 4 & 4 & 4 & 4 & 4 & 4  & 截止频率 \\
        \hline
        $f(\mathrm{Hz})$ &1k  &2k  &4k  &8k  &16k  &32k  &64k  &128k  &16k  \\
        \hline
        $V_O(\mathrm{V})$ &0.32  &0.56  &0.96  &1.76  &2.80  &3.52 &3.84  &3.84  &2.715  \\
        \hline
    \end{tabular}
\end{table}

\begin{table}[htbp]
    \centering
    \caption{高通有源滤波器逐点测量法}
    \label{tab:4-2b-res}
    \begin{tabular}{|c|c|c|c|c|c|c|c|c|c|}
        \hline
        $V_I(\mathrm{V})$ & 4 & 4 & 4 & 4 & 4 & 4 & 4 & 4  & 截止频率 \\
        \hline
        $f(\mathrm{Hz})$ &1k  &2k  &4k  &8k  &16k  &32k  &64k  &128k  &13.7k  \\
        \hline
        $V_O(\mathrm{V})$ &0.24  &0.56  &0.96  &1.76  &2.80  &3.52  &3.6  &2.24  &2.715  \\
        \hline
    \end{tabular}
\end{table}

\subsubsection{任务三:带通滤波器频响特性测量结果}
实验结果如下表\ref{tab:4-3a-res}和表\ref{tab:4-3b-res}所示。
\begin{table}[htbp]
    \centering
    \caption{带通无源滤波器逐点测量法}
    \label{tab:4-3a-res}
    \begin{tabular}{|c|c|c|c|c|c|c|c|c|c|}
        \hline
        $V_I(\mathrm{V})$ & 4 & 4 & 4 & 4 & 4 & 4 & 4 & 4  & 截止频率($f_L, f_H$) \\
        \hline
        $f(\mathrm{Hz})$ &1k  &2k  &4k  &8k  &16k  &32k  &64k  &128k  &1.21k, 24.6k  \\
        \hline
        $V_O(\mathrm{V})$ &2.08  &2.8  &3.2  &3.2  &2.72  &1.92  &1.2  &0.72  &2.624  \\
        \hline
    \end{tabular}
\end{table}

\begin{table}[htbp]
    \centering
    \caption{带通有源滤波器逐点测量法}
    \label{tab:4-3b-res}
    \begin{tabular}{|c|c|c|c|c|c|c|c|c|c|}
        \hline
        $V_I(\mathrm{V})$ & 4 & 4 & 4 & 4 & 4 & 4 & 4 & 4  & 截止频率($f_L, f_H$) \\
        \hline
        $f(\mathrm{Hz})$ &1k  &2k  &4k  &8k  &16k  &32k  &64k  &128k  &2.31k, 24.5k  \\
        \hline
        $V_O(\mathrm{V})$ &1.28  &2.08  &2.96  &3.2  &2.8  &1.92  &1.12  &0.64  &2.624  \\
        \hline
    \end{tabular}
\end{table}
\subsubsection{任务四:带阻滤波器频响特性测量结果}
实验结果如下表\ref{tab:4-4a-res}和表\ref{tab:4-4b-res}所示。
\begin{table}[htbp]
    \centering
    \caption{带阻无源滤波器逐点测量法}
    \label{tab:4-4a-res}
    \begin{tabular}{|c|c|c|c|c|c|c|c|c|c|}
        \hline
        $V_I(\mathrm{V})$ & 4 & 4 & 4 & 4 & 4 & 4 & 4 & 4  & 截止频率($f_L, f_H$) \\
        \hline
        $f(\mathrm{Hz})$ &1k  &2k  &4k  &8k  &16k  &32k  &64k  &128k  &4.31k, 65.3k  \\
        \hline
        $V_O(\mathrm{V})$ &3.76  &3.44  &2.72  &1.52  &0.32  &1.44  &2.56  &3.44  &2.658  \\
        \hline
    \end{tabular}
\end{table}

\begin{table}[htbp]
    \centering
    \caption{带阻有源滤波器逐点测量法}
    \label{tab:4-4b-res}
    \begin{tabular}{|c|c|c|c|c|c|c|c|c|c|}
        \hline
        $V_I(\mathrm{V})$ & 4 & 4 & 4 & 4 & 4 & 4 & 4 & 4  & 截止频率($f_L, f_H$) \\
        \hline
        $f(\mathrm{Hz})$ &1k  &2k  &4k  &8k  &16k  &32k  &64k  &128k  &6.51k, 43.2k  \\
        \hline
        $V_O(\mathrm{V})$ &3.92  &3.84  &3.52  &2.4  &0.32  &2.32  &3.2  &2.24  &2.771  \\
        \hline
    \end{tabular}
\end{table}

\subsubsection{各个任务的扫频法测量结果}
各个任务的扫频法测量结果如图\ref{fig:4-5-res}
\begin{figure}[H]
  \begin{minipage}{0.22\textwidth}
    \centering
    \includegraphics[width=\textwidth]{4-5-1.png}
    \caption{无源低通滤波器扫频法测量结果}
  \end{minipage}
  \begin{minipage}{0.22\textwidth}
    \centering
    \includegraphics[width=\textwidth]{4-5-2.png}
    \caption{有源低通滤波器扫频法测量结果}
  \end{minipage}
  \begin{minipage}{0.22\textwidth}
    \centering
    \includegraphics[width=\textwidth]{4-5-4.png}
    \caption{无源高通滤波器扫频法测量结果}
  \end{minipage}
  \begin{minipage}{0.22\textwidth}
    \centering
    \includegraphics[width=\textwidth]{4-5-5.png}
    \caption{有源高通滤波器扫频法测量结果}
  \end{minipage}
  \begin{minipage}{0.23\textwidth}
    \centering
    \includegraphics[width=\textwidth]{4-5-3.png}
    \caption{无源带通滤波器扫频法测量结果}
  \end{minipage}
  \begin{minipage}{0.25\textwidth}
    \centering
    \includegraphics[width=\textwidth]{4-5-6.png}
    \caption{有源带通滤波器扫频法测量结果}
  \end{minipage}
  \begin{minipage}{0.25\textwidth}
    \centering
    \includegraphics[width=\textwidth]{4-5-7.png}
    \caption{无源带阻滤波器扫频法测量结果}
  \end{minipage}
  \begin{minipage}{0.25\textwidth}
    \centering
    \includegraphics[width=\textwidth]{4-5-8.png}
    \caption{有源带阻滤波器扫频法测量结果}
  \end{minipage}
  \caption{各任务扫频法测量结果}
  \label{fig:4-5-res}
\end{figure}

\subsection{实验程序及运行结果}
\subsubsection{问题一程序及运行结果}

\begin{lstlisting}[language=Matlab, 
                  caption={信号波形生成与绘图代码}, 
                        % 指定代码语言(此处为MATLAB)
  basicstyle=\ttfamily\small,  % 等宽字体+小字号(Markdown常用)
  backgroundcolor=\color{gray!5},  % 浅灰背景(接近多数Markdown渲染效果)
  frame=none,           % 无边框(Markdown代码块通常无框)
  keywordstyle=\color{blue},  % 关键字高亮(蓝色,可选)
  commentstyle=\color{green},  % 注释高亮(绿色,可选)
  showstringspaces=false,  % 不显示字符串中的空格标记
  numbers=none,         % 不显示行号(Markdown默认无行号)
  breaklines=true,      % 自动换行(避免溢出)
  columns=fullflexible  ] 


dt = 0.01;
t = -5:0.01:5;
ft = sin(2.*pi.*(t-1))./(pi.*(t-1));
ft(isnan(ft)) = 2;  % 替换t=1处的NaN为极限值2

domega = 0.01;
omega = -10:domega:10;
Fft = dt*ft*exp(-1j*t'*omega);


figure;
subplot(2,1,1);
plot(t, ft);
title('Time Domain Signal f(t)');

subplot(2,1,2);
plot(omega, abs(Fft));
title('Frequency Domain Signal |F(\omega)|');

saveas(gcf, './p1_result.png');

\end{lstlisting}

运行结果如下图\ref{fig:p1_result}所示。
\begin{figure}[H]
  \centering
  \includegraphics[width=0.7\textwidth]{./matlab/p1_result.png}
  \caption{问题一运行结果}
  \label{fig:p1_result}
\end{figure}

\subsubsection{问题二程序及运行结果}

\begin{lstlisting}[language=Matlab, 
                  caption={信号波形生成与绘图代码}, 
                        % 指定代码语言(此处为MATLAB)
  basicstyle=\ttfamily\small,  % 等宽字体+小字号(Markdown常用)
  backgroundcolor=\color{gray!5},  % 浅灰背景(接近多数Markdown渲染效果)
  frame=none,           % 无边框(Markdown代码块通常无框)
  keywordstyle=\color{blue},  % 关键字高亮(蓝色,可选)
  commentstyle=\color{green},  % 注释高亮(绿色,可选)
  showstringspaces=false,  % 不显示字符串中的空格标记
  numbers=none,         % 不显示行号(Markdown默认无行号)
  breaklines=true,      % 自动换行(避免溢出)
  columns=fullflexible  ] 

w = -10:0.01:10;
dw = 0.01;
Fw = -1j*2*w./(16+w.^2);

%inverse Fourier Transform
t = -5:0.01:5;
dt = 0.01;
ft = (1/(2*pi))*dw*Fw*exp(1j*w'*t);

figure;
subplot(2,1,1);
plot(w, abs(Fw));
title('Frequency Domain Signal |F(\omega)|');

subplot(2,1,2);
plot(t, real(ft));
title('Time Domain Signal f(t)');

saveas(gcf, './p2_result.png');

\end{lstlisting}

运行结果如下图\ref{fig:p2_result}所示。

\begin{figure}
  \centering
  \includegraphics[width=0.7\textwidth]{./matlab/p2_result.png}
  \caption{问题二运行结果}
  \label{fig:p2_result}
\end{figure}

\subsubsection{任务三程序及运行结果}


\begin{lstlisting}[language=Matlab, 
                  caption={信号波形生成与绘图代码}, 
                        % 指定代码语言(此处为MATLAB)
  basicstyle=\ttfamily\small,  % 等宽字体+小字号(Markdown常用)
  backgroundcolor=\color{gray!5},  % 浅灰背景(接近多数Markdown渲染效果)
  frame=none,           % 无边框(Markdown代码块通常无框)
  keywordstyle=\color{blue},  % 关键字高亮(蓝色,可选)
  commentstyle=\color{green},  % 注释高亮(绿色,可选)
  showstringspaces=false,  % 不显示字符串中的空格标记
  numbers=none,         % 不显示行号(Markdown默认无行号)
  breaklines=true,      % 自动换行(避免溢出)
  columns=fullflexible  ] 

t = -2:0.01:2;
ft = (t+2).*(t<-1) + 1*(t<=1&t>=-1) + (-(t-2)).*(t>1);

%Fourier Transform
dt = 0.01;
dw = 0.01;
w = -10:0.01:10;
Fw = dt*ft*exp(-1j*t'*w);

figure;
subplot(3,1,1);
plot(t, ft);
title('Time Domain Signal f(t)');
subplot(3,1,2);
plot(w, abs(Fw));
title('Frequency Domain Signal |F(\omega)|');

%Inverse Fourier Transform
ft_reconstructed = (1/(2*pi))*dw*Fw*exp(1j*w'*t);
subplot(3,1,3);
plot(t, real(ft_reconstructed));
title('Reconstructed Time Domain Signal f(t) from F(\omega)');

saveas(gcf, './p3_result.png');

\end{lstlisting}

运行结果如下图\ref{fig:p3_result}所示。

\begin{figure}[H]
  \centering
  \includegraphics[width=0.7\textwidth]{./matlab/p3_result.png}
  \caption{任务三运行结果}
  \label{fig:p3_result}
\end{figure}

\subsubsection{任务四程序及运行结果}


\begin{lstlisting}[language=Matlab, 
                  caption={信号波形生成与绘图代码}, 
                        % 指定代码语言(此处为MATLAB)
  basicstyle=\ttfamily\small,  % 等宽字体+小字号(Markdown常用)
  backgroundcolor=\color{gray!5},  % 浅灰背景(接近多数Markdown渲染效果)
  frame=none,           % 无边框(Markdown代码块通常无框)
  keywordstyle=\color{blue},  % 关键字高亮(蓝色,可选)
  commentstyle=\color{green},  % 注释高亮(绿色,可选)
  showstringspaces=false,  % 不显示字符串中的空格标记
  numbers=none,         % 不显示行号(Markdown默认无行号)
  breaklines=true,      % 自动换行(避免溢出)
  columns=fullflexible  ] 

t = -5:0.01:5;
dt = 0.01;
ft = heaviside(t+0.5) - heaviside(t-0.5);
% f(t/2)
ft2 = heaviside(t/2+0.5) - heaviside(t/2-0.5);
% f(2t)
ft3 = heaviside(2*t+0.5) - heaviside(2*t-0.5);
w = -10:0.01:10;
 dw = 0.01;
% Fourier Transform
% ft
Fft = dt*ft*exp(-1j*t'*w);
Fft2 = dt*ft2*exp(-1j*t'*w);
Fft3 = dt*ft3*exp(-1j*t'*w);

figure;
subplot(2,2,1);
plot(t,ft);
title('Time Domain Signal f(t)');

subplot(2,2,2);
plot(w, abs(Fft));
title('Frequency Domain Signal |F(\omega)|');

subplot(2,2,3);
plot(w, abs(Fft2));
title('Frequency Domain Signal |F_{f(t-2)}(\omega)|');

subplot(2,2,4);
plot(w, abs(Fft3));
title('Frequency Domain Signal |F_{f(2t)}(\omega)|');

saveas(gcf, './p4_result.png');

\end{lstlisting}

运行结果如下图\ref{fig:p4_result}所示。
\begin{figure}[H]
  \centering
  \includegraphics[width=0.7\textwidth]{./matlab/p4_result.png}
  \caption{任务四运行结果}
  \label{fig:p4_result}
\end{figure}

\subsubsection{实验五程序及运行结果}

\begin{lstlisting}[language=Matlab, 
                  caption={信号波形生成与绘图代码}, 
                        % 指定代码语言(此处为MATLAB)
  basicstyle=\ttfamily\small,  % 等宽字体+小字号(Markdown常用)
  backgroundcolor=\color{gray!5},  % 浅灰背景(接近多数Markdown渲染效果)
  frame=none,           % 无边框(Markdown代码块通常无框)
  keywordstyle=\color{blue},  % 关键字高亮(蓝色,可选)
  commentstyle=\color{green},  % 注释高亮(绿色,可选)
  showstringspaces=false,  % 不显示字符串中的空格标记
  numbers=none,         % 不显示行号(Markdown默认无行号)
  breaklines=true,      % 自动换行(避免溢出)
  columns=fullflexible  ] 

b = [0,1j,0];
a = [-1,1j,100];
w = -10:0.01:10;
res = freqs(b,a,w);

figure;
subplot(2,1,1);
plot(w, abs(res));
title('Frequency Domain Signal |F(\omega)|');

subplot(2,1,2);
plot(w, angle(res));
title('Frequency Domain Signal Phase \angleF(\omega)');

saveas(gcf, './p5_result.png');


\end{lstlisting}

运行结果如下图\ref{fig:p5_result}所示。
\begin{figure}[H]
  \centering
  \includegraphics[width=0.7\textwidth]{./matlab/p5_result.png}
  \caption{任务五运行结果}
  \label{fig:p5_result}
\end{figure}

\subsubsection{实验六程序及运行结果}

\begin{lstlisting}[language=Matlab, 
                  caption={信号波形生成与绘图代码}, 
                        % 指定代码语言(此处为MATLAB)
  basicstyle=\ttfamily\small,  % 等宽字体+小字号(Markdown常用)
  backgroundcolor=\color{gray!5},  % 浅灰背景(接近多数Markdown渲染效果)
  frame=none,           % 无边框(Markdown代码块通常无框)
  keywordstyle=\color{blue},  % 关键字高亮(蓝色,可选)
  commentstyle=\color{green},  % 注释高亮(绿色,可选)
  showstringspaces=false,  % 不显示字符串中的空格标记
  numbers=none,         % 不显示行号(Markdown默认无行号)
  breaklines=true,      % 自动换行(避免溢出)
  columns=fullflexible  ] 

%0.2*dy/dt + y = x
%%系统传递函数H(jw) = Y(jw)/X(jw) = 1/(0.2jw+1)
x(t) = u(t) - u(t-1)
t = -5:0.01:5;
sys = tf([0,0,1],[0,0.2,1]);
xt = heaviside(t) - heaviside(t-1);

figure;
%频域分析 数值法
dt = 0.01;
w = -10:0.01:10;
dw = 0.01;
Xw = dt*xt*exp(-1j*t'*w);
Hw = freqs([0,0,1],[0,0.2,1],w);
Yw = Hw.*Xw;
yt = (1/(2*pi))*dw*Yw*exp(1j*w'*t);
%time domain output
subplot(2,1,1);
plot(t, abs(yt));
title('Time Domain Signal y(t) from Frequency Domain Analysis');

%时域分析
yt2 = lsim(sys, xt, t);
subplot(2,1,2);
plot(t, yt2);
title('Time Domain Signal y(t) from Time Domain Analysis');

saveas(gcf, './p6_result.png');


\end{lstlisting}

实验结果如下图\ref{fig:p6_result}所示。

\begin{figure}[H]
  \centering
  \includegraphics[width=0.7\textwidth]{./matlab/p6_result.png}
  \caption{任务六运行结果}
  \label{fig:p6_result}
\end{figure}




\end{document}

